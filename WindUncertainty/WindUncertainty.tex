%% LyX 2.1.2 created this file.  For more info, see http://www.lyx.org/.
\documentclass[12pt,twoside,english]{article}\usepackage[]{graphicx}\usepackage[]{color}
%% maxwidth is the original width if it is less than linewidth
%% otherwise use linewidth (to make sure the graphics do not exceed the margin)
\makeatletter
\def\maxwidth{ %
  \ifdim\Gin@nat@width>\linewidth
    \linewidth
  \else
    \Gin@nat@width
  \fi
}
\makeatother

\definecolor{fgcolor}{rgb}{0.345, 0.345, 0.345}
\newcommand{\hlnum}[1]{\textcolor[rgb]{0.686,0.059,0.569}{#1}}%
\newcommand{\hlstr}[1]{\textcolor[rgb]{0.192,0.494,0.8}{#1}}%
\newcommand{\hlcom}[1]{\textcolor[rgb]{0.678,0.584,0.686}{\textit{#1}}}%
\newcommand{\hlopt}[1]{\textcolor[rgb]{0,0,0}{#1}}%
\newcommand{\hlstd}[1]{\textcolor[rgb]{0.345,0.345,0.345}{#1}}%
\newcommand{\hlkwa}[1]{\textcolor[rgb]{0.161,0.373,0.58}{\textbf{#1}}}%
\newcommand{\hlkwb}[1]{\textcolor[rgb]{0.69,0.353,0.396}{#1}}%
\newcommand{\hlkwc}[1]{\textcolor[rgb]{0.333,0.667,0.333}{#1}}%
\newcommand{\hlkwd}[1]{\textcolor[rgb]{0.737,0.353,0.396}{\textbf{#1}}}%

\usepackage{framed}
\makeatletter
\newenvironment{kframe}{%
 \def\at@end@of@kframe{}%
 \ifinner\ifhmode%
  \def\at@end@of@kframe{\end{minipage}}%
  \begin{minipage}{\columnwidth}%
 \fi\fi%
 \def\FrameCommand##1{\hskip\@totalleftmargin \hskip-\fboxsep
 \colorbox{shadecolor}{##1}\hskip-\fboxsep
     % There is no \\@totalrightmargin, so:
     \hskip-\linewidth \hskip-\@totalleftmargin \hskip\columnwidth}%
 \MakeFramed {\advance\hsize-\width
   \@totalleftmargin\z@ \linewidth\hsize
   \@setminipage}}%
 {\par\unskip\endMakeFramed%
 \at@end@of@kframe}
\makeatother

\definecolor{shadecolor}{rgb}{.97, .97, .97}
\definecolor{messagecolor}{rgb}{0, 0, 0}
\definecolor{warningcolor}{rgb}{1, 0, 1}
\definecolor{errorcolor}{rgb}{1, 0, 0}
\newenvironment{knitrout}{}{} % an empty environment to be redefined in TeX

\usepackage{alltt}
\usepackage{mathptmx}
\usepackage{helvet}
\renewcommand{\ttdefault}{lmtt}
\usepackage[T1]{fontenc}
\usepackage[latin9]{inputenc}
\usepackage[letterpaper]{geometry}
\geometry{verbose,tmargin=1in,bmargin=1in,lmargin=1.2in,rmargin=1in}
\usepackage{fancyhdr}
\pagestyle{fancy}
\setcounter{tocdepth}{2}
\setlength{\parskip}{\medskipamount}
\setlength{\parindent}{0pt}
\setlength{\headheight}{14.5pt}
\usepackage{color}
\definecolor{page_backgroundcolor}{rgb}{1, 1, 1}
\pagecolor{page_backgroundcolor}
\usepackage{babel}
\usepackage{array}
\usepackage{prettyref}
\usepackage{float}
\usepackage{booktabs}
\usepackage{calc}
\usepackage{ifthen}
\usepackage{amsmath}
\usepackage{graphicx}
\usepackage{esint}
\usepackage[authoryear]{natbib}
\PassOptionsToPackage{normalem}{ulem}
\usepackage{ulem}
\usepackage[unicode=true,
 bookmarks=true,bookmarksnumbered=false,bookmarksopen=false,
 breaklinks=true,pdfborder={0 0 0},backref=false,colorlinks=true]
 {hyperref}
\hypersetup{pdftitle={Technical Note: Uncertainty in Wind Measurements},
 pdfauthor={RAF},
 pdfsubject={characterization of the uncertainty in wind measurements, GV},
 pdfkeywords={wind,uncertainty, NCAR Research Aviation Facility, research aircraft, NCAR/EOL/RAF}}
\usepackage{breakurl}
\usepackage{html}

\makeatletter

%%%%%%%%%%%%%%%%%%%%%%%%%%%%%% LyX specific LaTeX commands.
%% Because html converters don't know tabularnewline
\providecommand{\tabularnewline}{\\}

%%%%%%%%%%%%%%%%%%%%%%%%%%%%%% Textclass specific LaTeX commands.
\newenvironment{lyxcode}
{\par\begin{list}{}{
\setlength{\rightmargin}{\leftmargin}
\setlength{\listparindent}{0pt}% needed for AMS classes
\raggedright
\setlength{\itemsep}{0pt}
\setlength{\parsep}{0pt}
\normalfont\ttfamily}%
 \item[]}
{\end{list}}
 \newlength{\lyxhang}
 \IfFileExists{hanging.sty}{
   \usepackage{hanging}
   \newenvironment{hangparagraphs}
     {%
       \ifthenelse{\lengthtest{\parindent > 0pt}}%
         {\setlength{\lyxhang}{\parindent}}%
         {\setlength{\lyxhang}{2em}}%
       \par\begin{hangparas}{\lyxhang}{1}%
     }
     {\end{hangparas}}
 }{%else
   \newenvironment{hangparagraphs}
     {%
       \ifthenelse{\lengthtest{\parindent > 0pt}}%
         {\setlength{\lyxhang}{\parindent}}%
         {\setlength{\lyxhang}{2em}}%
       \begin{hangparas}%
     }
     {\end{hangparas}}
   \newcommand{\hangpara}{\hangindent \lyxhang \hangafter 1 \noindent}
   \newenvironment{hangparas}{\setlength{\parindent}{\z@}
   \everypar={\hangpara}}{\par}
 }

%%%%%%%%%%%%%%%%%%%%%%%%%%%%%% User specified LaTeX commands.


\definecolor{fgcolor}{rgb}{0.345, 0.345, 0.345}\definecolor{messagecolor}{rgb}{0, 0, 0}\definecolor{warningcolor}{rgb}{1, 0, 1}\definecolor{errorcolor}{rgb}{1, 0, 0}\usepackage{babel}
% macro for italic page numbers in the index
\newcommand{\IndexDef}[1]{\textit{#1}}
\newcommand{\IndexPrimary}[1]{\textbf{#1}}
% force a page break at the start of sections
\let\stdsection\section
\renewcommand\section{\newpage\stdsection}


% workaround for a makeindex bug,
% see sec. "Index Entry Order"
% only uncomment this when you are using makindex
%\let\OrgIndex\index 
%\renewcommand*{\index}[1]{\OrgIndex{#1}}
%\usepackage{splitidx}

\AtBeginDocument{
  \def\labelitemii{\(\circ\)}
  \def\labelitemiii{\(\triangleright\)}
}


\IfFileExists{upquote.sty}{\usepackage{upquote}}{}

\newenvironment{lyxlist}[1]
{\begin{list}{}
{\settowidth{\labelwidth}{#1}
\setlength{\leftmargin}{\labelwidth}
\addtolength{\leftmargin}{\labelsep}
\renewcommand{\makelabel}[1]{##1\hfil}}}
{\end{list}}
\newcommand{\datetoday}{\number\day\space
     \ifcase\month\or January\or February\or March\or April\or May\or
     June\or July\or August\or September\or October\or November\or
     December\fi
     \space\number\year}
\newcommand{\EOLmemo}{\null \vskip-1.5truein
{\raggedright \textsf{\textsc{\large \textcolor{blue}{Earth Observing Laboratory}}}}\par
{\raggedright \textsf{\textsl{\textcolor{blue}{Memorandum:}}}} \par \vskip6pt
{\color{blue}{\hrule}}\par
\vskip0.3truein \leftline{\hskip \longindent \datetoday} \vskip0.2truein
\thispagestyle{empty}}
\newcommand{\attachm}[1]{\begin{lyxlist}{Attachments:00}
\item [Attachments:] {#1}
\end{lyxlist}}
\newcommand{\cc}[1]{\begin{lyxlist}{Attachments:00}
\item [cc:] {#1}
\end{lyxlist}}
\newcommand{\attach}[1]{\begin{lyxlist}{Attachments:00}
\item [Attachment:] {#1}
\end{lyxlist}}
\makeatother
\IfFileExists{upquote.sty}{\usepackage{upquote}}{}
\begin{document}

\title{Characterization of Uncertainty \\
 in Measurements of Wind\\
 from the NSF/NCAR Gulfstream V Research Aircraft}


\author{Al Cooper, Dick Friesen, Matt Hayman, Jorgen Jensen, Don Lenschow, Allen Schanot, \\Scott Spuler, Jeff Stith\\(and, we hope, many others)}


\date{\textcolor{red}{DRAFT} 4/6/2015}

\maketitle
\vfill\eject
\tableofcontents{}
\vfill\eject



%% LyX 2.1.2 created this file.  For more info, see http://www.lyx.org/.




\section{Introduction}

Wind is the motion of the atmosphere relative to the Earth. Most reseearch
aircraft have the capability to measure wind, and these measurements have
many uses in research projects using aircraft. They help define the
flow and so provide
context for other measurements, and they are often used to study fluxes 
of atmospheric constituents, turbulence, wave motions, cloud updrafts and 
downdrafts, convergence and divergence, and many other topics. They can
provide important information transferred to models for data assimilation or
for validation tests of model results.

This report applies to the Gulfstream V research aircraft owned by the
National Science Foundation and operated by the Research Aviation Facility
(RAF), Earth Observing Laboratory (EOL), National Center for Atmospheric
Research (NCAR). This aircraft is referred to here either as the NSF/NCAR GV or
simply the GV.
Its range and endurance makes it
possible to measure wind over large distances and so to characterize mesoscale
and even larger features in the atmosphere. However, its high speed 
(typically Mach 0.8, or about 240\,m\,s$^{-1}$ for flight near 40,000 ft)
poses special problems for wind measurement. The flow distortion around the aircraft
perturbs pressure measurements that are central to the measurement of wind, and
the measurement of air temperature, needed in the calculation of wind, requires
corrections of typically more than 20$^{\circ}$C to account for dynamic heating of
the sensors. Accurate measurement of wind thus is particularly challenging on
this and other high-speed aircraft. 

This report documents how measurements of wind are made from the GV research aircraft and provides a characterization
of the uncertainty associated with those measurements. The characterization
applies to the system as it existed in 2014, in particular as it operated
in the DEEPWAVE research project flown from New Zealand in June-July
2014. Key features of the instrumentation influencing the uncertainty in the measurements
as characterized here are the presence of ``OmniSTAR'' GPS (Global Positioning
System) measurements,
providing measurements of the velocity of the aircraft relative to
the earth, a calibration of airspeed provided by the Laser Air Motion
Sensor (\cite{CooperEtAl2014}), and a newly developed all-weather wind
sensor or ``gust pod'' employing a Rosemount 858 probe mounted under
the wing of the GV. These complemented the standard wind-sensing system
comprised of a Honeywell Laseref IV inertial reference unit,
GPS measurements from Novatel and Garmin units,
and a gust-sensing system based on pressure ports in the nose radome.
The results obtained here do not necessarily apply to measurements
from earlier projects when not all these components were available in
their present form, but they should apply to measurements subsequent to 2014.

The intent in this report is to follow the conventions established by
the International Committee on Weights and Measures and by the
National Institute of Standards and Technology. Appendix A summarizes
key aspects of those recommendations and how they are addressed in this
report. This report also contains additional information resulting from various
studies of the measurements that have tested the validity of the
measurements or have been used for calibration. 

The organization is as follows. Section \ref{sec:Components} describes the 
components of the wind-measuring
system in more detail, with examples of the measurements and information on
the specifications for the sensors involved. 
That section is followed by a summary of the uncertainty in wind
measurements (Sect.~\ref{sec:Uncertainty-components}), with a tabulation of 
individual error sources and many
references to other parts of the document where studies 
have led to estimates of limits on those error sources. It is
our hope that this section will provide a summary of the results that
can either stand alone or provide a guide to the further information
in this report.

Each of the subsequent sections provides information used to assess
uncertainty. Section \ref{sec:Calibrations} describes
how the key gust measurements have been calibrated, tested, and intercompared.
Section \ref{sec:vertical-wind} discusses the choice of measurement to
represent the vertical motion of the aircraft and some aspects of the
uncertainty in that measurement, the relative timing of acquisition of the
measurements entering vertical wind, and a procedure for detecting the
Schuler oscillation in the pitch measurement and applying a correction that
reduces the uncertainty in that measurement that, uncorrected, accounts for
the dominant uncertainty in vertical wind. Section \ref{sec:HWind} then
uses drifting or Lagrangian circle maneuvers to establish limits on some
critical components entering measurement of the horizontal wind, the true
airspeed and offsets in heading and sideslip, which account for some of the
bias limits used in Sect.~\ref{sec:Uncertainty-components}, and it describes
how the measurements of ground-speed components from the inertial units and
GPS units have been combined to reduce uncertainty in the components of the
horizontal wind. Finally, Appendix B presents some of the characteristics of
turbulence measurements, including variance spectra and potential to measure
fluxes, and notes some limitations of such measurements.

% The program that performed the calculations reported here
% can be found on NCAR/EOL computer
% space in the directory /h/eol/cooperw/RStudio/DEEPWAVE/WindUncertainty 
% and in the GitHub repository \href{https://github.com/WilliamCooper/WindUncertainty.git}{https://github.com/WilliamCooper/WindUncertainty.git}.
% The former is accessible via computers like tikal.eol.ucar.edu.
% The main program is WindUncertainty.Rnw; other files in those repositories
% with suffixes .Rnw are sections in this report and are linked to the
% main program. Normal usage is to run this main program within ''RStudio''
% to execute the contained R code and also, via ''knitr,'' to generate
% this text document from LaTeX statements embedded in those files.
% An R package 'Ranadu' is used extensively in the R code; it resides on GitHub as \href{https://github.com/WilliamCooper/Ranadu.git}{https://github.com/WilliamCooper/Ranadu.git}.
% The data used reside in NCAR/EOL project archives and, in subsetted
% form, are archived in the directory /h/eol/cooperw/RStudio/DEEPWAVE/WindUncertainty
% as R data files, with names ending in ''Rdata.'' Those files are
% too large to be appropriate for GitHub but can be provided from the
% NCAR/EOL computers. It is thus possible to reproduce this document
% from archived data and to repeat the included analyses with new data
% as needed. This document thus attempts to be ''reproducible research''
% as that term is used by the author of knitr. References for knitr
% and the analysis packages in R are included in the \nameref{sec:acknowledgements} and References at the end of this document.



\section{Components of the wind-sensing systems\label{sec:Components}}

\subsection{General comments\label{sub:General-comments}}

\subsubsection{Overview}

Three wind-sensing systems are available for use on the GV:
\begin{enumerate}
\item The ``standard'' wind sensing system that uses pressure ports on
the radome combined with airspeed measured using a pitot tube and
ground-speed measured by an inertial reference unit and a global positioning
system (GPS) receiver. 
\item A gust-pod system consisting of a Rosemount 858 airflow sensor mounted
under the wing combined with an inertial system co-located with the
airflow sensor and linked to GPS measurements via a Kalman filter. 
\item A laser air motion sensor (LAMS) described by \cite{SpulerEtAl2011} combined
with its own IRU/GPS system. See also \cite{CooperEtAl2014}.
\end{enumerate}
All three also depend on a measurement of temperature and, for the first
two, humidity which is used to correct for the influence of moisture
on the specific heats and gas constant of moist air. 
Those two also share dependence on the measurement of ambient pressure
as delivered by static sources on the fuselage.
Although the other measurement components differ, for each of these the
measurement of wind involves the vector sum of two components, the
motion of the air relative to the aircraft and the motion of the aircraft
relative to the Earth. The former is the ``relative wind'' and is
measured as a three-component vector having magnitude equal to the
``true airspeed'' and angles relative to the aircraft reference
frame characterized by the angle of attack and the sideslip angle.
The angle of attack is considered positive if the relative wind is
from below the aircraft, and the sideslip angle is considered positive
if the relative wind is from the starboard side of the aircraft. The
relative wind defined in the coordinate system of the aircraft (conventionally
with $\hat{x^{\prime}}$ forward, $\hat{y^{\prime}}$ in the starboard
direction, and $\hat{z^{\prime}}$ obtained from the cross product
$\hat{x^{\prime}}\times\hat{y^{\prime}}$ and so approximately downward
but oriented to align with the aircraft reference frame) must be transformed
to an Earth-based reference system (conventionally with $\hat{x}$
toward east, $\hat{y}$ toward north, and $\hat{z}$ upward) so that
the components can be combined to yield the Earth-relative wind. This
transformation is a function of the attitude angles of the aircraft
(heading $\psi$, pitch $\theta$, and roll $\phi$), measured
by an inertial reference unit in all three cases discussed here. Once
in an Earth-based reference system, the relative wind vector is added
to the vector representing the aircraft motion relative to the Earth
to obtain the wind. The sources of the various measurements entering
this processing sequence vary among the three systems and will be
discussed separately below for each system.


\subsubsection{The Relative Wind\label{sub:The-Relative-Wind}}

In the standard aircraft coordinate system with $x$ forward, $y$
starboard, and $z$ downward, the three corresponding components of
the relative wind $\mathbf{v}$ (cf.~\cite{NCAR_OpenSky_TECH-NOTE-000-000-000-064}
and \href{https://www.eol.ucar.edu/raf/Bulletins/bulletin23.html}{RAF Bulletin 23})
are:

\begin{equation}
\mathbf{v=}\begin{pmatrix}u_{r}\\
v_{r}\\
w_{r}
\end{pmatrix}=\begin{pmatrix}V^{*}\\
V^{*}\thinspace\tan\beta\\
V^{*}\thinspace\tan\alpha
\end{pmatrix}\label{eq:relative wind}
\end{equation}
where, if $V$ is the true airspeed, $V^{*}=$ $V/\sqrt{1+\tan^{2}\alpha+\tan^{2}\beta}$
is the component of true airspeed along the aircraft longitudinal
($x$) axis, $\alpha$ is the angle of attack and $\beta$ the sideslip
angle. The sign convention is such that the relative wind is positive
when \emph{from }the direction of the axis for each component. (The
magnitude of $\mathbf{v}$ is thus $V$ as required.) The
relative wind is therefore determined from measurements of true airspeed,
angle of attack, and sideslip angle.


\subsubsection{Transformation to an Earth reference frame\label{sub:EarthRef}}

The orientations of the aircraft, the gust pod, and the LAMS are measured
by IRUs located respectively in the fuselage and in the pod itself.
Each independently measures heading, pitch, and roll, so the calculations
of wind from the three systems can be fully independent except that,
because it is considered to have the smallest uncertainty, the true airspeed
measured from the fuselage system is used also for the gust pod. 
In each case, the IRU measurements and GPS ground-speed components are
used to transform the measurements to the reference frame of the Earth.

The required transformation is described by three rotation matrices,
defined in \href{https://www.eol.ucar.edu/raf/Bulletins/bulletin23.html}{RAF Bulletin 23} Eqs. 2.5 and 2.6:\footnote{An additional correction is applied to account for the
effect of the rotation rate of the aircraft on the measurements. This correction is
needed when the reference unit for motion relative to the Earth, the IRU, is 
separated from the measurement of relative wind on the radome or gust pod. For
the gust pod, this is negligible because the IRU is co-located with the gust-measuring system. See the cited reference for details.}

\[
\mathbf{T_{1}}=\left(\begin{array}{ccc}
1 & 0 & 0\\
0 & \cos\phi & -\sin\phi\\
0 & \sin\phi & \cos\phi
\end{array}\right)
\]


\[
\mathbf{T_{2}}=\left(\begin{array}{ccc}
\cos\theta & 0 & \sin\theta\\
0 & 1 & 0\\
-\sin\theta & 0 & \cos\theta
\end{array}\right)
\]\label{eq:rotation-matrices}


\[
\mathbf{T_{3}}=\left(\begin{array}{ccc}
\cos\psi & -\sin\psi & 0\\
\sin\psi & \cos\psi & 0\\
0 & 0 & 1
\end{array}\right)
\]


where \{$\phi,\,\theta,\,\psi$\} are \{roll, pitch, heading\}.

The transformation needs to be in the following order to conform to
conventional definitions of the attitude angles: 
\begin{enumerate}
\item Rotate by $\mathbf{T_{1}}$ using the roll angle $\phi$ (ROLL or CROLL\_GP)
to level the wings by a rotation about the x axis. 
\item Rotate by $\mathbf{T_{2}}$ using the pitch angle $\theta$ (PITCH or CPITCH\_GP)
to level the aircraft by a rotation about the y axis. 
\item Rotate by $\mathbf{T_{3}}$ using the heading angle $\psi$ (THDG or CTHDG\_GP)
to obtain components in a true-north reference frame. At this point,
the relative-wind vector in an Earth-reference coordinate system is
$\mathbf{v}_{r}=\mathbf{T_{3}}(\mathbf{T_{2}}(\mathbf{T_{1}}\mathbf{v}))$ where $\mathbf{v}$
is given by (\ref{eq:relative wind}). 
\end{enumerate}
The measured ground-speeds (with components VNS, VEW, VSPD) then can be added to the
relative wind to get the true Earth-relative wind. In the ``R''
file associated with this document, the required transformations are
coded to provide a processing option, but the RAF ``nimbus''
routine ``gust.c'' provides the transformation as implemented in
standard processing.

The final equations, defining the Earth-relative wind $\mathbf{v}_{E}$
in terms of the three wind variables \{WDx, WSx, WIx\} where x denotes
the measuring system (radome, gust-pod, or LAMS) and subscripts $_x$ or $_y$ the respective
east or north component of the wind, are:

\begin{equation}
\mathbf{v}_{E}=\mathbf{v}_{r}+\left(\begin{array}{c}
-\mathrm{VNSx}\\
-\mathrm{VEWx}\\
\mathrm{VSPDx}
\end{array}\right)\label{eq:vg}
\end{equation}
\begin{equation}
\mathrm{WDx=\arctan2(v_{E,y},}v_{E,x})\label{eq:wd}
\end{equation}
\begin{equation}
\mathrm{WSx=\sqrt{(v_{E,x}^{2}+v_{E,y}^{2})}}\label{eq:ws}
\end{equation}
\begin{equation}
\mathrm{WIx=v_{E,z}}\label{eq:wi}
\end{equation}



\subsection{The radome-based system\label{sub:The-radome-based-system}}


\subsubsection{Overview}

The primary measurement of wind on the GV is that based on measurement of
true airspeed via a pitot tube, airflow angles via pressure differences
measured on the nose radome, attitude angles measured by an inertial
reference unit, and ground-speed components measured by the same inertial
reference unit and also by a Global Positioning System receiver. A
cursory description of this system was provided by \cite{CooperEtAl2014}.
A more extensive description will be provided here. Table~\ref{tab:Radome-system-measurements}
provides a summary of the measurements used to determine the wind
and the characteristics of the sensors used, and 
the \href{http://www.eol.ucar.edu/aircraft-instrumentation}{EOL instrument web pages} 
(cf.~"State Parameters", "Wind")
provide additional information on these measurements.
% \begin{table}
% \begin{tabular}{>{\centering}p{2.3cm}>{\centering}p{2.5cm}>{\centering}p{2.5cm}>{\centering}p{2.2cm}>{\centering}p{2.5cm}}
% \toprule 
% \textbf{Measurement \small{(VARIABLE)}} & \textbf{Instrument} & \textbf{Range, Characteristics} & \textbf{Standard Uncertainty} & \textbf{Comments}\tabularnewline
% \midrule
% \midrule 
% pitch, roll\\
% (PITCH, ROLL) & Laseref IV Model \small{HG2001 GD03} & ring gyros, strap-down system & 0.05$^{\circ}$ & mixed bias and random error\tabularnewline
% \midrule 
% heading\\
% (THDG) & `` &  & 0.2$^{\circ}$ & \tabularnewline
% \midrule 
% ambient pressure (PSF) & absolute transducer Paroscientific 1000-15A-28 & 0--15 PSI $\simeq$0--1035\,hPa & 0.10\,hPa &
% (from specs, assumed to give std uncertainty)\tabularnewline
% \midrule 
% dynamic pressure (QCF, QCR) & differential transducer PPT0005DXX2VB-S021 & 0--5 PSI $\simeq$0--345 hPa & 0.34 hPa\\
% 0.68 hPa max. & ``\tabularnewline
% \midrule 
% pressure differences, radome (ADIFR, BDIFR) & differential transducer PPT0001DXX2VB-S021 & range $\pm1$ PSI$\simeq$$\pm$68.95 hPa & 0.07 hPa\\
% 0.14 hPa max. & the first is ``typ.'', average over the range\tabularnewline
% \midrule 
% horizontal GV velocity components (VNS, VEW) & Honeywell Laseref IV HG2001 GD03 & strap-down accelerometers & 2.1\,m\,s$^{-1}$  & 0.1\,m\,s$^{-1}$ with slow updating to GPS\tabularnewline
% \midrule 
% '' '' '' (GGVNS, GGVEW) & Novatel OEM-3 differential GPS  & (L1/L2) correction via OmniSTAR XP & 0.03\,m\,s$^{-1}$ & <0.1\,m\,s$^{-1}$ when OmniSTAR is not available \tabularnewline
% \midrule 
% vertical GV speed (VSPD) &  Laseref IV (see above) & & 0.76 m\,s$^{-1}$ & with baro-loop updating\tabularnewline
% \midrule 
%  '' '' '' (GGVSPD) & Novatel GPS (see above)&  & 0.03\,m\,s$^{-1}$ with OmniSTAR & 0.1\,m\,s$^{-1}$ without OmniSTAR\tabularnewline
% \midrule 
% air temperature &HARCO 100009-1  &anti-iced, -80 to +40$^{\circ}$C  &0.3$^{\circ}$C  & needed for TAS calc.\tabularnewline
% \bottomrule
% \end{tabular}
% 
% \protect\caption{Characteristics of measurements from the radome-based system that
% are used for the standard calculation of the wind. See further
% discussion of each measurement in Sect.~\ref{sub:The-radome-based-system}.\label{tab:Radome-system-measurements}}
% \end{table}

\begin{table}
\begin{tabular}{>{\centering}p{2.3cm}>{\centering}p{2.7cm}>{\centering}p{2.5cm}>{\centering}p{2.2cm}>{\centering}p{2.5cm}}
\toprule 
\textbf{Measurement  \small{(VARIABLE)}} &
\textbf{Instrument} &
\textbf{Range, Characteristics} &
\textbf{Standard Uncertainty} &
\textbf{Comments}\tabularnewline
\midrule
\midrule 
pitch, roll\\
(PITCH, ROLL) &
Honeywell Laseref IV HG2001 GD03 &
ring gyros, strap-down system &
0.05$^{\circ}$ &
mixed bias and random error\tabularnewline
\midrule 
heading\\
(THDG) &
`` &
`` &
0.2$^{\circ}$ &
``\tabularnewline
\midrule 
ambient pressure (PSF) &
Paroscientific 1000-15A-28 (absolute) &
o--15 psi $\simeq$0--1035\,hPa &
0.10\,hPa\\
 &
specs. assumed st. uncertainty\tabularnewline
\midrule 
dynamic pressure (QCF, QCR) &
Honeywell PPT0005-DXX2VB-5021 &
0--5 PSI $\simeq$0--345 hPa &
0.34 hPa\\
0.68 hPa max. &
``\tabularnewline
\midrule 
pressure differences (\{A,B\}DIFR) &
Honeywell PPT0001-DXX2VB-5021 &
$\pm1$\textasciitilde{}psi $\simeq$$\pm$68.95 hPa &
0.07 hPa\\
0.14 hPa max. &
the first is ``typ.'', average over the range\tabularnewline
\midrule 
horizontal GV velocity components

(VNS, VEW) &
Laseref IV (see above) &
strap-down accelerometers &
2.1\,m\,s$^{-1}$  &
0.1~m~s$^{-1}$ with slow updating to GPS\tabularnewline
\midrule 
``\\
(GGVNSB, GGVEWB) &
Novatel OEM-3 differential GPS  &
(L1/L2) correction via OmniSTAR XP &
0.03\,m\,s$^{-1}$ &
 <0.1\,m\,s$^{-1}$ when OmniSTAR is not available\tabularnewline
\midrule 
vertical GV speed (VSPD) &
Laserref IV (see above) &
strap-down accelerometers &
0.76 m\,s$^{-1}$ &
with baro-loop updating\tabularnewline
\midrule 
 " (GGVSPDB) &
Novatel GPS (see above) &
 &
0.03\,m\,s$^{-1}$ with OmniSTAR &
0.1\,m\,s$^{-1}$ without OmniSTAR\tabularnewline
\midrule 
temperature (ATX) &
HARCO 100009-1 &
$-80$ to +40$^{\circ}$C, anti-iced &
0.3$^{\circ}$ &
needed for true airspeed\tabularnewline
\midrule 
dewpoint

(DPX) &
Buck Research 1011C &
$-70$ to +30$^{\circ}$C &
<5$^{\circ}$ &
for level flight\tabularnewline
\bottomrule
\end{tabular}

\protect\caption{Characteristics of measurements from the radome-based system that
are used for the standard calculation of the wind. See the
discussion of each measurement in the text of Sect.~\ref{sub:The-radome-based-system}.\label{tab:Radome-system-measurements}}
\end{table}


\subsubsection{Attitude angles\label{sub:Attitude-angles}}

Attitude angles (roll, pitch, and heading) are provided by the IRU
discussed in the preceding sub-section, with specifications as listed
in Table~\ref{tab:Radome-system-measurements}. There are duplicate
inertial systems, so a measure of uncertainty is how well they agree.
For DEEPWAVE flight 16, the mean difference
in pitch was 0.2$^{\circ}$ and the standard deviation in the difference
between measurements was about 0.015$^{\circ}$ (for measurements with absolute value
of the roll smaller than 3$^{\circ}$), and if the measurements are filtered to
remove variations with period exceeding about 1000~s the slowly varying component
of the difference has standard deviation of about 0.012$^{\circ}$ while
the fast-varying component has standard deviation of 0.008$^{\circ}$. This was
characteristic of most flights, although there were two (8 and 20) that had
slightly larger standard deviations. The project averages were 0.02$^{\circ}$ for the
slow component and about 0.007$^{\circ}$ for the fast component. This is an indication that
the system performs better than indicated by the specifications, and indeed 
additional evidence for this is provided in Section 3. The difference in 
pitch and enhanced standard deviation in turns likely arises from small mis-alignment of the units relative to
the longitudinal axis of the aircraft. As discussed in Section 3, the error in
pitch tends to precess with a period of about 84.4~min, so the slowly varying
component tends to be dominated by this precession which, for periods short compared to 84.4~min, introduces a bias while the faster varying component has the character of a random error.


\subsubsection{Ambient pressure\label{sub:Ambient-pressure}}

Ambient or ``static'' pressure is measured on the GV at pressure
ports called static buttons that are located at positions on the fuselage
where in normal flight the pressure is approximately the pressure
present outside the airflow disturbance produced by the aircraft.
Because there are residual effects of airflow that change with angle
of attack and Mach number, corrections are applied to these measurements
to obtain better representation of the true ambient pressure. These
corrections are described in \href{https://drive.google.com/file/d/0B1kIUH45ca5ATFV5d3QyQ0JpSjA/view?usp=sharing}{this document on processing algorithms},
Sect.~4.3, and in \cite{CooperEtAl2014}. The latter reference has
additional information on the locations of the sensors and the system
configuration. The transducer characteristics are listed in Table~\ref{tab:Radome-system-measurements},
and the transducer is temperature-compensated to maintain these characteristics
in flight when the cabin-mounted transducer can encounter fluctuating
temperature. It is a digital transducer with resolution of 0.001\,hPa,
equivalent to about 20-bit resolution, when sampling at 1\,Hz. The
characteristic response time of the sensor is 0.02\,s and measurements
are normally sampled at 50\,Hz and filtered to 25\,Hz. However,
lines of length \marginpar{XXX}XXX and diameter XXX connect the transducer
to the static buttons, and these lines affect the response time significantly,
as discussed in Sect.~\marginpar{XXX}. More information is available
on the EOL instrument pages; see \href{https://www.eol.ucar.edu/instruments/ambient-static-pressure}{this link}.


\subsubsection{Dynamic pressure\label{sub:Dynamic-pressure}}

The dynamic pressure is the pressure difference above ambient that
develops if air is compressed and adiabatically brought to rest relative
to the moving aircraft. The total pressure, the sum of ambient and
dynamic pressure, is sensed using a pitot tube, a tube pointed in
the direction of the relative airflow and specially designed to be
relatively insensitive to small-angle changes in the direction of
the relative airflow. Figure~\ref{fig:pitot-tube-photo} shows the
location of the research-system pitot tube on the GV as well as one
of the avionics-system pitot tubes. The excess pressure above the
ambient sensed by a pitot tube or, approximately, by the center port
on the radome is $0.5\rho_a v^2$ where $\rho_a$ is the density of air
and $v$ the airspeed, so this excess pressure can be used to determine the airspeed
of the aircraft. On NSF/NCAR aircraft, the measurement of dynamic
pressure is made using differential sensors connected between a static
source and a total-pressure source from either a pitot tube (QCF)
or the front port on the radome (QCR). The sensor used, with characteristics
listed in Table~\ref{tab:Radome-system-measurements}, has these
additional characteristics: Specified resolution is 0.0011\% of full
scale or 0.0076 hPa, which is better than 16-bit resolution; the maximum
sampling rate is 120 Hz, response time 50ms and response delay 21 ms
or about one sample period at 50\,Hz sample rate. The response time
is affected further by the pressure lines between the pressure ports
and the transducers; see Sect.~XXX\marginpar{XXX} for discussion of
this effect. The transducer provides 50-Hz output that is then filtered
digitally to 25\,Hz or 1\,Hz in processing.%
\footnote{The specifications indicate that the appropriate time lag to apply
in processing would be 21 ms but there is additional delay introduced
by the sample tubing. Most processing including preliminary processing
for DEEPWAVE has not introduced a delay for QCF or QCR.%
}

Because any errors affecting the measurement of static pressure also
affect the difference between dynamic and static pressure, the same
corrections that are applied to static pressure (for errors in the
pressure delivered by the static ports) are also applied to the dynamic
pressure. See \href{https://www.eol.ucar.edu/instruments/dynamic-pressure}{the EOL instrument pages}
for more information. \cite{CooperEtAl2014} argue that the measurements
of static and dynamic pressure, corrected for flow distortion or generation
of a ``static defect'' at the static-pressure ports, each have standard
uncertainty of 0.3 hPa and precision (for straight and level flight)
of 0.1 hPa.

\begin{figure}
\noindent \begin{centering}
\includegraphics[width=8.5cm]{PitotTube-r}
\par\end{centering}

\protect\protect\caption{A pitot tube used for the measurement of dynamic pressure.\label{fig:pitot-tube-photo}}
\end{figure}



\subsubsection{Airflow angles\label{sub:Airflow-angles}}

The radome gust-sensing system consists of five pressure ports installed
in a standard GV radome, as shown in Fig.~\ref{fig:radome-photo}.
\begin{figure}
\noindent \begin{centering}
\includegraphics[width=8cm]{RadomePhotoGV} 
\par\end{centering}

\protect\caption{Photograph of the GV radome showing three of the five pressure ports
on the radome used for measurement of components of the relative wind.\label{fig:radome-photo}}
\end{figure}
 The pressure ports are connected to differential pressure sensors,
one between the top and bottom ports (variable ADIFR), one between
the left and right ports (variable BDIFR), and one between the center
port and the static source (variable QCR). The latter provides an
alternative measurement of dynamic pressure but is not normally used.
The measurements ADIFR and BDIFR are obtained from differential 
pressure transducers, with characteristics as shown in Table~\ref{tab:Radome-system-measurements}.
The transducers have specified response times of 0.05\,s and resolution
0.0015\,hPa, with quoted stability of about 0.03\,hPa per year.
These measurements are used with procedures discussed in Section \ref{sub:radome-sensitivity}
to find the angle of attack and sideslip angle of the relative wind.
Additional information is available at \href{https://www.eol.ucar.edu/instruments/radome-gust-probe-3-d-wind-measurements}{this URL}.


\subsubsection{Components of aircraft velocity relative to the Earth}

There are two sources of information regarding the ground-speed vector,
an inertial reference unit (IRU) and a Global Positioning System (GPS). 
\begin{enumerate}
\item \uline{The IRU:} The inertial system on the GV is a Honeywell Laseref
IV Model HG2001 GD03, with characteristics as listed in Table~\ref{tab:Radome-system-measurements}.
There are three units on the aircraft, two of which are recorded via
the ARINC data bus to standard data files. These are strap-down ring
laser gyro micro inertial systems. The measurements of ground-speed
components are affected by errors that arise from initial alignment
errors or orientation errors resulting from gyro responses to acceleration
and so often exhibit a characteristic Schuler oscillation with magnitude
that can be as much as 1--3 m\,s$^{-1}$. This is the primary source of error
in the measurements of wind,
so for aircraft velocity components it is important
to remove these errors by comparison to lower-uncertainty measurements
provided by the GPS that are not subject to the Schuler oscillation.
In addition, there are signal delays that are accounted for in post
processing to align measurements with other recorded data, and there
are some inherent filters in the IRU computer that affect the signals
transmitted to the GV data system. The orientation of this unit was defined and checked
by initial survey to coincide with the aircraft reference axes.
\item The \uline{GPS:} The primary GPS unit is a Novatel OEM-3 differential
GPS unit (L1/L2) with OmniSTAR XP satellite update for (ionospheric)
corrections. As used on the GV, it reports ground-speed components
at a rate of 5 Hz, although faster rates are possible. The claimed
standard uncertainty for position is 0.15~m for vertical position;
the standard uncertainty in velocity is 0.03 m\,s$^{-1}$ when OmniSTAR corrections
are available and <0.1~m\,s$^{-1}$ otherwise.\marginpar{XXX}
\end{enumerate}

\subsubsection{Temperature\label{sub:Temperature}}

A measurement of temperature is needed to calculate the wind because
the conversion from dynamic pressure to true airspeed involves the
temperature, as documented in \href{https://drive.google.com/file/d/0B1kIUH45ca5ATFV5d3QyQ0JpSjA/view?usp=sharing}{this document on processing algorithms}.
The measurements of temperature were checked against expectations for
height-vs-pressure changes from the hydrostatic equation by \cite{CooperEtAl2014}, with the result that the measurements were validated to an uncertainty of about 0.3$^{\circ}$C. Documentation of the temperature uncertainty will be
presented in more detail in a separate document. 

\subsubsection{Humidity (dew point)\label{sub:dewpoint}}

The calculation of true airspeed from measured dynamic and static pressure involves
the specific heats and gas constant for air, and this can depart from dry-air values
when water vapor is present in significant amounts. This correction is usually
insignificant for dew-point temperatures below about -20$^{\circ}$C but can be important
at higher dew-point temperature. The equations used are those in \href{https://drive.google.com/file/d/0B1kIUH45ca5ATFV5d3QyQ0JpSjA/view?usp=sharing}{the document on processing algorithms}. The correction 
to true airspeed is approximately a factor of (1+0.3$q_h$ where $q_h$ is the
dimensionless specific humidity, typically about 0.01 at 10$^{\circ}$C dewpoint and
700~hPa pressure. In this case the correction to airspeed, typically 150~m/s at
this altitude, is about 0.45~m/s, so the correction is not negligible but is relatively
insensitive to uncertainty in the measured humidity.The dewpoint measurements become more uncertain than listed here at the low end of this range, but the humidity correction is insignificant there. They are likely 
better than listed here for the upper range, in level flight, but lags and overshooting
introduce errors when conditions are changing rapidly as in climbs or descents.

\subsubsection{Examples of measurements}
\begin{knitrout}
\definecolor{shadecolor}{rgb}{0.969, 0.969, 0.969}\color{fgcolor}\begin{figure}
\includegraphics[width=\maxwidth]{figure/radome-plot-angles-1} \caption[Attitude angles pitch, roll, and heading as measured by three independent systems inertial reference systems]{Attitude angles pitch, roll, and heading as measured by three independent systems inertial reference systems. The systems are: (1) the standard Honeywell IRU (PITCH, ROLL, THDG, blue lines); (2) a duplicate backup Honeywell IRU (PITCH\_IRS2, ROLL\_IRS2, THDG\_IRS2, green lines), and the C-MIGITS-III IRU mounted in the gust pod (CPITCH\_GP, CROLL\_GP, CTHDG\_GP, red lines). All units are degrees. Data from DEEPWAVE flight 16 (4 July 2014), 9:00:00 to 10:00:00\label{fig:radome-plot-angles}}
\end{figure}


\end{knitrout}

Typical measurements of the attitude angles are shown in Fig.~\ref{fig:radome-plot-angles}. The large difference in pitch is a result of the gust pod being installed in a canister below the wing where it points downward by several degrees relative to the aircraft longitudinal axis. (The pods were designed this way to provide better approaching airflow for cloud-imaging probes and other sampling from the airstream.) There is also a significant difference in heading and in roll for similar reasons.

\begin{knitrout}
\definecolor{shadecolor}{rgb}{0.969, 0.969, 0.969}\color{fgcolor}\begin{figure}
\includegraphics[width=\maxwidth]{figure/radome-plot-pressures-1} \caption[The measurements of dynamic pressure (QCF and, after correction QCFC), ambient pressure (PSF and corrected PSFC) and the resulting true airspeed TASX]{The measurements of dynamic pressure (QCF and, after correction QCFC), ambient pressure (PSF and corrected PSFC) and the resulting true airspeed TASX. Data from DEEPWAVE flight 16.\label{fig:radome-plot-pressures}}
\end{figure}


\end{knitrout}
The measurements of pressures and the true airspeed calculated from these measurements are shown in Fig.~\ref{fig:radome-plot-pressures} for the same period as in the preceding figure. Corrections have been applied to the pressure measurements according to the calibration determined from LAMS measurements, as described by \cite{CooperEtAl2014}; these corrections vary with flight conditions but normally are smaller than a few hPa so are not evident in these plots. They are nevertheless crucial to reducing the uncertainty in the true airspeed to about 0.3\,m\,s$^{-1}$,
as shown in that reference.

\begin{knitrout}
\definecolor{shadecolor}{rgb}{0.969, 0.969, 0.969}\color{fgcolor}\begin{figure}
\includegraphics[width=\maxwidth]{figure/radome-plot-airflow-angles-1} \caption[The pressure differences measured on the radome (ADIFR and BDIFR, respectively between the vertically separated ports and the horizontally separated ports) and the resulting airflow angles AKRD (angle of attack) and SSRD (sideslip angle)]{The pressure differences measured on the radome (ADIFR and BDIFR, respectively between the vertically separated ports and the horizontally separated ports) and the resulting airflow angles AKRD (angle of attack) and SSRD (sideslip angle). Data from DEEPWAVE flight 16.\label{fig:radome-plot-airflow-angles}}
\end{figure}


\end{knitrout}

Figure~\ref{fig:radome-plot-airflow-angles} shows the measurements of differential pressure at the radome and the resulting angle-of-attack and sideslip angle calculated from those pressure differences. The calculation is described in Section 4 of this document. Fluctuations in sideslip angle are seldom more than a fraction of a degree, while there can be several-degree fluctuations in the angle-of-attack. The gradual decrease in angle-of-attack is a result of the change in fuel load on the aircraft, which requires a smaller angle-of-attack to keep the aircraft level as the weight becomes smaller.

\begin{knitrout}
\definecolor{shadecolor}{rgb}{0.969, 0.969, 0.969}\color{fgcolor}\begin{figure}
\includegraphics[width=\maxwidth]{figure/radome-plot-groundspeed-1} \caption[Top two panels]{Top two panels: Ground-speed components as measured by the IRU and GPS, and (red lines) the difference between the two measurements multiplied by a factor of 100. Bottom panel: Aircraft vertical speed as measured by the IRU (with built-in damping to the pressure altitude) and by the GPS unit. Data from DEEPWAVE flight 16.\label{fig:radome-plot-groundspeed}}
\end{figure}


\end{knitrout}

The last set of components entering the measurement of wind consists of the measurements of the motion of the aircraft with respect to the Earth. These measurements must be combined with the measurement of relative wind to transform the measurements to an Earth-referenced measurement. Figure~\ref{fig:radome-plot-groundspeed} shows the east and north components of the ground speed as measured by the IRU and GPS. They are close enough to lie almost on top of each other in this plot, but the red lines show the difference magnified by a factor of 100. They clearly show the Schuler oscillation that results from an IRU error in pitch, having magnitude of about 1--2~m\,s$^{-1}$. This error is discussed in the next section, and Section~\ref{sub:comp-filter} discusses how the IRU measurements (having good short-term response but long-term drift) and the GPS measurements (having long-term accuracy but inferior short-term response) are combined in the measurement of wind. In addition to the Schuler oscillation, additional perturbations associated with turns result from the mixing of pitch, roll, and heading errors when the aircraft is banked.

\begin{knitrout}
\definecolor{shadecolor}{rgb}{0.969, 0.969, 0.969}\color{fgcolor}\begin{figure}
\includegraphics[width=\maxwidth]{figure/radome-plot-wind-1} \caption[Wind measurements for DEEPWAVE research flight 16]{Wind measurements for DEEPWAVE research flight 16.\label{fig:radome-plot-wind}}
\end{figure}


\end{knitrout}

Finally, Fig.~\ref{fig:radome-plot-wind} shows the resulting wind measurements for this flight.
These measurements will be discussed extensively in the remainder of this report, and the uncertainty associated with them will be estimated in the closing section.

\subsection{The gust-pod system\label{sub:The-gust-pod-system}}


\subsubsection{Overview}

The all-weather wind pod was developed by Allen Schanot and is available
for mounting under the wing of the GV, where it was installed during
the 2014 project DEEPWAVE. It was still regarded as experimental for
this project, and this is the first full documentation of its characteristics.
It will be called the gust-pod system here; another name used has
been the all-weather wind pod, because the primary reason for its
development was to provide a backup wind measurement for cases when
the radome system was not available, including times when it was blocked
by ice or frozen water in the pressure lines. The gust pod fits into
a standard ``PMS-style'' canister and uses a Rosemount 858 probe,
but the location under the wing is one where there is substantial
flow distortion in comparison to the free stream so an unconventional
calibration is needed to use the measurements. The 858 probe is anti-iced
by heaters and should be unaffected by icing or ice accumulation.
Five ports are oriented forward, upward and downward 45$^{\circ}$,
and left and right 45$^{\circ}$ on the leading edge of the sensor,
which has the shape of a hemisphere. There are also ports in a ring
around the cylinder behind the hemisphere that provide a static source.
The measurements are the pressure difference between the top and bottom
ports (ADIF\_GP), the pressure difference between the right and left
ports (BDIF\_GP), and the pressure difference between the forward
port and the static ports (QC\_GP). In addition, the pressure provided
by the ring of static ports is recorded as PS\_GP. The system incorporates
a Systron-Donner C-MIGITSIII IRU, which is mounted in the pod to be
able to measure vibrations and wing-flex motions that will affect
the measurements of wind. This unit provides measurements of attitude angles,
ground-speed components, and accelerations and uses a GPS signal in a Kalman-filter
feedback loop to reduce errors in the measurements. The relevant specifications
are listed in Table~\ref{tab:Gust-pod-measurements}.

\noindent \begin{center}
\includegraphics[width=12.1cm]{GustPod1}\includegraphics[height=5.1cm]{GustPod2x-r}\\
 \textsl{Photograph of the Gust Pod (bottom left) and the ports on
the Rosemount 858 probe (right).} 
\par\end{center}

\begin{table}
\begin{minipage}[t]{0.93\columnwidth}%
\begin{tabular}{>{\centering}p{2.3cm}>{\centering}p{2.7cm}>{\centering}p{2.5cm}>{\centering}p{2.2cm}>{\centering}p{2.5cm}}
\toprule 
\textbf{Measurement \small{(VARIABLE)}} & \textbf{Instrument} & \textbf{Range, Characteristics} & \textbf{Standard Uncertainty} & \textbf{Comments}\tabularnewline
\midrule
\midrule 
velocity components
(CVNS\_GP, CVEW\_GP, CVSPD\_GP) & C-MIGITSIII & with GPS & 0.1\,m\,s$^{-1}$  & horizontal and vertical\tabularnewline
\midrule 
pitch, roll\\
(CPITCH\_GP, CROLL\_GP) & C-MIGITSIII & with GPS & 1\,mrad$\simeq$0.06$^{\circ}$ & with Kalman filter \tabularnewline
\midrule 
heading\\
CTHDG\_GP & C-MIGITSIII & with GPS & 1.5\,mrad\\
$\simeq$0.09$^{\circ}$ & valid when in motion%
\footnote{Without occasional turns the heading error grows linearly at about
1--10$^{\circ}$/hour%
}\tabularnewline
\midrule 
pressure differences,858 ports\\
(ADIFR\_GP, BDIFR\_GP) & Honeywell PPT0001-DXX2VB-S021 & $\pm1$ psi$\simeq$$\pm$68.95 hPa & 0.07 hPa\\
0.14 hPa max. & the first is ``typ.''; same transducers as for radome\tabularnewline
\midrule 
dynamic pressure (QC\_GP) & Honeywell PPT0005-DXX2VB-S021%
 & 0--5 psi $\simeq$0--345 hPa & 0.34 hPa\\
0.68 hPa max. & ``\tabularnewline
\midrule 
ambient pressure (PS\_GP) & Paroscientific 6000-15A-28 & 0--15 PSI $\simeq$0--1035\,hPa & 0.10\,hPa & digital transducer\tabularnewline
\bottomrule
\end{tabular}%
\end{minipage}

\protect\caption{Characteristics of measurements from the gust-pod that are used for
the calculation of the wind. There is further discussion of each measurement
in the text of Sect.~\ref{sub:The-gust-pod-system}.\label{tab:Gust-pod-measurements}}
\end{table}



\subsubsection{Attitude angles}

The C-MIGITSIII INS/GPS unit provides measurements of the attitude angles,
recorded as variables CROLL\_GP, CPITCH\_GP, and CTHDG\_GP. The estimated
standard uncertainty in measurement of pitch, 1~mrad (cf.~Table~\ref{tab:Gust-pod-measurements}),
is supported by comparison to the cabin-mounted inertial systems;
the standard deviation in the difference between the two systems was
about 0.1$^{\circ}\simeq1.7$~mrad for extensive multi-flight comparisons,
while the expected difference for two systems each having standard
uncertainties of 1 mrad would be 1.4\,mrad. Some contribution would
be expected from vibrations and wing flex that affect the gust pod,
so these comparisons are good support for the approximate validity
of the specifications. However, the feedback from the Kalman filter using
GPS measurements can be ineffective in the absence of turns or maneuvers,
so some of the specified uncertainties, esp.~for heading, can be
exceeded significantly in the absence of maneuvers, as noted in 
Table~\ref{tab:Gust-pod-measurements}. 


\subsubsection{Ambient or static pressure}

Ambient pressure (variable PS\_GP) is measured by a digital transducer with low uncertainty,
as listed in Table~Table~\ref{tab:Gust-pod-measurements}. However,
the Rosemount 858 probe is located under the wing in a location where
there is significant airflow distortion, so these measurements often
differ from the measurements from the static buttons on the fuselage
by 10--20\,hPa and significant corrections are needed if these measurements
are to be used for pressure measurements. With the gust-pod, the use
is as a reference for the differential measurement of dynamic pressure
because the dynamic-pressure transducer is connected between the total-pressure
port on the front of the 858 probe and the static ports. No calibration has been
determined that would make this measurement useful as a measure of true ambient pressure,
although that could be done by fitting to match the standard static pressure. In the
absence of such a correction, PS\_GP should not be considered an alternate measurement of ambient pressure. The use of this measurement to determine an alternate measure of true airspeed will be discussed in Section~\ref{sub:GP-TAS}.


\subsubsection{Dynamic pressure}

The dynamic pressure QC\_GP is measured by a differential pressure
transducer, with characteristics shown in Table~\ref{tab:Gust-pod-measurements}.
The measurement is the pressure difference between the forward-pointing
port on the Rosemount 858 probe and the static ports on the side
of that probe. Because the system is located under the wing in a region
of disturbed airflow, the dynamic pressure requires unconventional
correction to obtain the airspeed, as discussed in Sect.~\ref{sec:Calibrations}. This
measurement is also used in the calculation of flow angles from the gust-pod pressure
ports, as also discussed in that section.


\subsubsection{Airflow angles}

The difference between pressures at the top and bottom ports of the
Rosemount 858 probe (ADIFR\_GP) and the corresponding difference between
right and left ports (BDIFR\_GP) are also measured using differential
transducers listed in Table~\ref{tab:Gust-pod-measurements}. These
are analog transducers that produce voltages representing the pressure
differences, and those voltages are digitized and recorded by the
aircraft data system.


\subsubsection{Components of aircraft velocity relative to the Earth}

A Systron-Donner C-MIGITSIII INS/GPS, mounted with the gust pod, provides
digital representations of the ground-speed components CVEW\_GP and
CVNS\_GP and the vertical speed CVSPD\_GP. (cf.~Table~\ref{tab:Gust-pod-measurements}).
The unit incorporates a GPS receiver and uses GPS information as input
to a Kalman filter for adjustment of these measurements and others discussed in this
section. 


\subsubsection{Temperature}

Air temperature is measured in the same way as discussed for the radome
system in Sect.~\ref{sub:Temperature}, and the same variable (ATX) is used.


\subsubsection{Examples of measurements}

A preliminary calibration was in use for in-field processing during
DEEPWAVE, but it did not perform very well. A new calibration
is developed in Section \ref{sec:Calibrations}. On the basis of that new
calibration, it appears that the gust pod provides a useful alternative
to wind measurements based on the radome. Plots and average values are
presented there to support the validity of this measurement.

% For coding of the processing and construction of the plots used here,
% see the file \textasciitilde{}cooperw/ RStudio/DEEPWAVE/GustPodProcessor.Rnw
% on EOL file systems. This program, if run from within the html interface
% to RStudio on tikal using the \textquotedbl{}Compile PDF\textquotedbl{}
% function, will produce this section of the current document while
% processing a netCDF file to add new gust-probe variables to it. The
% processor also attempts to flag bad CTHDG\_GP measurements, which
% apparently occur as the heading moves through 180 deg. (This would
% be better replaced by interpolation at these points.)

%For similar
%results from DEEPWAVE rf05, saved for reference, see GPcal.pdf, a
%note saved on 22 June 2014.\marginpar{XXX}

\begin{knitrout}
\definecolor{shadecolor}{rgb}{0.969, 0.969, 0.969}\color{fgcolor}\begin{figure}
\includegraphics[width=\maxwidth]{figure/vw-gp-vs-wic-1} \caption[Comparison of vertical wind calculated from the gust pod (WI\_GP) to the conventional measurement of vertical wind (WIC)]{Comparison of vertical wind calculated from the gust pod (WI\_GP) to the conventional measurement of vertical wind (WIC). The dashed orange line is a 1:1 reference line, and each blue dot represents a 1-s pair of corresponding measurements.\label{fig:vw-gp-vs-wic}}
\end{figure}


\end{knitrout}
\begin{knitrout}
\definecolor{shadecolor}{rgb}{0.969, 0.969, 0.969}\color{fgcolor}\begin{figure}
\includegraphics[width=\maxwidth]{figure/hw-gp-vs-std-1} \caption[Comparison of horizontal wind direction (top panel) and horizontal wind speed (bottom panel) as determined from the gust pod and from the conventional radome-based system, for measurements where the true airspeed is greater than 130 m/s (to exclude periods of possible flap deployment)]{Comparison of horizontal wind direction (top panel) and horizontal wind speed (bottom panel) as determined from the gust pod and from the conventional radome-based system, for measurements where the true airspeed is greater than 130 m/s (to exclude periods of possible flap deployment). Only measurements considered valid are plotted for the gust pod; the restrictions where the gust-pod measurement is flagged as missing and therefore is not plotted here are: altitude (GGALT) > 5000 m, absolute value of roll (ROLL) < 5$^{\circ}$. This causes exclusion of some measurements near the start and end of the flight and during turns.\label{fig:hw-gp-vs-std}}
\end{figure}


\end{knitrout}

The following are some plots that show the results of this processing,
in this case from DEEPWAVE flight RF16 on 4 July 2014. Figure~\ref{fig:vw-gp-vs-wic} shows 
a comparison of the vertical wind calculated from the gust pod (WI\_GP) vs the conventional vertical wind (WIC). The standard deviation between the two measurements is 0.27\,m/s. % mean diff 0.13
This is a good illustration of performance because this is a flight with large-amplitude waves and some
of the largest vertical-wind measurements in the DEEPWAVE project, so the consistency of these
measurements even to extremes in this plot indicates that the measurements from the gust pod
are useful even for these large-amplitude measurements. Figure~\ref{fig:hw-gp-vs-std} shows
the corresponding horizontal-wind measurements, and also shows good agreement between the
gust pod and the conventional wind-sensing system.\footnote{Some spikes would occur
in turns if the exclusions listed in the figure caption were not applied to the measurements from the gust pod. These are the result of a problem with the C-MIGITSIII
measurement of heading, which exhibits noisy fluctuations as it moves through 180$^{\circ}$. 
These fluctuations introduce large perturbations in the horizontal wind measurements, 
affecting esp.~the east component of the wind.}

\begin{knitrout}
\definecolor{shadecolor}{rgb}{0.969, 0.969, 0.969}\color{fgcolor}\begin{figure}
\includegraphics[width=\maxwidth]{figure/vw-gp-small-segment-1} \caption[A 30-min segment from flight 16 of the DEEPWAVE project, showing the good agreement of the vertical wind measurements from the gust pod (WI\_GP) and from the conventional wind-sensing system on the GV (WIC)]{A 30-min segment from flight 16 of the DEEPWAVE project, showing the good agreement of the vertical wind measurements from the gust pod (WI\_GP) and from the conventional wind-sensing system on the GV (WIC).\label{fig:vw-gp-small-segment}}
\end{figure}


\end{knitrout}

A small segment of flight from a period with large-amplitude waves is shown in Fig.~\ref{fig:vw-gp-small-segment}. The two measurements match quite well in regard to the structure of these waves and the amplitudes of the fluctuations. The measurements of horizontal wind speed are in similar agreement, but the wind direction for this period shows an offset for the gust pod measurement relative to the conventional measurement, varying from about 5$^{\circ}$ near the
start of this period to about 1$^{\circ}$ near the end. This is a result of an apparent
error in heading from the C-MIGITS IRU, a common feature to see near the start of flights but
one that usually was made smaller during flight by the action of the GPS updating via Kalman
filter in that unit. 

The result of this error in wind direction and the problems with measured heading for flight
exactly southbound complicate the use of the measurements of horizontal wind from the gust
pod and make them of lesser quality that the standard measurements. Fortunately, in this and
most projects, the horizontal-wind measurements are available from the radome-based system
on all flights and are usually not compromised even when there is loss of the measurement
of angle-of-attack from plugging of the lines in the radome, because the side-mounted ports
for the measurement of sideslip seldom are plugged.


% \begin{figure}
% {\centering \includegraphics[width=300pt]{figure/Plotted-Results1}
% 
% }
% 
% \protect\protect\caption[WIG (vertical wind based on the gust pod) plotted against WIC (vertical
% wind from the conventional radome-based gust system)]{WIG (vertical wind based on the gust pod) plotted against WIC (vertical
% wind from the conventional radome-based gust system)\label{fig:Plotted-Results1}}
% \end{figure}
% 
% 
% \begin{figure}
% {\centering \includegraphics[width=300pt]{figure/Plotted-Results2}
% 
% }
% 
% \protect\protect\caption[Angle of attack determined from gust-pod measurements, plotted vs]{Angle of attack determined from gust-pod measurements, plotted vs.
% corresponding measurements AKRD from the standard wind sensing system\label{fig:Plotted-Results2}}
% \end{figure}
% 
% 
% \begin{figure}
% {\centering \includegraphics[width=300pt]{figure/Plotted-Results3}
% 
% }
% 
% \protect\protect\caption[Comparison of horizontal wind measurements from the gust pod (red
% lines) and from the standard wind measurements WDC and WSC (thicker
% blue lines)]{Comparison of horizontal wind measurements from the gust pod (red
% lines) and from the standard wind measurements WDC and WSC (thicker
% blue lines).\label{fig:Plotted-Results3}}
% \end{figure}
% 
% 
% \begin{figure}
% {\centering \includegraphics[width=300pt]{figure/Plotted-Results4}
% 
% }
% 
% \protect\protect\caption[Comparison of vertical wind measurements from the gust pod (red line)
% and from the standard wind measurement (WIC, blue line)]{Comparison of vertical wind measurements from the gust pod (red line)
% and from the standard wind measurement (WIC, blue line).\label{fig:Plotted-Results4}}
% \end{figure}
% 
% 
% \begin{figure}
% \noindent \begin{centering}
% \includegraphics[width=10cm]{GPrf11Fig1} 
% \par\end{centering}
% 
% \protect\protect\protect\caption{\label{fig:3panelHWVW}A comparison between the conventional wind
% components (blue lines) and the wind from the gust pod (red lines).
% Data from flight RF05.}
% \end{figure}
% 
% 
% Some plots showing the nature of the new variables (WDG, WSG, WIG)
% are included here. The first one (Fig.~\ref{fig:3panelHWVW}) shows
% a comparison of all three wind variables for a segment from flight
% RF11. The spike in WSG at about 9:43 has a known origin and will be
% removed. There is also a short period near 9:23 when the vertical
% wind measurements appear offset by a few tenths m\,s$^{-1}$. Otherwise, the
% gust pod provides a good representation of the standard variables.
% \begin{figure}
% \noindent \begin{centering}
% \includegraphics[width=10cm]{GPrf11Fig2} 
% \par\end{centering}
% 
% \protect\protect\protect\caption{\label{fig:ExpandedViewF1}An expanded view of a section of Fig.~\ref{fig:3panelHWVW}.}
% \end{figure}
% 
% 
% Figure \ref{fig:ExpandedViewF1} shows an expanded view and emphasizes
% the small difference between wind speeds from the two systems; this
% is likely caused by IRU differences and can later be improved by reference
% to the GPS signals with OMNISTAR. The similarities and differences
% in WIG vs WIC are also more evident in this figure.

\begin{figure}
\noindent \begin{centering}
\includegraphics[width=10cm]{GPrf11Fig5} 
\par\end{centering}

\protect\protect\protect\caption{\label{fig:VarSpecWICandWIG}Variance spectra for WIC 
(red line labeled PSD, unsmoothed, and
also shown smoothed as the blue line), compared to the smoothed spectrum for
WIG from the gust pod (green line). Data from flight RF05, 9:15:00--9:40:00.}
\end{figure}


Figure \ref{fig:VarSpecWICandWIG} shows the high-rate variance spectra
from the two systems. There is a significant difference at frequencies
above about 3 Hz, with the gust-pod distribution dropping faster and
the standard wind WIC showing more variance. The high frequency spectrum
from the gust pod may be more realistic; it is unusual to see high
variance at these frequencies without a related generating source.
The coherence between the radome and gust-pod measurements was above 0.9
for frequencies less than 1 Hz but then fell to around 0.2 at 10 Hz.
This is an indication that the two measurements are different in important
ways at high frequency. This would not be the case if they were
responding with different amplitudes; the signals must really be mostly
incoherent at the highest frequency. The separation wing-to-fuselage
is about 7 m lateral and 13 m longitudinal, so that doesn't seem enough
to cause the low high-frequency coherence. The phase changes from
in-phase at frequencies less than 1 Hz to 180$^{\circ}$ out-of-phase
at 9 Hz, with WIG lagging, so this is consistent with the longitudinal
offset. Shifting WIG relative to WIC also gave maximum coherence when
WIG was shifted forward 1/25 s. Figure~\ref{fig:FurtherExpandedWIGshifted}
shows an example of the good correspondence between gust-pod and
radome measurements of vertical wind after application of such a shift to 
25-Hz measurements.

\begin{figure}
\noindent \begin{centering}
\includegraphics[width=10cm]{GPrf11WIGshiftM1} 
\par\end{centering}

\protect\protect\protect\caption{\label{fig:FurtherExpandedWIGshifted}Comparison of WIG (red line)
and WIC (blue line) after shifting WIG 1/25 s earlier to compensate
for the longitudinal displacement of the sensor.}
\end{figure}


\subsubsection{Mixing of Attitude Angles for the Gust Pod}

% The gust pod does not yet perform well
% in turns. The reason is that the gust-pod IRU is oriented to point a few degrees from
% the longitudinal axis of the aircraft, and as
% a result any roll of the aircraft introduces complex combinations
% of roll, heading, and pitch into the gust-pod IRU. It should be possible
% to remove these to some extent, but a procedure for correction of the
% angles in turns has not yet been developed or implemented.

% The circle maneuvers
% from the calibration flight should provide valuable information regarding
% the potential for improving the in-turn performance of the gust pod.

The attitude angles (pitch, roll, heading) as measured by the gust-pod IRU are defined
relative to the orientation of the inertial unit in the gust pod,
which is aligned a few degrees from the longitudinal axis of the aircraft. The canisters
on the GV are oriented with axis pointing inward and downward relative to the
aircraft longitudinal axis, in order to align with normal airflow at the pods. This
is desirable for hydrometeor sampling and minimum drag, but it complicates the
calculation of wind because
roll introduced in turns, primarily a rotation
about the aircraft longitudinal axis, will appear as a combination
of attitude-angle changes in the gust pod. Errors arising from the
initial alignment at the start of flights will also cause problems
with the measured attitude angles, and it is likely that these will
be more significant near the start of flights because the built-in
Kalman filter uses GPS measurements to correct such errors in the
course of the flight.

This problem with the reference frame for attitude angles has
two consequences: 
\begin{enumerate}
\item Measurements from the gust pod in turns have large errors
in comparison to the errors in level flight. The measurements from
the gust pod should probably be flagged as of poor quality whenever
the roll exceeds some threshold like perhaps $\pm5^{\circ}$. The
measurements usually look reasonable in turns despite this worry, but
the largest errors occur there.
\item There may be an offset introduced by the mixing of sideslip and angle-of-attack,
arising from the difference in roll angle, 
and this will affect the reference or average value of the measured
vertical wind. Some flights (e.g., DEEPWAVE flight 18) have a significant
offset in vertical wind from the gust pod at the start of the flight that
is related to offsets in heading and roll, gradually corrected in the
course of the flight via the C-MIGITSIII Kalman filter.
\end{enumerate}
Because the weight of the aircraft decreases during the flight, so
does the angle-of-attack.
Because the wing flexes, the measured sideslip at the gust pod also varies with weight
of the aircraft. This change in sideslip causes an offset in the mean lateral
component of the wind as measured by the gust-pod system.

% as shown by Fig.~\ref{fig:SSGPrf18}, so there will be an offset
% in sideslip that will affect the mean lateral component of the wind.
% \begin{figure}
% \noindent \begin{centering}
% \includegraphics[width=10cm]{GPF1} 
% \par\end{centering}
% 
% \protect\protect\protect\caption{\label{fig:AOAGPrf18}Angle of attack measured by the gust-pod system
% on DEEPWAVE flight RF18.}
% \end{figure}
% 
% 
% \begin{figure}
% \noindent \begin{centering}
% \includegraphics[width=10cm]{GPF2RF18} 
% \par\end{centering}
% 
% \protect\protect\protect\caption{\label{fig:SSGPrf18}Sideslip angle measured by the gust-pod system
% on DEEPWAVE flight RF18.}
% \end{figure}
% 
% 
% In addition, this flight had a particularly large offset in the vertical
% wind from the gust pod (WIG) at the start of the flight, as shown
% in 
% \begin{figure}
% \noindent \begin{centering}
% \includegraphics[width=10cm]{GPF3} 
% \par\end{centering}
% 
% \protect\protect\protect\caption{\label{fig:WIGoffset}The vertical wind determined by the gust pod
% (WIG) and via standard processing (WIX, to distinguish this from WIC
% as determined by erronous in-field processing), for flight RF18 of
% the DEEPWAVE project.}
% \end{figure}
% 
% 
% Fig.~\ref{fig:WIGoffset}. This offset is associated with a brief
% period where the difference in heading between the IRU in the gust
% pod and that in the fuselage were unusually large, as shown in 
% \begin{figure}
% \noindent \begin{centering}
% \includegraphics[width=10cm]{GPF4} 
% \par\end{centering}
% 
% \protect\protect\protect\caption{\label{fig:HDGerror}The difference in heading between the values
% measured by the IRU in the gust pod and the IRU in the cabin, for
% DEEPWAVE flight RF18.}
% \end{figure}
% 
% 
% Fig.~\ref{fig:HDGerror}. In addition, the mean value of the measured
% roll was close to zero for the in-cabin IRU but about 1.375$^{\circ}$
% for the gust-pod IRU. Coupling among these attitude angles appears
% to have caused the large offset in WIG at the start of this flight.
% Because the gust-pod IRU incorporates a Kalman filter that can use
% measured errors in position and ground speed as determined by comparison
% to a GPS measurement to refine the measurements, the large error appears
% to have been corrected as the flight progressed.

These effects suggest that the vertical wind measured by the gust
pod may have an offset in some cases, particularly at the start of
flights, and the sideslip can also have an offset that will contribute
to the lateral component of the measured wind. These are weaknesses
in the measurements from the gust pod that, at this stage, do not
appear easily corrected. A future study implementing Schuler tuning
in a post-processing step and correcting for the entwined-angle effects
may be able to reduce these weaknesses, but that will require continued
analysis not performed for this report.\marginpar{XXX}
It may also be possible to develop special processing that corrects the measurement
of attitude angles in turns, but that has not yet been developed or implemented.

% 
% \subsubsection{Next Steps}
% \begin{enumerate}
% \item A procedure for processing the gust-pod measurements will be needed.
% It may be possible to use code in gusto.c for this purpose or otherwise
% to implement the process used here into nimbus, because this is being
% done second-by-second as needed for nimbus. Alternately, the code
% used here can run after nimbus processing to add the gust-pod variables.
% If this course is followed, the R script could be improved because
% I didn't vectorize the angle transformations yet, in the interest
% of getting it to work first. The software / data processing group
% will have to decide which is better, and it may be preferable to use
% different procedures for DEEPWAVE and for future use in order to produce
% the DEEPWAVE data quickly. 
% \item As noted above, there are issues with the gust-pod measurements in
% turns that need to be addressed. These issues will affect the standard
% data also because in turbulence the roll changes and this may feed
% into the other measurements even at small roll angles. 
% \item The relative timing of the measurements entering the calculation of
% wind still needs to be addressed. The IRU information is transmitted
% for recording with some delay, and corrections for those delays are
% in place. However, small errors in those delays can cause phase shifts
% among the measurements that, for example, result in residual perturbations
% of the vertical wind during pitch maneuvers. These should be tuned
% to minimize those perturbations. 
% \item An additional worthwhile study will be to see if using GPS measurements
% to determine the Schuler oscillation can improve measurements of pitch
% and heading and so lead to improved wind measurements. There is some
% evidence of this problem in an apparent changing offset in the vertical
% wind, so any improvement here can be very useful to DEEPWAVE. \end{enumerate}

\subsection{The laser air-motion sensor\label{sub:LAMS-description}}

The laser air-motion sensor (LAMS) is still under development so results
presented here will be less extensive than for the other systems. The
characteristics and associated uncertainties in measured wind are discussed
by \cite{SpulerEtAl2011} and \cite{CooperEtAl2014}. Figure~\ref{fig:LAMSschematic} 
shows the one-beam LAMS as installed on the GV, and Fig.~\ref{fig:LAMS-4Beam} shows the
configuration of the three or four-beam version. \noindent\begin{center}%
\begin{figure}%
\includegraphics[width=0.9\columnwidth]{amt-2014-0031-f01-r}%
%\includegraphics[width=120mm]{amt-2014-0031-f01}
\caption{Diagram of the LAMS. Light generated by
the laser in the cabin is transmitted by optical fibres to a~wing
pod, where it is transmitted in a beam that has a~focal point well
ahead of the aircraft (farther ahead than suggested by this not-to-scale
diagram). The light backscattered from aerosol particles in the focal
region is collected by the lens, and a circulator mixes a~portion
of the transmitted signal with the returned signal. The resulting
signal, with interference patterns that measure the Doppler shift
of the backscattered light, is returned via optical fibre to the cabin
for digitization. Also illustrated in this figure are the approximate
locations of the static pressure ports and the fuselage pitot tube
used by the research data system to measure static and dynamic pressures.
This figure appears in Applied Optics in the article by \citet{SpulerEtAl2011}
and is used here with permission from the Optical Society of America.}%
\label{fig:LAMSschematic}%
\end{figure}%
\end{center}%
\noindent\begin{center}%
\begin{figure}%
\includegraphics[width=0.59\columnwidth]{LAMS-1r}%
\includegraphics[width=0.4\columnwidth]{LAMS-2r}%
\caption{The four-beam configuration of the LAMS optical head as installed in a GV 
under-wing pod.}%
\label{fig:LAMS-4Beam}%
\end{figure}%
\end{center}%
The system consists of
one, three or four fiber-based laser beams focused ahead of the aircraft and
a collection system to detect the Doppler shift in light backscattered
from aerosols. The transmitter optical components are mounted in an
underwing pod, similar to that used for the gust pod, and like the
gust pod the system incorporates a compact C-MIGITS inertial reference
unit to measure the attitude angles and ground speed of the pod.
Early measurements from this system have been used to determine corrections
to the pressure measurements, and those serve an important role in
reducing uncertainty in the wind measurements, as described in
\cite{CooperEtAl2014} and later in this report.

LAMS hardware supports up to four beams, with three pointing $35^{\circ}$ off the forward direction, with $120^{\circ}$ separation about the azimuthal direction.  The fourth beam is directed forward from the LAMS unit.  The LAMS channel designations changed after DEEPWAVE (summer 2014) when it was decided the instrument would benefit from having a lower noise channel for forward pointing measurements (the laser/detector on channel 4 had lower SNR than the other three channels).  Table \ref{tab:BeamConfig} summarizes the labels and pointing angles of each beam for both up to and after DEEPWAVE.  Note the number designation refers to the laser, detector, and data processing channels.  The letter designation is associated with the exit window on the LAMS head.  The pointing angles reported here are ideal.  The angle $\theta$ is measured relative to the forward direction of the LAMS pod (approximately the angle between the beam and $\hat{x^{\prime}}$) and the angle $\phi$ is the rotation angle about $\hat{x^{\prime}}$ relative to the $\hat{z^{\prime}}$ axis.  This geometry is designed for GV installation where the LAMS pod is installed on the left wing, outboard pylon, outboard position.  The pointing geometry relative to the aircraft may change based on the installation configuration on the C-130 (to be determined on ARISTO in fall 2015) where the LAMS optical fiber are routed to the aircraft right wing.

\begin{table}
\begin{minipage}[t]{0.93\columnwidth}%
\begin{center}
\begin{tabular}{>{\centering}p{3.5cm}>{\centering}p{3.5cm}>{\centering}p{2.5cm}>{\centering}p{2.5cm}}
\toprule 
\textbf{Beam Name\\ After DEEPWAVE} & \textbf{Beam Name\\ Before/During DEEPWAVE} & \textbf{$\theta_L$} & \textbf{$\phi_L$} \tabularnewline
\midrule
\midrule 
Beam 1/A & Beam 1/A & $35^{\circ}$ & $180^{\circ}$ \tabularnewline
\midrule 
Beam 4/B & Beam 2/B & $35^{\circ}$ & $60^{\circ}$ \tabularnewline
\midrule 
Beam 3/C & Beam 3/C & $35^{\circ}$ & $-60^{\circ}$ \tabularnewline
\midrule
Beam 2/D & Beam 4/D & $0^{\circ}$ & -- \tabularnewline
\bottomrule
\label{tab:BeamConfig}
\end{tabular}
\end{center}
\end{minipage}
\caption{Designations and ideal orientations of LAMS beams for installation on the GV. 
$\theta_L$ is the angle from the longitudinal axis of the aircraft to the beam, and
$\phi_L$ is the azimuthal angle of rotation about the longitudinal axis relative to
the downward (aircraft-axis) direction, with positive values to starboard. In
particular, beam 1 is upward at an angle of 35$^{\circ}$ from the centerline
of the aircraft.}
\end{table}

During IDEAS-4-GV, the LAMS beams were focused approximately 15 m from the LAMS exit ports on the pod.  After this flight program, concerns were raised about the aircraft causing flow distortion in the LAMS sample volumes, and the beam focus was moved to 20 m from the exit port.  In both cases the sample volume is approximately 2 m long.

\subsubsection{Relative Beam Pointing}

The relative pointing of the LAMS beams is defined by the orientation of each beam relative to the others.  If the absolute pointing direction of one beam is known, the relative pointing may be used to determine the absolute pointing of all other beams.  We define the relative pointing coordinates such that the z-axis is directed along the forward beam (D).  The angle $\theta_L$ is the angle between the beam and the z-axis and the azimuthal angle $\phi_L$ is measured relative to the x-axis such that beam 1 (A) is positioned at $\phi_L = 0$.

Two methods were used to determine the relative pointing angles.  The first, performed prior to IDEAS-4-GV, used a Laser Survey, in which a commercial laser surveying company measured the positions of the LAMS head and the beam focal points.

The second method used a 30x beam expanded with focusing lens in a fixed position (referred to here as the receiver system) while LAMS was mounted in an astronomical telescope mount.  A camera was placed at the focal point of the receiver system and the telescope mount was used to steer the LAMS beams into the receiver system such that each beam was centered on the CCD.  This method had redundant pointing measurements, using both C-MIGITS data as well as the telescope mount angle read-outs.

Both methods were relatively accurate, with less than $0.2^{\circ}$ difference in beam pointing.  It should be noted that the fibers on the LAMS head may have moved slightly between the two measurement methods.  The forward pointing beam was added after IDEAS-4-GV which required some disassembly of the LAMS head.  So the relatively small difference in beam pointing may be partially attributable to that work.

%The beam pointing results from the telescope measurement system are typically used going forward, first because they are the most recent beam pointing measurements, but also because redundant measurements and the position invariance of Fourier plane measurements generally illicit greater confidence in those pointing results.

%\subsubsubsection{Laser Survey}

Prior to IDEAS-4-GV in Sept.--Oct.~2013, a laser survey system was used to estimate the LAMS beam pointing angles of Beams 1, 2, and 3 (the forward pointing beam had not been added at that point).  The LAMS head was positioned in the hanger so the bottom two beams were approximately parallel to the floor.  Two IR card targets were placed at the focal points of those beams.

The laser survey system locked onto a corner cube reflector and register its position in space.  The corner cube was first used to estimate the position of the LAMS exit ports by taking a series of points around their circumference.  The tracker then registered the positions of the IR card targets by the same method.  Once complete, the LAMS head was rotated by $120^{\circ}$ and the process was repeated.  This measurement was repeated for three different LAMS head rotations, where two beams were surveyed at a time.

After data was collected, the beam exit and target information were used to provide a best fit for the beam pointing.

The benefits of this method are
\begin{itemize}
\item It measures the beam focal distance.

\end{itemize}

The weaknesses of this method are
\begin{itemize}
\item Only two beams can be surveyed at any one time.

\item The measurement process can perturb the beam target.

\item There is no reference to the C-MIGITS IRU.

\item There is no independent redundant measurement and uncertainty is difficult to quantify.

\item The survey can be expensive, requiring the hire of an outside vendor.
\end{itemize}


The results of the laser survey method were used to process IDEAS-4-GV data, but those angle calibrations have since been traded for the Telescope technique described below.


%\subsubsubsection{Telescope Measurement}

The relative positions of the LAMS system can be determined by defining a fixed reference vector and directing each beam onto it.  The pointing of the LAMS module is recorded when each beam is directed on the reference.  Without absolute knowledge of the reference vector, this only allows the relative beam geometry to be determined.  Thus, we define a LAMS beam pointing coordinate frame such that the z-axis is along the forward beam.

The telescope beam measurement system was developed in house to provide position insensitive angular measurement.  It uses a 30x beam expander consisting of an off axis parabola and convex secondary (exact type is unknown).  The light from that beam expander is focused onto an IR card using a 50 mm lens, yielding a total focal length of 1.5 m.  The IR card was then reimaged using a 45 mm lens onto a CCD to monitor the beam position (see Figure \ref{fig:TelescopePointingRX} and \ref{fig:TelescopePictures}).

\begin{figure}
\noindent \begin{centering}
\includegraphics[width=10cm]{BeamPointing_ReceiverDiagram-r.png} 
\par\end{centering}
\protect\protect\protect\caption{\label{fig:TelescopePointingRX}Diagram of the receiver system.  The invisible infared beam enters the beam expander which amplifies the angle of the beam relative to the telescope optic axis, and is then focused onto an IR card.  The IR card fluoresces at a visible wavelength and that light is reimaged onto a CCD.}
\end{figure}

\begin{figure}
\noindent \begin{centering}
\includegraphics[width=10cm]{BeamPointingCollection-r.png} 
\par\end{centering}
\protect\protect\protect\caption{\label{fig:TelescopePictures}Photographs of the telescope angle measurement system in the RAF optics lab.  View of the setup from behind the LAMS pod (top left), the receiver system (bottom left), LAMS mounted in the telescope mount (right).}
\end{figure}

With the IR card at the back focus of the optical system, the beam image is position independent.  Its location in this plane is dictated solely by the angle of the input beam.  Even though each beam is slightly translated relative to its counterparts, we can still perform accurate angle measurements.

The LAMS system is mounted in a precision telescope mount with accuracy on the order of $1^{\prime\prime}$ or about $0.0003^{\circ}$.  The mount gives fine resolution in adjusting LAMS pointing into the receiver reference system and provides a digital readout of Alt/Az coordinates.  These coordinates provide the first set of pointing data for the LAMS system.  The second set of pointing data is provided by the C-MIGITS INS, which is attached and operating during our data collection.  Thus, two measurements of the LAMS pointing are provided.

The telescope mount readout provides the quantities Alt, $\vartheta_T$, and Az, $\varphi_T$.  When a beam is directed into the reference telescope, it is accomplished through rotation operations:
\begin{equation}\label{TelMountReference}
\hat{r} = \mathbf{R}(\vartheta_T,\hat{x})\mathbf{R}(\varphi_T,\hat{y})\hat{u},
\end{equation}
where $\hat{r}$ is the reference direction of the receiving telescope, $\hat{u}$ is the beam pointing direction and $\mathbf{R}(\theta,\hat{v})$ is a rotation matrix of angle $\theta$ about the vector $\hat{v}$.

To determine the beam pointing angle, we invert the rotation operations (or perform the opposite rotations in the opposite order), which results in
\begin{equation}\label{TelMountBeam}
\hat{u} = \mathbf{R}(-\varphi_T,\hat{y})\mathbf{R}(-\vartheta_T,\hat{x})\hat{r}.
\end{equation}
To obtain a beam pointing vector, we first need to know the reference vector in some coordinate basis.  We let Beam 4 (the forward pointing beam) define the z-axis in this coordinate basis and obtain $\hat{r}$ by evaluating Eq. \eqref{TelMountReference} for the recorded Alt/Az angles of Beam 4 (D) where $\hat{u} = \left[\begin{array}{ccc} 0 & 0 & 1 \end{array}\right]^{T}$.  We then obtain all other beam vectors from their respective Alt/Az coordinates using Eq. \eqref{TelMountBeam}.

The process for obtaining the beam pointing vectors from C-MIGITS is essentially identical to the process above, except that C-MIGITS provides roll, pitch and heading.  It should be noted, as will be addressed later, the C-MIGITS heading is not reliable when the device is stationary.

After the pointing angles of the beams are obtained from the telescope mount and C-MIGITS, the pointing angles need to be compared.  However, the beam vectors are recorded in different coordinate frames and C-MIGITS heading data cannot be treated as reliable.  A transformation matrix between these two frames and heading adjustments to C-MIGITS are determined using minimization of errors so we can compare the beam pointing results.  Table \ref{tab:AngleDiff} shows the difference in angle between each of the four beams after optimizing the transformation between the two coordinate frames.  The two beam pointing measurements give results that are quite close. 

\begin{table}
\begin{minipage}[t]{0.93\columnwidth}%
\noindent\begin{center}
\begin{tabular}{>{\centering}p{2.5cm}>{\centering}p{7cm}}
\toprule 
\textbf{Beam} & \textbf{Angle difference between\\ C-MIGITS and Telescope Mount}\tabularnewline
\midrule
\midrule 
Beam 1 (A)\\ (Upward) & $0.024^{\circ}$ \tabularnewline
\midrule 
Beam 2 (B)\\ (Down-In) & $0.004^{\circ}$ \tabularnewline
\midrule 
Beam 3 (C)\\ (Down-Out) & $0.003^{\circ}$ \tabularnewline
\midrule
Beam 4 (D)\\ (Forward) & $0.024^{\circ}$ \tabularnewline
\bottomrule
\label{tab:AngleDiff}
\end{tabular}
\end{center}
\end{minipage}
\end{table}

An assessment of the beam pointing accuracy was performed by translating the focused beam across the IR card and finding the spread of angles accepted by the optical system.  The full angle field-of-view of the system was approximately $2^{\prime}$ or about $0.02^{\circ}$.  The beam spot position can be repeated to greater accuracy than this (we can see when it is well centered using the CCD), however this is probably a reasonable uncertainty figure because the IR card may not be located at the exact Fourier plane of the optical system.
\vfill\eject

The benefits of the telescope beam measurement method are:
\begin{itemize}
\item All four beams are measured without changes to the setup.

\item The measurement is "hands off", so there is very little risk of perturbing the system during the measurement processes.

\item The processes provides redundant angle measurements (C-MIGITS INS and the telescope mount).

\item The entire LAMS pod is used and referenced directly to the same C-MIGITS INS used in flight.
\end{itemize}

The weaknesses of this method are:
\begin{itemize}
\item Beam focal positions are not measured.

\item Setup and alignment of the system is time consuming (approximately 2 days).

\item The procedure still does not provide an absolute pointing measurement because the telescope acceptance vector is not known.
\end{itemize}


\begin{table}
\footnotesize
\begin{minipage}[t]{0.93\columnwidth}%
\begin{tabular}{>{\centering}p{1.8cm}>{\centering}p{1.8cm}>{\centering}p{1.8cm}>{\centering}p{1.8cm}>{\centering}p{1.8cm}>{\centering}p{1.8cm}>{\centering}p{1.8cm}}
\toprule 
\textbf{Beam} & \textbf{$\theta_L$ Telescope} & \textbf{$\theta_L$ C-MIGITS} & \textbf{$\theta_L$ Laser Survey} & \textbf{$\phi_L$ Telescope} & \textbf{$\phi_L$ C-MIGITS} & \textbf{$\phi_L$ Laser Survey}\tabularnewline
\midrule
\midrule 
Beam 1 (A)\\ (Upward) & $35.03^{\circ}$ & $35.08^{\circ}$ & $34.95^{\circ}$ & --\footnote{Beam 1 is used as the basis for $\phi = 0$ in the relative pointing coordinate frame} & -- & -- \tabularnewline
\midrule 
Beam 2 (B)\\ (Down-In) & $34.86^{\circ}$ & $34.85^{\circ}$ & $34.96^{\circ}$ & $119.74^{\circ}$ & $119.74^{\circ}$ & $120.01^{\circ}$\tabularnewline
\midrule 
Beam 3 (C)\\ (Down-Out) & $34.86^{\circ}$ & $34.85^{\circ}$ & $35.01^{\circ}$ & $120.16^{\circ}$ & $120.16^{\circ}$ & $120.00^{\circ}$\tabularnewline
\bottomrule
\label{tab:PointingAngles}
\end{tabular}
\end{minipage}
\end{table}

The following uncertainty analysis treats unknown biases in pointing as random variables with standard deviations or variances.  These figures represent uncertainty, not stochastic processes.  The resultant variances provide bounds on the LAMS wind vector accuracy based strictly on the instrument characterization, independent of other factors such as signal fidelity.

The accuracy to which we know the LAMS relative beam pointing directly impacts our estimates of airspeed and wind.  A sequence of line-of-sight velocity measurements are given by the equation:
\begin{equation}\label{Vmeas}
\vec{m} = \mathbf{U}\vec{v}_L,
\end{equation}
where $\vec{m}$ is the set of measurements (3 or 4 elements corresponding to the number of beams in use), $\vec{v}_L$ is the air velocity vector in the LAMS beam coordinate basis and $\mathbf{U}$ is the matrix describing each beam's pointing angle given by
\begin{equation}\label{Umatrix}
\mathbf{U} = \left[\begin{array}{cccc} \hat{u}_1 & \hat{u}_2 & \hat{u}_3 & \hat{u}_4 \end{array}\right]^T,
\end{equation}
where $\hat{u}_i$ is the $ith$ beam pointing direction.  The exact vector entries for $\mathbf{U}$ are dependent on which beams are used for LAMS operation.  For example, $\hat{u}_4$ would not be included for HCR-TEST (post-DEEPWAVE configuration) where the down-inboard beam was not used in that three beam configuration.  For this analysis the beam and air velocity vectors are defined in the LAMS relative coordinate frame.  Though this is not inherently required, it allows the analysis presented in this section to flow directly into further uncertainty analysis described in Section \ref{sub:Absolute}.

The covariance matrix of a beam pointing vector is given by
\begin{equation}\label{CovUvec}
\Sigma_{\hat{u}_i}^2 = \left(\frac{\partial \hat{u}_i}{\partial \phi}\right)\sigma_{\phi}^2\left(\frac{\partial \hat{u}_i}{\partial \phi}\right)^T + \left(\frac{\partial \hat{u}_i}{\partial \theta}\right)\sigma_{\theta}^2\left(\frac{\partial \hat{u}_i}{\partial \theta}\right)^T,
\end{equation}
where $\sigma_{\phi}$ and $\sigma_{\theta}$ are the standard deviation of the beam pointing angles.  Note that the uncertainty in these angles may be different for each beam, but this analysis will treat them as identical.

The variance in a measurement due to beam pointing uncertainty is given by 
\begin{equation}\label{measUncert}
\sigma_{m_{ii}}^2 = \vec{v}_L^T\Sigma_{\hat{u}_i}^2\vec{v}_L.
\end{equation}  
The uncertainty in the relative pointing angles are assumed to be independent for each beam (common uncertainty in the instrument pointing will be addressed in section \ref{sub:Absolute}).  Thus the total measurement covariance matrix is diagonal and given by 
\begin{equation}\label{measCov}
\Sigma_{\vec{m}}^2 = \left[\begin{array}{cccc} \sigma_{m_{11}}^2 & 0 & 0 & 0 \\ 
						0 & \sigma_{m_{22}}^2 & 0 & 0 \\
						0 & 0 & \sigma_{m_{33}}^2 & 0 \\
						0 & 0 & 0 & \sigma_{m_{44}^2} \end{array} \right].
\end{equation}

The total velocity uncertainty in the LAMS coordinate frame is ultimately bounded by its covariance matrix
\begin{equation}\label{VlamsError}
\Sigma_{\vec{v}_L}^2 = \mathbf{U}_{inv}\left(\Sigma_{\vec{m}}^2 + \Sigma_{f}^2 + \mathbf{U}\Sigma_{\vec{v}}^2\mathbf{U}^T\right)\mathbf{U}_{inv},
\end{equation}
where the velocity uncertainty resulting from beam pointing uncertainty is given by the covariance matrix $\Sigma_{\vec{m}}^2$ from Eq. \eqref{measCov}, the uncertainty due to FFT Doppler peak estimation is described by the covariance matrix $\Sigma_{f}^2$ and is defined in section \ref{sub:FreqPrec}, the covariance matrix $\Sigma_{\vec{v}}^2$ is the result of velocity variability between the three or four beam sample volumes with $\mathbf{U}$ being the beam projection matrix from Eq. \eqref{Umatrix} and $\mathbf{U}_{inv}$ is the inversion matrix for finding the total velocity vector from the three or four line-of-sight measurements.  In the case of the three beam LAMS, $\mathbf{U}_{inv} = \mathbf{U}^{-1}$.  However, because the four beam system is overdefined, a pseudo-inverse may be used, or if the uncertainty in measurements are well known,
\begin{equation}\label{OptInv}
\mathbf{U}_{inv} = \left[\mathbf{U}^T\left(\Sigma_{L}^2\right)^{-1}\mathbf{U}\right]^{-1}\mathbf{U}^T\left(\Sigma_{L}^2\right)^{-1},
\end{equation}
where
\begin{equation}\label{TotalMeasUncert}
\Sigma_{L}^2 = \Sigma_{\vec{m}}^2 + \Sigma_{f}^2 + \mathbf{U}\Sigma_{\vec{v}}^2\mathbf{U}^T.
\end{equation}

Assuming the angle uncertainty of all four beams is approximately $0.02^{\circ}$ and an aircraft velocity of 200 m/s along the LAMS pointing direction, the total velocity uncertainty resulting from beam pointing uncertainty is approximately 0.06 m/s in the horizontal and vertical directions with the four beam configuration, 0.09 m/s in the horizontal and vertical directions with the three beam configuration used in HCR-TEST.  To first order approximation, The forward velocity is insensitive to small perturbations beam pointing.


Add note that C-MIGITS cannot provide reliable heading information without moving.  Heading adjustments had to be added to the fit of the two measurements.

\subsubsection{C-MIGITS/Absolute Beam Pointing\label{sub:Absolute}}

With the relative pointing angles of the beams, an air velocity vector can be retrieved reliably in the predefined LAMS coordinate frame.  However, the exact transformation between the LAMS beams and the C-MIGITS coordinate frame (defined by the unit's principal axes) is unknown.  We typically assume that one C-MIGITS axis is exactly aligned to the forward pointing beam and one is directed along the angle beam 1.  However, it stands to reason that there will be some slight differences between the LAMS beam coordinate frame and the principal axes of the C-MIGITS.  At present, the only method we have for determining this transformation relies on making small angle adjustments based on flight maneuvers.

Let the transformation matrix between the LAMS relative coordinate frame and the C-MIGITS principal axes be $\mathbf{T_{L}}$.  The velocity from LAMS is converted to the C-MIGITS coordinate frame using
\begin{equation}\label{velL2C}
\vec{v}_c = \mathbf{T_{L}}\vec{v}_L,
\end{equation}
where $\vec{v}_c$ is the air velocity vector in the C-MIGITS coordinate frame and $\vec{v}_L$ the air velocity vector in the LAMS relative beam coordinate frame.  To propagate error in the transformation matrix we reframe the problem by vectorizing the matrix such that Eq. \eqref{velL2C} becomes
\begin{equation}\label{velL2Cvect}
\vec{v}_c = \mathbf{V_L} \vec{t}_{L},
\end{equation}
where the matrix $\mathbf{T_L}$ has been converted to the vector $\vec{t}_L$ given by 
\begin{equation}\label{T2vec}
\vec{t}_L = \left[\begin{array}{cccc} T_{11} & T_{12} & \cdots & T_{33} \end{array}\right]^T,
\end{equation}
where $T_{ij}$ is the element of $\mathbf{T_L}$ from the $ith$ row and $jth$ column.  The matrix $\mathbf{V_L}$ is constructed from the LAMS coordinate frame velocity vector $\vec{v}_L$ and is given by
\begin{equation}\label{v2mat}
\mathbf{V_L} = \left[\begin{array}{ccccccccc} 	v_1 & v_2 & v_3 & 0 & 0 & 0 & 0 & 0 & 0 \\
						0 & 0 & 0 & v_1 & v_2 & v_3 & 0 & 0 & 0 \\
						0 & 0 & 0 & 0 & 0 & 0 & v_1 & v_2 & v_3 \end{array}\right],
\end{equation}
where $v_i$ is the $ith$ element of $\vec{v}_L$.

The covariance matrix of $\vec{t}_L$ can be estimated from the uncertainties in the roll, pitch and yaw transformation angles denoted here as $\gamma$, $\beta$ and $\alpha$ respectively using partial derivatives
\begin{equation}\label{tCov}
\Sigma_{\vec{t}_L}^2 = \left(\frac{\partial \vec{t}_L}{\partial \gamma}\right)\sigma_{\gamma}^2\left(\frac{\partial \vec{t}_L}{\partial \gamma}\right)^T + \left(\frac{\partial \vec{t}_L}{\partial \beta}\right)\sigma_{\beta}^2\left(\frac{\partial \vec{t}_L}{\partial \beta}\right)^T + \left(\frac{\partial \vec{t}_L}{\partial \alpha}\right)\sigma_{\alpha}^2\left(\frac{\partial \vec{t}_L}{\partial \alpha}\right)^T.
\end{equation}

Thus the velocity covariance matrix in the C-MIGITS coordinate frame is given by
\begin{equation}\label{CMvelError}
\Sigma_{\vec{v}_c}^2 = \mathbf{V_L}\Sigma_{\vec{t}_L}^2\mathbf{V_L}^T + \mathbf{T_L}\Sigma_{\vec{v}_L}^2\mathbf{T_L}^T,
\end{equation}
where $\Sigma_{\vec{v}_L}$ is obtained from Eq. \eqref{VlamsError}.

It should be noted that a similar analysis can be performed for the transformation between C-MIGITS and a Global or aircraft coordinate frame where uncertainties in C-MIGITS roll, pitch and heading are better known.

ADD DETAILS ON DETERMINING THIS TRANSFORMATION MATRIX FROM FLIGHT DATA.

With the C-MIGITS INS, the air velocity measurements can be transformed into a global coordinate frame.  The transformation matrix is determined using $\mathbf{T_1}$, $\mathbf{T_2}$ and $\mathbf{T_3}$ from the roll, pitch and yaw reported by C-MIGITS.  The analysis needed for this step is covered in Section \ref{sub:EarthRef}.



\subsubsection{Frequency Precision\label{sub:FreqPrec}}

The LAMS A/D samples each beam detection channel at 200 MHz and performs a 1024 point FFT.  The frequency resolution of the FFT is thus given by the sample rate divided by the number of data points
\begin{equation}
\Delta f = \dfrac{f_s}{N_s} = 195 \mathrm{kHz}.
\end{equation}
The Doppler shift measured on a particular beam is 
\begin{equation}\label{DopplerShift}
f_D = \frac{2}{\lambda}\hat{u}\cdot\vec{v},
\end{equation}
where $\lambda$ is the laser wavelength (1560 nm), $\hat{u}$ is the beam direction and $\vec{v}$ is the velocity vector of the air relative to the instrument.  A factor of two is included because the Doppler shift is imposed twice, first when the beam is absorbed by the aerosol and second when it is re-emitted.  Along the beam line-of-sight, each FFT bin corresponds to a velocity resolution of $\Delta v_{LOS} = 0.15 \mathrm{m/s}$. 
%With post processing the backscatter Doppler peak can be determined at sub-bin resolution.  %This is accomplished using linear interpolation on the derivative of the backscatter peak to find the derivative zero crossing. 
However, the peak in the detected Doppler spectrum can be determined with resolution of
about 0.3~bin
by using a fit to the derivative of the spectrum.
%For the purposes of this analysis, we assume a sub-sample resolution of about 0.3, 
%giving 
This gives a line-of-sight velocity resolution of about $\Delta v_{LOS} = 0.05 \mathrm{m/s}$.

The resulting covariance matrix for each line-of-sight velocity measurement (only accounting for frequency accuracy) is diagonal with identically distributed variances
\begin{equation}\label{CovFreq}
\Sigma_{f}^2 = \left[ \begin{array}{cccc} \sigma_{v_{LOS}}^2 & 0 & 0 & 0 \\ 
						0 & \sigma_{v_{LOS}}^2 & 0 & 0 \\
						0 & 0 & \sigma_{v_{LOS}}^2 & 0 \\
						0 & 0 & 0 & \sigma_{v_{LOS}}^2 \end{array} \right],
\end{equation}
where, for our purposes, we assume $\sigma_{v_{LOS}} = \Delta v_{LOS}$.

The analysis presented here assumes variance in the sample rate to be small compared to other error sources.

Note that at GV speeds, the Doppler shift is expected to exceed the Nyquist frequency at the present sample rate.  We use the true air speed measurements from the aircraft radome to determine which frequency fold contains the Doppler peak.

\subsubsection{Flow Distortion}

Initial operation of the three beam LAMS on IDEAS-4-GV revealed that the aircraft can influence flow fields in the LAMS sample volume.  This issue became recognizable when data processing showed the down-inboard beam registered significantly slower velocities than the other two beams.  An analysis using Gulfstream's computational fluid dynamics analysis confirmed that all three beams could be expected to observe some flow distortion.  The expected flow distortion depends on the aircraft flight parameters.  The down-inboard beam is expected to experience the largest effect, typically observing a line-of-sight flow effect between -0.5 and -2.0 m/s when the beam is focused at 20 m.  The upward beam may see flow effects on the order of $\pm 0.5$ m/s and the down-outboard may see flow effects between 0 and -0.5 m/s.

On the GV, LAMS no longer uses the down-inboard beam due to the substantial flow effects in its sample volume.  The flow effects around the C-130 will be better determined after the ARISTO flight campaign in Fall 2015.

\subsubsection{Uncertainty arising from separation of measurement volumes}


In turbulent conditions the three sensitive volumes can be characterized by slightly
different values of the wind vector $\mathbf{v}=\{u,v,w\}.$ The single
forward beam measures $u$ directly while the 3-beam system must solve
for $u$ using the relative wind measurements at three locations displaced
from each other. If there are variations in the wind vector at these
three locations, that will introduce an error that can be significantly
larger than the measurement errors for a single-beam-forward system.

If for simplicity it is assumed that the 3-beam system is aligned
so that the longitudinal axis matches the $u$ axis and the vertical
axis matches the $w$ axis, then the unique solution (for a 35$^{\circ}$
diverging-beam angle) for the true airspeed ($u$) is

\begin{equation}
u=\frac{(a_{1}+a_{2}+a_{3})}{3\cos(35^{\circ})}\label{eq:TAS3beam}
\end{equation}


where $a_{i}$ is the relative airspeed measured by the ith beam.%\footnote{The direction cosines for the three beams are given by the following matrix $\sigma$:
% 
% \[
% \sigma=\left(\begin{array}{ccc}
% \cos\Theta & \sin\Theta\sin\Phi & \sin\Theta\cos\Phi\\
% \cos\Theta & -\sin\Theta\sin\Phi & \sin\Theta\cos\Phi\\
% \cos\Theta & 0 & \sin\Theta
% \end{array}\right)
% \]
% where $\Theta=35^{\circ}$ and $\Phi=120^{\circ}$. The direction
% cosine matrix relates the wind components to the 3-beam measured components
% ($a_{i}$\} via 
% 
% \[
% \begin{pmatrix}a_{1}\\
% a_{2}\\
% a_{3}
% \end{pmatrix}=\mbox{\textbf{\ensuremath{\sigma}}}\begin{pmatrix}u\\
% v\\
% w
% \end{pmatrix}
% \]
% 
% 
% Inverting $\sigma$ then gives the wind components in terms of the
% measured components:
% 
% \[
% \begin{pmatrix}u\\
% v\\
% w
% \end{pmatrix}=\sigma^{-1}\begin{pmatrix}a_{1}\\
% a_{2}\\
% a_{3}
% \end{pmatrix}
% \]
% 
% 
% For the above angles, application of the inverse below to the component
% $u$ leads to (\ref{eq:TAS3beam}): 
% 
% \[
% \sigma^{-1}=\begin{pmatrix}0.4069249 & 0.4069249 & 0.4069249\\
% 1.006579 & -1.006579 & 0\\
% -0.5811489 & -0.5811489 & 0.9166754
% \end{pmatrix}
% \]
% }
If each beam measures relative airspeed in its direction of alignment
to an uncertainty $\delta$, then a one-beam system aligned along
the airflow measures with uncertainty $\delta$ while a three-beam
system measures to uncertainty $\sqrt{3}\delta/(3\cos(35^{\circ})=0.7\delta$,
so if each beam is an independent measurement the 3-beam system measures
TAS more accurately than a single-beam system. However, the unique
solution for the wind vector obtained from the 3-beam system relies
on the assumption that all three beams are viewing air that has the
same wind vector $\mathbf{v}$. If there is variation in the wind
vector at the three viewed locations, that variation is not necessarily
just variation in $u$ (that it might be desirable to average) but
can also result from other variations because the beams are not aligned
along the $u$ axis. 

Because the uncertainty $\delta$ is less than 0.1 m/s, variations
of this magnitude would introduce errors comparable to the measurement
error. The spatial separation
between any two sensitive volumes in the 3-beam system is about 
1.5($\sqrt{2}L\sin(35^{\circ})\simeq18$\,m for focal distance L=15 m. 
The variance in the wind for points separated
by 18 m can be estimated as follows:

\begin{equation}
(u^{\prime})^{2}=\int_{k_{0}}^{\infty}\alpha\epsilon^{2/3}k^{-5/3}dk=\alpha\epsilon^{2/3}\frac{3}{2}k_{0}^{-2/3}\label{eq:Uvariance}
\end{equation}


where $k_{0}=2\pi/\Delta$ with $\Delta=18$ m. For modest eddy dissipation
rates in the range $\epsilon=0.001$ to 0.01 m$^{2}$s$^{-3}$, and
for $\alpha=0.67$ (FIND REFERENCE XXX)\marginpar{XXX}, (\ref{eq:Uvariance}) results in estimates of
the velocity variance of about .02--0.1 m$^{2}$s$^{-2}$, or standard
deviations of about 0.14 to 0.3 m/s. These fluctuations, entering
(\ref{eq:TAS3beam}), will cause errors in the estimate of $u$ that
are not negligible in comparison
to the measurement errors in \{$a_{1},\,a_{2},\,a_{3}$\}.







%% LyX 2.1.2 created this file.  For more info, see http://www.lyx.org/.
%% Do not edit unless you really know what you are doing.
% \documentclass{article}
% \usepackage{mathptmx}
% \usepackage[T1]{fontenc}
% \usepackage[latin9]{inputenc}
% \usepackage{geometry}
% \geometry{verbose}
% \setcounter{secnumdepth}{2}
% \setcounter{tocdepth}{2}
% \setlength{\parskip}{\medskipamount}
% \setlength{\parindent}{0pt}
% \usepackage{booktabs}
% \usepackage{amsmath}
% \usepackage{amssymb}
% \usepackage{graphicx}
% \usepackage[unicode=true]
%  {hyperref}
% \usepackage{breakurl}
% 
% \makeatletter
% 
% %%%%%%%%%%%%%%%%%%%%%%%%%%%%%% LyX specific LaTeX commands.
% %% Because html converters don't know tabularnewline
% \providecommand{\tabularnewline}{\\}
% 
% \AtBeginDocument{
%   \def\labelitemii{\(\rightarrow\)}
% }
% 
% \makeatother
% 
% \begin{document}




\section{Uncertainty components and summary\label{sec:Uncertainty-components}}


\subsection{General structure of an analysis of uncertainty}

Here we follow a particular style for construction of an analysis of 
uncertainty by including these components:

\begin{enumerate}

\item{\underbar{A description of the measuring system.}} Section 2 of this report
serves this function by providing extensive discussion of each component
contributing to the measurement of wind, and it discusses what is known
about specifications for uncertainty associated with those components. 
It also includes description of the algorithm leading to components of the
measured wind, and it discusses the three independent systems available
for measuring wind on the GV. 
\item{\underbar{Tests and calibrations.}} Section 4 provides key information
on how calibration maneuvers are used to determine the sensitivity of
some of the measurements to components of the wind. This section on 
calibration is key to the uncertainty analysis because to a large extent
many of the potential errors from sensors are removed by this calibration,
and the calibration becomes the central factor affecting the final
uncertainty. Other intermediate sections discuss some specific tests
applied to the measurements in order to check or reduce the uncertainty
limits associated with these measurements. These sections (3, 5, and 6)
go somewhat beyond a conventional analysis of uncertainty in that there
are some new developments discussed there and some unconventional ways
of checking the measurements. 
\item{\underbar{Discussion of the elemental contributions to uncertainty.}} This discussion will follow in this section, but first it appears useful
to provide a more general although incomplete discussion of what is expected
from that detailed analysis. The standard tabulation of  elemental 
sources of uncertainty then follows in subsections \ref{sub:vw-elements} and
\ref{sub:hw-elements}.
\item{\underbar{Summary and comprehensive estimate of uncertainty.}} This follows
at the end of this section and presents the key conclusions of this study.
\end{enumerate}

Because three wind-measuring systems are characterized in this report,
they will be discussed separately. However, the standard system is the
radome-based system, so that will be treated in the most depth. The other
two systems, the under-wing gust pod and the LAMS, are new systems and
their characteristics are still being developed and explored, so their
discussion necessarily will be incomplete until additional flight data
are collected.



\subsection{Preliminary estimates of uncertainty\label{subsub:prelim}}


\paragraph{Vertical wind.}

Because the calculation of wind from the contributing measurements
involves coordinate transformations, evaluation of the uncertainty
in wind components involves difficult error propagation through the
transformation matrices and other equations of Sect.~\ref{sub:General-comments}.
For application to straight-and-level flight (or flight where the
intent is to remain level), simplified equations suffice for evaluation
of the error terms, but it will also be useful to employ Monte-Carlo
simulations to be sure that error contributions are propagated correctly.

For the vertical wind, Eq.~\ref{eq:Weq} provides an approximate
relationship that leads to straightforward error propagation, esp.~if
it is assumed that the angle of attack $\alpha$ and pitch $\theta$
are small angles so that

\begin{equation}
w=V(\alpha-\theta)+w_{p}\label{eq:Weq-1}
\end{equation}
and errors in $w$ ($\delta w$) can be related to the errors in the
basic measurements ($\delta V$, $\delta\alpha$, $\delta\theta$, $\delta w_{p}$)
by differentiating Eq.~\ref{eq:Weq-1}: 
\begin{equation}
\delta w=(\alpha-\theta)\delta V+V(\delta\alpha-\delta\theta)+\delta w_{p}\,\,\,.\label{eq:delta-w}
\end{equation}
Correlations among these error terms are possible, so a full evaluation
that does not assume independence among the errors leads to:

\begin{eqnarray}
\left\langle (\delta w)^{2}\right\rangle  & = & (\alpha-\theta)^{2}\left\langle (\delta V)^{2}\right\rangle +V^{2}\left(\left\langle (\delta\alpha)^{2}\right\rangle +\left\langle (\delta\theta)^{2}\right\rangle \right)+\left\langle (\delta w_{p})^{2}\right\rangle \nonumber \\
 &  & +2\left((\alpha-\theta)\left(V(\left\langle \delta V\delta\alpha\right\rangle -\left\langle \delta V\delta\theta\right\rangle )+\left\langle \delta V\delta w_{p}\right\rangle \right)\right)\nonumber \\
 &  & +2V\left(\left\langle \delta\alpha\delta w_{p}\right\rangle +\left\langle \delta\theta\delta w_{p}\right\rangle \right)-2V^{2}\left\langle \delta\alpha\delta\theta\right\rangle \label{eq:delta-w-sq}
\end{eqnarray}


The approximate magnitudes of these terms, discussed in detail below,
are: $\delta V$=0.1~m/s, $\delta\alpha=0.1^{\circ}\simeq2$~mrad,
$\delta\theta\simeq1$~mrad, and $\delta w_{p}=0.03$~m/s. Other typical
magnitudes are $V\simeq200$~m/s, $\alpha\simeq\theta\simeq2^{\circ}\simeq3.5$~mrad.
For these typical magnitudes, the only terms in Eq.~\ref{eq:delta-w-sq}
that make potentially significant contributions are:

\begin{eqnarray}
\left\langle (\delta w)^{2}\right\rangle  & = & V^{2}\left(\left\langle (\delta\alpha)^{2}\right\rangle -2\left\langle \delta\alpha\delta\theta\right\rangle +\left\langle (\delta\theta)^{2}\right\rangle \right)\label{eq:delta-w-sq-simplified}
\end{eqnarray}


The error in pitch arises from measurements from the IRU and is affected
mostly by an initial offset during alignment and then further changes
in this error arising from the Schuler oscillation and from accelerations
that affect this error. The error in angle-of-attack, on the other
hand, arises from a combination of measurement error from the pressure
transducers and error in the formula used to deduce angle-of-attack
from the pressure measurements. This error is not independent of the error
in pitch, however, because the calibration as presented in Section \ref{sub:Calibration-AOA}
relies on the measured difference between pitch and angle of attack being zero
when the vertical wind and vertical aircraft motion are zero. Thus any bias in pitch
is transferred to a bias in angle of attack, so this component of the error in \eqref{eq:delta-w-sq-simplified} should cancel. The remaining errors arise from sources that are independent
and likely uncorrelated, so the middle term on the right side of \eqref{eq:delta-w-sq-simplified}
will be neglected in this preliminary analysis.

It is possible to estimate the magnitudes of these errors from measurements.
Because the GV instrumentation includes two identical inertial systems,
the difference in measurements of pitch from those two systems provides
one lower-bound estimate of the uncertainty in pitch. Measurements
discussed in Sect.~\ref{sub:Attitude-angles} suggest that, for periods
when the roll of the aircraft was within 5$^{\circ}$ of zero, the
difference between these redundant measurements was about 0.3~mrad or less than 
about 0.02$^{\circ}$. Section \ref{sub:Schuler} presents additional evidence from evaluation of the Schuler coupling between pitch or roll errors and ground-speed errors, also indicating
that typical pitch errors are of about this magnitude. This is significantly smaller
than the manufacturer's specified uncertainty for this instrument, $\pm 0.05^{\circ}$.

This uncertainty in pitch is comparable to the estimated random uncertainty in angle-of-attack,
for which the variance spectra discussed in Sect.~\ref{sub:Variance-spectra-for-W-components}
suggest a random error of about 0.02$^{\circ}\simeq0.35$~mrad.%
\footnote{DEEPWAVE flight 15, 3:40--3:55, spectrum for AKRD, compared to simulated
spectrum with random-noise amplitude of 0.07. As generated, this has
mean of half the amplitude, so it corresponds to a standard deviation
of .07/$\sqrt{12}\simeq0.02^{\circ}\simeq0.35$~mrad.%
} The estimated uncertainties in angle-of-attack and pitch are thus
comparable, and both influence the uncertainty in vertical wind. 

The rough-estimate uncertainty in vertical wind (to be refined later)
is then, for flight at about 230~m/s, about $230\sqrt{0.0003^{2}+0.00035^{2}}\simeq0.1$~m/s.
Other factors, esp.~the radome calibration, need to be considered
for a refined estimate, as presented in Section \ref{sub:vw-elements}.


\paragraph{Horizontal wind}

Errors in measurements affect the longitudinal and lateral components
of the horizontal wind in different ways. For the longitudinal component (along the
aircraft longitudinal axis),
the relative-wind measurement (essentially the true airspeed) 
has been calibrated by comparison to
measurements of the single-beam LAMS, leading to an estimated standard uncertainty
of about 0.1~m/s (Cooper et al., 2014). The LAMS measurement is based on measurement
of the Doppler shift and thus has low limits for bias and precision, dependent
primarily on the uncertainty associated with determining the peak in the returned
Doppler spectrum in the presence of turbulence. The lateral component is
measured using the sideslip angle, which is determined in a manner
similar to that used to calibrate the measurement of angle-of-attack.
However, the zero reference angle for attack is determined well by
assuming that, on average in quiescent air, the vertical wind should
be zero. No similar reference angle exists for sideslip, and trim
adjustments to the aircraft can change the mean sideslip angle, so
a different procedure with additional sources of uncertainty must
be used. 

Two maneuvers are particularly effective for determining
the zero reference for sideslip (at which angle the lateral horizontal
component of the relative wind would be zero): (i) Reverse-heading
maneuvers, where a straight leg is flown for a short time (typically
2 min), then the aircraft reverses heading and flies back along
the reverse heading, with the result that the wind component perpendicular
to the longitudinal axis of the aircraft should reverse sign; or (ii)
constant-bank circles flown drifting with the wind, for which the
lateral component of the wind should exhibit a sinusoidal variation
around the circle reflecting a possible heading and/or sideslip error. 
Both maneuvers can provide a reference for correcting a combination
of heading and sideslip errors, and the circle maneuver provides a
mechanism for determining the sideslip error alone, as discussed
in Sect.~\ref{sub:Analysis-of-circle}. There the uncertainty
associated with the bias in sideslip was indicated to be about 0.06$^{\circ}$
and the random uncertainty in heading was estimated to be less than 0.09$^{\circ}$,
with a bias limit comparable to that in sideslip.

As for vertical wind, the GPS-measured components of aircraft motion
relative to the Earth have uncertainty significantly less than that
of the relative wind, so errors are dominated by those arising from
the relative-wind components and from the transformation to Earth coordinates. 
The first-order equation for the lateral
component of the relative wind ($v_{y}$), analogous to Eq.~\ref{eq:Weq-1},
is $v_{y}=V\beta$ where $\beta$ is the sideslip angle. $V$ is
known to typically 0.05\% so the error is dominated by that in sideslip and
in heading.
For the random error, one limit is obtained as for angle-of-attack
by examining the variance spectrum in calm conditions. However, BDIFR
and SSLIP do not exhibit such pure noise spectra at high frequency
as did ADIFR and AKRD. The QCF variance spectrum has characteristics of noise
for amplitudes of random errors of about 0.15~hPa, slightly higher than estimated
for ADIFR. For BDIFR, an example is shown in Fig.~\ref{fig:bdifrSpectrum}.
\begin{figure}
\noindent \begin{centering}
\includegraphics[width=0.9\textwidth]{BDIFRspectrum} 
\par\end{centering}

\protect\protect\caption{\label{fig:bdifrSpectrum}Variance spectrum for BDIFR for DEEPWAVE
flight 15, 3:40--3:55 UTC. The dashed orange lines indicate expected
white-noise spectra for respective standard deviations of 0.1 and
0.05~hPa. Units for $fP(f)$ are hPa$^{2}$ per logarithmic interval
in frequency.}
\end{figure}


The slope in the high-frequency region does not match that expected
for a white-noise signal, for which the variance would be constant
and in these plots, where the spectral density is multiplied by the
frequency, the plotted line should increase linearly with frequency
parallel to the orange dashed lines. Instead, the spectrum for BDIFR
has smaller slope than this, looking perhaps as if the spectrum starts
out at a point where a white-noise spectrum with standard deviation
0.1~hPa would be appropriate, but then is filtered or otherwise smoothed%
\footnote{perhaps by the response in the lines connecting the ports to the pressure
sensors%
} so that at the high-frequency limit the equivalent white-noise standard
deviation would be 0.05~hPa. The spectrum for SSLIP looks similarly
filtered, with an equivalent white-noise spectrum somewhere in the
range of 0.02$^{\circ}$, similar to that characterizing AKRD. However, the
estimated uncertainty in the bias for the combined effects of sideslip and
heading was, from the circle analysis, 0.02$\pm 0.09^{\circ}$, and these
lead to contributions to the uncertainty in bias of the lateral wind of
about 0.3~m/s. The horizontal wind thus has an asymmetrical estimated
uncertainty, about three times as large for the lateral component as for
the longitudinal component.

Like vertical wind, this uncertainty estimate for the horizontal wind
needs to be refined for application
to the Earth-relative wind by consideration of the uncertainty in
the determination of sensitivity coefficients for sideslip and other 
contributions to the net uncertainty in wind.


\subsection{Elemental sources, vertical wind\label{sub:vw-elements}}

Next, we tabulate the elemental sources of uncertainty in the measurement
of vertical wind. 

\subsubsection{Radome-based system}

Table~\ref{tab:Elemental-w}
summarizes the results for the radome-based wind system. 
The following is a discussion of the individual
elements in that table.

\begin{table}
\begin{tabular}{cccccc}
\toprule 
\textbf{element} & \textbf{uncertainty source}  & \textbf{bias}  & \textbf{random}  & \textbf{$\delta w$ bias} & \textbf{$\delta w$ random }\tabularnewline
& & & & [m/s] & [m/s] \tabularnewline
\midrule
\midrule 
1 & ADIFR transducer  & 0.07~hPa  & 0.002~hPa  & -- & --\tabularnewline
\midrule 
2 & AKRD coefficients & 0.01$^{\circ}$ & 0.001$^{\circ}$ & 0.04 & 0.004\tabularnewline
\midrule 
3 & BDIFR transducer  & 0.07~hPa  & 0.002~hPa  & -- & --\tabularnewline
\midrule 
4 & QCF transducer  & 0.34~hPa  & 0.01~hPa  & <0.02 & 0.001\tabularnewline
\midrule 
5 & pitch  & 0.02$^{\circ}$ & 0.007$^{\circ}$  & 0.08 & 0.03\tabularnewline
\midrule 
6  & GV vertical velocity  & 0.03~m/s  & <0.03~m/s  & 0.03 & <0.03\tabularnewline
\midrule 
7 & PSF transducer  & 0.10\,hPa & 0.001\,hPa & -- & --\tabularnewline
\midrule 
8 & ATX & 0.3$^{\circ}$ & 0.1$^{\circ}$C  & -- & --\tabularnewline
\bottomrule
\end{tabular}

\protect\caption{Elemental contributions to the uncertainty in measurement of vertical
wind by the radome-based system. Entries '--' indicate negligible contribution to uncertainty.\label{tab:Elemental-w}}
\end{table}

\begin{enumerate}
\item \textbf{ADIFR: }See Sect.~\ref{sub:Airflow-angles} and Table~\ref{tab:Radome-system-measurements}.
The uncertainty is assigned to bias because it is likely
a calibration uncertainty and the resolution and stability are much
smaller than this bias. However, a calibration bias in this measurement
does not affect the final wind measurement because the procedure in
Sect.~\ref{sub:radome-sensitivity} determines the angle of attack
from flight data in a way that can be considered a calibration of
the measurement of angle of attack, and a bias in ADIFR would be reflected
in a change in sensitivity coefficients determined in that section
that would compensate for that bias. Only random errors in ADIFR
would propagate to the final measurement $w$, and such errors are
thought to be negligible, so the propagated error for ADIFR is listed
as negligible. The next item considers the determination of sensitivity
coefficient for angle of attack and is the dominant contribution to
uncertainty in $w$ arising from the measurement of angle of attack.
\item \textbf{AKRD coefficients:} The calibration procedure of Sect.~\ref{sub:radome-sensitivity}
effectively removes the effects of possible biases in ADIFR and QCF
and instead replaces them with uncertainties arising from the coefficients
\{$c_{0},\, c_{1}$\} in (\ref{eq:AOArecommended}). The estimated
bias and random error are those obtained and discussed in that section
using the estimated uncertainties in the coefficients in (\ref{eq:AOArecommended}).
For propagation to vertical wind $w$, (\ref{eq:VWind}) indicates
that the result is approximately $\delta w=V\delta\alpha$ where $V$
is true airspeed TASX, with additional contributions from correlated
errors involving $V$ that are small in comparison to that listed.
A typical value for V is about 240\,m\,s$^{-1}$, leading to the
listed elemental uncertainties in $w$ arising from uncertainty in
AKRD. It is important, though, that the calibration is dependent on the 
assumption that the mean vertical wind where the calibration data were 
collected be zero. This is discussed in Sect.~\ref{sub:radome-sensitivity}.
There is no independent way to check this except by 
comparing results from different regions as done in this report. That
remains a major weakness in calibration and is the major contributor to
uncertainty in angle of attack, but it introduces a bias arising from a
possible calibration error. Fluctuations from the mean value are measured with
uncertainty a factor of ten smaller than this bias. The assumed mean value of
the vertical wind leading to this bias was $\pm$0.03\,m/s, which could 
well be an over-estimate for the 
large datasets used in Section~\ref{sub:radome-sensitivity}. 
\item \textbf{BDIFR:} The sideslip angle has negligible effect on the vertical
wind as long as the roll angle is small, so for measurements made
during straight-and-level flight this contribution to uncertainty
in vertical wind is negligible.
\item \textbf{QCF:} The values listed are the characteristics of the transducer.
Application of the calibration procedure based on comparison to the
laser air motion sensor (\cite{CooperEtAl2014}) led to an alternate
uncertainty estimate of 0.3\,hPa. As in the case of ADIFR, the procedure
to determine sensitivity coefficients removes any effect of bias in
QCF by calibration in terms of the coefficients \{$c_{0},\, c_{1}$\}
so the effect on bias in $w$ is replaced by possible bias in those
coefficients, as discussed for element 2. The effect of a random error
in QCF of 0.01~hPa is, from (\ref{eq:AOArecommended}). to introduce
an uncertainty in angle of attack of about 0.0002$^{\circ}$ or a
contribution to uncertainty in $w$ of less than 0.001\,m\,s$^{-1}$.
QCF is also used to determine the true airspeed, which affects $w$,
but the effect is negligible for the estimated uncertainty in QCF
(<0.2\% of the measured value of $w$, or 0.02\,m\,s$^{-1}$ for
10\,m\,s$^{-1}$ vertical wind).
\item \textbf{PITCH:} The estimates listed are those that apply without
the pitch-correction procedure of Sect.~\ref{sub:Schuler}. In that section,
it was estimated that the standard error in pitch is 0.02$^{\circ}$ and that
this is primarily in the form of a slowly varying error that, over measurement
periods short compared to the Schuler oscillation period of 84.4~min, will 
appear as a bias. The correction procedure represented by
Eq.~\ref{eq:pitch-error-wo-approx} corrects for this error well enough to
leave the residual bias negligible, so the bias entry in Table~\ref{tab:Elemental-w}
can be eliminated by application of that algorithm. The partitioning
between bias and random uncertainty depends
on the interval considered, because Schuler precession will
cause variation in this error with the Schuler-oscillation period
of about 84 min. For periods long compared to this the error will
have the character of a random-error component, so using 0.08\,m/s would
be appropriate for random uncertainty of 
such long-term measurements while the bias should be
reduced substantially, perhaps to 0.02\,m/s. For periods small compared to
the Schuler period, the pitch error appears as a bias and there is a much
smaller random error, evaluated in Sect.~\ref{sub:attitude-angles} to be
about 0.007$^{\circ}$ in pitch or about 0.03\,m/s in vertical wind. 
This is the usual case for measurements of interest,
so the bias and random errors are partitioned as appropriate for this
case in the table. 
 The uncertainty in pitch
is the leading contributor to the standard uncertainty in vertical wind
and is also the leading contributor to the overall estimate of bias.
The correction technique of Sect.~\ref{sub:Schuler} is not
incorporated in routine processing so needs special calculation.
\item \textbf{Aircraft Vertical Velocity:} The measurement used for vertical
motion of the aircraft is discussed is Sect.~\ref{sec:VerticalVelocity}.
The values listed here are those specified for measurements when ``OmniSTAR''
corrections are available; if not, the values should be increased
to about 0.1\,m\,s$^{-1}$ and so will make a contributor to uncertainty
in vertical wind that is comparable to the contributions from pitch and angle 
of attack. The error in aircraft vertical speed is likely a mixture of bias and
random error, because the primary source is uncertainty in ionospheric
corrections which will be persistent for important parts of flights
but likely to change at least from flight to flight. Because of the
likely persistence of the error, it is assigned here primarily to
bias. 
\item \textbf{PSF:} The measured ambient pressure affects vertical wind
only through the dependence of true airspeed TASX on PSF, as described
in the document on \href{https://drive.google.com/file/d/0B1kIUH45ca5ATFV5d3QyQ0JpSjA/view?usp=sharing}{RAF processing algorithms},
Section 4.7.1. Evaluation at typical values shows that the dependence
of measured vertical wind on uncertainty in this variable is negligible.
For example, TASX for PSF=300\,hPa, QCF=80\,hPa, and ATX=$-40^{\circ}$C
differs from that for PSF=300.1 by 0.03\,m\,s$^{-1}$ or about 0.01\%,
so this would also be the percentage change in vertical wind. 
\item \textbf{ATX:} Temperature is needed to calculate TASX, but as for
PSF the effect is negligible. This was tested as for PSF by evaluating
at representative points. A representative result was that the listed
bias in temperature would lead to a bias in TASX of about 0.05\%,
leading to a similar percentage change in the value of the vertical
wind. This is negligible in comparison to other sources of uncertainty.
\end{enumerate}

The result of adding the elemental sources of uncertainty in quadrature
is a bias estimate of 0.10\,m\,s$^{-1}$ and a random-uncertainty estimate
of 0.04\,m\,s$^{-1}$, with pitch correction.
Without the correction, the bias estimate increases to 0.35\,m\,s$^{-1}$,
so the pitch correction results in significant reduction in uncertainty and without
that correction the uncertainty is dominated by the bias introduced by the measurement
of pitch. In the corrected case, the dominant contributions are those from pitch and
angle of attack, as was argued in the preliminary discussion in 
Section~\ref{subsub:prelim}. The bias and random errors from pitch and from
angle of attack, as listed here, arise from different sources so it is
reasonable to combine them in quadrature to obtain composite estimates.

The APPLANIX IRU should provide
another route to improvement in the measurement of pitch, because it offers
significantly lower specified uncertainty. It achieves this through use of
a full Kalman-filter correction to the measurements, which would remove the need
for the pitch correction proposed in this document. 

\subsubsection{Gust-pod system}

For the gust pod system, many of the uncertainties associated with measurement components
are known less well than for the radome system, but some similar estimates can be made.
This section will duplicate the structure of the radome-based system, but will be
less definitive and more sketchy in some of the components while emphasizing the 
differences that apply to the gust pod.

Table~\ref{tab:GP-Elemental-w} lists the elemental contributions to uncertainty in
the measurement of vertical wind from the gust-pod system.
The following is a discussion of the individual
elements in that table.

\begin{table}
\begin{tabular}{cccccc}
\toprule 
\textbf{element} & \textbf{uncertainty source}  & \textbf{bias}  & \textbf{random}  & \textbf{$\delta w$ bias} & \textbf{$\delta w$ random }\tabularnewline
& & & & [m/s] & [m/s] \tabularnewline
\midrule
\midrule 
1 & ADIF\_GP transducer  & 0.07~hPa  & 0.002~hPa  & -- & --\tabularnewline
\midrule 
2 & AK\_GP coefficients & 0.01$^{\circ}$ & 0.001$^{\circ}$ & 0.04 & 0.004\tabularnewline
\midrule 
3 & BDIF\_GP transducer  & 0.07~hPa  & 0.002~hPa  & -- & --\tabularnewline
\midrule 
4 & QCF/QC\_GP transducer  & 0.34~hPa  & 0.01~hPa  & 0.02 & 0.001\tabularnewline
\midrule 
5 & pitch  & 0.04$^{\circ}$ & 0.02$^{\circ}$  & 0.17 & 0.08\tabularnewline
\midrule 
6  & GV vertical velocity  & --  & 0.07~m/s  & -- & 0.07\tabularnewline
\midrule 
7 & PS\_GP transducer  & 0.10\,hPa & 0.001\,hPa & -- & --\tabularnewline
\midrule 
8 & ATX & 0.3$^{\circ}$ & 0.1$^{\circ}$C  & -- & --\tabularnewline
\bottomrule
\end{tabular}

\protect\caption{Elemental contributions to the uncertainty in measurement of vertical
wind using the gust pod. Entries '--' indicate negligible contribution to uncertainty.\label{tab:GP-Elemental-w}}
\end{table}

\begin{enumerate}
\item \textbf{ADIF\_GP: }See Sect.~\ref{sub:Airflow-angles} and Table~\ref{tab:Radome-system-measurements}.
See the discussion related to the radome. The same transducers are used for
the pressure measurements on the gust pod, although the configuration of
ports is different. The next item considers the determination of sensitivity
coefficient for angle of attack and is the dominant contribution to
uncertainty in $w$ arising from the measurement of angle of attack.
\item \textbf{AK\_GP coefficients:} The calibration procedure of Sect.~\ref{sub:gust-pod-aoa}
effectively removes the effects of possible biases in ADIFR and QCF
and instead replaces them with uncertainties arising from the fit coefficients
\{$b_{0--3}$\} in (\ref{eq:AOAGfunction}). It was necessary to use
additional terms to obtain a good fit in this case, but the final result provided
a very good representation of the data, as good as in the radome case.
We have therefore used the same uncertainty estimate as for the radome,
although with less justification and study. As for the radome, the dominant
source of bias is again the uncertainty in vertical wind in the calibration
region, which is the same for this data set as for that used to determine
the radome sensitivity coefficients, so this estimate remains the
same as for the radome. The uncertainties also propagate
to the vertical wind in the same way. However, the restriction to low roll angle
(less than 5$^{\circ}$ from vertical) is still more important in the case of
the gust pod because the CMIGITS IRU used with the gust pod is not aligned 
with the aircraft longitudinal axis but rather is mounted in an under-wing pod
that was designed to point into the airflow and therefore slightly inward
relative to the longitudinal axis. That causes significant problems in turns
because the IRU rotates in ways that mix the attitude angles.

\item \textbf{BDIF\_GP:} The sideslip angle has negligible effect on the vertical
wind as long as the roll angle is small, so for measurements made
during straight-and-level flight this contribution to uncertainty
in vertical wind is negligible.
\item \textbf{QCF and QC\_GP:} Two measurements are listed because both are used
in the calculation of vertical wind. QC\_GP is used with ADIF\_GP to 
determine the angle of attack, and the calibration described with item 2 effectively
replaces uncertainty in this measurement with uncertainty in the sensitivity
coefficients. However, true airspeed is determined using QCF because the
conventional true airspeed is thought to be superior to the new value determined
solely from the gust pod (Section~\ref{sub:GP-TAS}). Therefore, the effect of uncertainty
in QCF on vertical wind is the same as that for the radome because the same
calculated true airspeed is used for both.
\item \textbf{pitch (CPITCH\_GP):} As was the case for the cabin-mounted inertial systems,
there were two nearly identical inertial systems used in the wing pods, one
for the gust pod and the other for the LAMS, so it is again possible to compare the
measurements and obtain estimates of the random errors in their measurements. For both
units (LAMS and gust pod), the inertial systems use GPS measurements with a Kalman
filter to apply corrections, but they align independently and so have different errors
and Schuler oscillations. 
There were many flights in DEEPWAVE where one of these was not operational: 1--4, 6--7, 15, 17, 19. For the other flights, the standard deviation in the difference in pitch between these
two units was 0.06$^{\circ}$, so this is a reasonable estimate of the random error that
characterizes these measurements.\footnote{The standard deviation in the difference
between two variables is actually $sqrt(2)\times\delta$ where $\delta$ is the
standard deviation in each variable, so a better estimate is 0.04; this correction
has not been made throughout this document.}%
As for the radome, a true bias in this measurement (e.g., from misalignment at installation)
has been subsumed by the calibration of item 2 so does not enter in this item. However,
the remaining error on most flights has a slowly varying component (consistent with the
long time period of the Schuler oscillation) and so appears as a bias for any measurement
made over a period short compared to the Schuler oscillation, so it appears appropriate
to assign the observed standard deviation primarily to a bias because it will appear
steady in normal applications that look at vertical wind over periods short compared
to the Schuler oscillation. We have therefore partitioned the standard deviation into
estimated components of 0.04$^{\circ}$ bias and $0.02^{\circ}$ random error.
 The uncertainty in pitch
is the leading contributor to the standard uncertainty in vertical wind
and is also the leading contributor to the overall estimate of bias.

\item \textbf{Aircraft Vertical Velocity:} For the gust pod, the measurement
of vertical motion of the aircraft must be that from the IRU mounted in the
under-wing pod because the wing can flex and vibrate and the aircraft can roll
in ways that cause that vertical motion to differ from that sensed in the cabin.
Again, comparing the two units mounted in side-by-side wing pods provides the
best indication of the random component of uncertainty in this measurement,
because both units experience almost identical vertical motion. These two units
measure project-mean vertical aircraft motions that differ by 0.04\,m/s, with
standard deviation in that difference of 0.07\,m/s. It seems reasonable then to
estimate the random component of uncertainty as 0.07\,m/s, but the bias is more
uncertain. Good flights usually produced mean vertical aircraft motion from
takeoff to landing of less than 0.005\,m/s, so it is reasonable to neglect the
possible bias in this measurement (which is updated for stability in the IRU using
pressure altitude as a reference). 

\item \textbf{PSF:} The measured ambient pressure affects vertical wind
only through the dependence of true airspeed TASX on PSF, as described
in the document on \href{https://drive.google.com/file/d/0B1kIUH45ca5ATFV5d3QyQ0JpSjA/view?usp=sharing}{RAF processing algorithms},
Section 4.7.1. The effect is the same as for the radome, and is negligible;
see the discussion above for the radome system.
\item \textbf{ATX:} Temperature is needed to calculate TASX, but as for
PSF the effect is negligible. See the radome discussion above.
\end{enumerate}

The result of adding the elemental sources of uncertainty in quadrature
is a bias estimate of 0.18\,m\,s and a estimate of random uncertainty 
of 0.11\,m\,s$^{-1}$. The dominant contribution in both cases is that
from measured pitch, although the uncertainty in vertical motion of the
aircraft also makes a significant contribution to the random component
of uncertainty. It is important that these estimates only apply to cases
where the roll is within 5$^{\circ}$ of level.


\subsection{Elemental sources, horizontal wind\label{sub:hw-elements}}

\subsubsection{Radome-based system}

Table \ref{tab:Elemental-h} lists the elemental contributions to uncertainty in
the measurement of horizontal wind from the radome-based system. The following itemization
discusses each element.

\newcommand{\myparallel}{{\mkern3mu\vphantom{\perp}\vrule depth 0pt\mkern2mu\vrule depth 0pt\mkern3mu}}
\begin{table}
\begin{tabular}{cccccc}
\toprule 
\textbf{element} & \textbf{uncertainty source}  & \textbf{bias}  & \textbf{random}  & $\delta u_{\perp,\,\myparallel}$ \textbf{bias} & $\delta u_{\perp,\,\myparallel}$ \textbf{random }\tabularnewline
& & & & [m/s] & [m/s] \tabularnewline
\midrule
\midrule 
1 & BDIFR transducer  & 0.07~hPa  & 0.002~hPa  & -- & --\tabularnewline
\midrule 
2 & SSRD coefficients & 0.03$^{\circ}$ & 0.002$^{\circ}$ & (0.12,~--) & (0.01,~--)\tabularnewline
\midrule 
3 & ADIFR transducer  & 0.07~hPa  & 0.002~hPa  & -- & --\tabularnewline
\midrule 
4 & QCF transducer  & 0.34~hPa  & 0.01~hPa  & (see item 9) & \tabularnewline
\midrule 
5 & heading & 0.09$^{\circ}$ & 0.04$^{\circ}$ &(~0.38, --) & (0.17,~--)\tabularnewline
\midrule
6 & pitch  & 0.02$^{\circ}$ & 0.02$^{\circ}$  & -- & --\tabularnewline
\midrule 
7 & horiz. velocity of GV  & 0.03~m/s  & <0.03~m/s  & 0.03 & 0.03\tabularnewline
\midrule 
8 & PSF transducer  & 0.10\,hPa & 0.001\,hPa & -- & --\tabularnewline
\midrule 
9 & ATX & 0.3$^{\circ}$ & 0.1$^{\circ}$C  & (--,~0.16) & (--,~0.05)\tabularnewline
\midrule 
10 & $\delta q$ & 0.2\,hPa & 0.1\,hPa  & (--,~0.3) & (--,~0.15)\tabularnewline
\bottomrule
\end{tabular}

\protect\caption{Elemental contributions to the uncertainty in measurement of horizontal
wind by the radome-based system. Entries '--' indicate negligible contribution to uncertainty.
Entries with subscript $\perp$ refer to the lateral component of the horizontal wind, and
those with subscript indicating parallel refer to the longitudinal component (along the axis of the
aircraft).
\label{tab:Elemental-h}}
\end{table}

\begin{enumerate}
\item \textbf{BDIFR:} 
The primary uncertainty in BDIFR is assigned to bias because it is likely
a calibration uncertainty and the resolution and stability are much
smaller than this bias. However, a calibration bias in this measurement
does not affect the final wind measurement because the procedure in
Sect.~\ref{sub:radome-sensitivity} determines the sideslip angle
from flight data in a way that can be considered a calibration of
the measurement of sideslip, and a bias in BDIFR would be reflected
in a change in sensitivity coefficients determined in that section
that would compensate for that bias. Only random errors in BDIFR
propagate to the final measurement of horizontal wind, and the effect
of the listed random error is typically less than 0.0001 m/s in lateral
wind, with even smaller contribution to the longitudinal wind. These 
contributions therefore are listed as negligible in the table.
The next item considers the determination of sensitivity
coefficient for sideslip and is the dominant contribution to
uncertainty in horizontal wind arising from the measurement of sideslip.
\item \textbf{SSRD coefficients:} The calibration procedure of Sect.~\ref{sub:radome-sensitivity}
effectively removes the effects of possible biases in BDIFR and QCF
and instead replaces them with uncertainties arising from the coefficients
\{$e_{0},\, e_{1}$\} in (\ref{eq:betaFunctionForm}) and the ability of the selected
formula to represent the calibration data. The uncertainty in the first coefficient,
the main contributor to sideslip bias, is obtained from the standard deviation in the
mean of results from the circle analysis, summarized in Sect.~\ref{sub:HWsummary}.
For propagation to lateral horizontal wind, (\ref{eq:HWind}) indicates
that the result is approximately $\delta u_{lateral}=V\delta\beta$ where $V$
is true airspeed TASX, with additional contributions from correlated
errors involving $V$ that are small in comparison to that listed.
A typical value for V is about 220\,m\,s$^{-1}$, leading to the
listed elemental uncertainties in horizontal wind arising from uncertainty in
SSRD. 
\item \textbf{ADIFR: }See Sect.~\ref{sub:Airflow-angles} and Table~\ref{tab:Radome-system-measurements}.
The angle of attack has negligible effect on the horizontal
wind as long as the roll angle is small, so for measurements made
during straight-and-level flight this contribution to uncertainty
in horizontal wind is negligible.
\item \textbf{QCF:} The values listed are the characteristics of the transducer.
Application of the calibration procedure based on comparison to the
laser air motion sensor (\cite{CooperEtAl2014}) led to an estimated standard
uncertainty of 0.1~m/s for steady flight conditions and 0.3~m/s for
fluctuating conditions, so this is used for the table entry pertaining to
the wind component in the direction of the longitudinal axis of the aircraft. 
For the lateral component, QCF affects the calculated sideslip SSRD, but as
in the case of AKRD the procedure
to determine sensitivity coefficients removes any effect of bias in
QCF by calibration in terms of the coefficients \{$c_{0},\, c_{1}$\}
so a potential bias in QCF does not enter the lateral 
component of the horizontal wind but instead is replaced by possible bias in those
coefficients, as discussed for element 2. The effect of a random error
in QCF of 0.01~hPa for a typical value of QCF$\approx$100\,hPa is, 
from (\ref{eq:BetaFunctionForm}), to introduce
an uncertainty in sideslip of about 0.01\% or, because typical values of sideslip
are smaller in magnitude than 1$^{\circ}$, an error propagated to horizontal wind 
smaller than 0.001~m/s. This contribution is therefore neglected.
\item \textbf{HEADING:} The random error in heading can be evaluated by comparing
two duplicate IRUs, as was done for pitch. The two systems on the GV for DEEPWAVE
differed in mean heading by about 0.45$^{\circ}$, evidently a result of an alignment
error on installation. However, the standard deviation of the difference between the
two measurements was only 0.04$^{\circ}$, a value that indicates the systems may
perform better than the manufacturer's specification (0.2$^{\circ}$) would indicate.
The uncertainty in the bias evaluated from the circle-maneuver study of Sect.~XXX is about 0.09$^{\circ}$, so this will be used as the bias estimate while 0.04$^{\circ}$ is
considered the random component of uncertainty in heading.
\item \textbf{PITCH:} The contribution to uncertainty from the measurement of
pitch was discussed above in connection with measurement of the vertical
wind. However, in the case of horizontal wind, for level flight with negligible
roll an uncertainty in pitch makes negligible contribution to uncertainty in
either component of the horizontal wind.
\item \textbf{Horizontal Velocity Components of the Aircraft:} The 
measurement of horizontal wind is the sum of the relative wind and the
horizontal motion of the aircraft relative to the Earth, so uncertainty in
this component enters directly into uncertainty in the measured wind components.
\item \textbf{PSF:} The measured ambient pressure affects horizontal wind
only through the dependence of true airspeed TASX on PSF, as described
in the document on \href{https://drive.google.com/file/d/0B1kIUH45ca5ATFV5d3QyQ0JpSjA/view?usp=sharing}{RAF processing algorithms},
Section 4.7.1. Evaluation at typical values shows that the dependence
of the measured lateral component of the horizontal wind on uncertainty in this variable is negligible.
For example, TASX for PSF=300\,hPa, QCF=80\,hPa, and ATX=$-40^{\circ}$C
differs from that for PSF=300.1 by 0.03\,m\,s$^{-1}$ or about 0.01\%,
so this would also be the percentage change in the lateral component of the relative wind. 
\item \textbf{ATX:} Temperature is needed to calculate TASX, and other studies 
(\cite{CooperEtAl2014}) indicate that the temperature uncertainty is about 0.3$^{\circ}$C, and
this error will propagate to uncertainty in both components of the horizontal wind.
Typical values of Mach number for the DEEPWAVE project were 0.8, for which a temperature
change of $+0.3^{\circ}$C led to an increase in true airspeed of about 0.16\,m/s.
The temperature error is likely a bias, so this difference also should be treated as
a bias. The result is that the longitudinal component of the horizontal wind has an elemental
contribution from temperature of 0.16\,m/s, while the lateral component (being small and
having an error proportional to the error in TASX of about 0.16/240 or smaller than 0.1\%) 
has negligible error from this source. 
\item \textbf{PCOR:} The dynamic and static pressure measurements are corrected for the
static defect at the pressure ports using the formulas developed in (\cite{CooperEtAl2014}).
The uncertainty in the determination of the correction was estimated in that source as
less than 0.3\,m/s. Here we use similar estimates of 0.2 (bias) and 0.1\,hPa (random error).
correlated such that the error in static pressure is the negative of the error in dynamic
pressure. For DEEPWAVE research flights these errors
propagate to 0.3\,m/s bias and 0.15\,m/s random uncertainty.
\end{enumerate}

For the lateral component of the wind, adding the elemental contributions to uncertainty in quadrature leads to a net bias estimate of 0.4\,m/s and a random uncertainty of 0.2\,m/s. The measurement of heading makes a dominating contribution to each. For the longitudinal
component, the corresponding results are 0.3 and 0.2\,m/s. Here the dominant contribution
arises from the corrections applied to dynamic pressure to address the measured static
defect as determined from calibrations. Measurements of the longitudinal wind provided by
the LAMS have uncertainty of only about 0.1\,m/s, so when this instrument is available
the uncertainty could be reduced; the tabulated uncertainty includes an estimate of how
well the parameterized function used to correct pressure in the absence of LAMS actually
represents those measurements adequately. Some of the uncertainty entering this assessment
arises because the LAMS and the radome gust-sensing system measure at locations displaced
from each other and so may encounter slightly different wind conditions.



\subsubsection{Gust-pod system}


\begin{table}
\begin{tabular}{cccccc}
\toprule 
\textbf{element} & \textbf{uncertainty source}  & \textbf{bias}  & \textbf{random}  & $\delta u_{\perp,\,\myparallel}$ \textbf{bias} & $\delta u_{\perp,\,\myparallel}$ \textbf{random }\tabularnewline
& & & & [m/s] & [m/s] \tabularnewline
\midrule
\midrule 
1 & BDIFR transducer  & 0.07~hPa  & 0.002~hPa  & -- & --\tabularnewline
\midrule 
2 & SSRD coefficients & 0.03$^{\circ}$ & 0.3$^{\circ}$ & (0.12,~--) & (1.25,~--)\tabularnewline
\midrule 
3 & ADIFR transducer  & 0.07~hPa  & 0.002~hPa  & -- & --\tabularnewline
\midrule 
4 & QCF transducer  & 0.34~hPa  & 0.01~hPa  & (see item 9) & \tabularnewline
\midrule 
5 & heading & 0.17$^{\circ}$ & 0.3$^{\circ}$ &(~0.7, --) & (1.2,~--)\tabularnewline
\midrule
6 & pitch  & 0.02$^{\circ}$ & 0.02$^{\circ}$  & -- & --\tabularnewline
\midrule 
7 & horiz. velocity of GV  & 0.05~m/s  & 0.05~m/s  & 0.05 & 0.05\tabularnewline
\midrule 
8 & PSF transducer  & 0.10\,hPa & 0.001\,hPa & -- & --\tabularnewline
\midrule 
9 & ATX & 0.3$^{\circ}$ & 0.1$^{\circ}$C  & (--,~0.16) & (--,~0.05)\tabularnewline
\midrule 
10 & $\delta q$ & 0.2\,hPa & 0.1\,hPa  & (--,~0.3) & (--,~0.15)\tabularnewline
\bottomrule
\end{tabular}

\protect\caption{Elemental contributions to the uncertainty in measurement of horizontal
wind from the gust pod.. Entries '--' indicate negligible contribution to uncertainty.
Entries with subscript $\perp$ refer to the lateral component of the horizontal wind, and
those with subscript indicating parallel refer to the longitudinal component (along the axis of the
aircraft).
\label{tab:GP-Elemental-h}}
\end{table}

\begin{enumerate}
\item \textbf{BDIF\_GP:} 
The primary uncertainty in BDIF\_GP is assigned to bias because it is likely
a calibration uncertainty and the resolution and stability are much
smaller than this bias. However, a calibration bias in this measurement
does not affect the final wind measurement because the procedure in
Sect.~\ref{sub:radome-and-gust-pod-beta} determines the sideslip angle
from flight data in a way that can be considered a calibration of
the measurement of sideslip, and a bias in BDIF\_GP would be reflected
in a change in sensitivity coefficients determined in that section
that would compensate for that bias. Only random errors in BDIF\_GP
propagate to the final measurement of horizontal wind, and the effect
of the listed random error is typically less than 0.0001 m/s in lateral
wind, with even smaller contribution to the longitudinal wind. These 
contributions therefore are listed as negligible in the table.
The next item considers the more uncertain determination of sensitivity
coefficient for sideslip. 
\item \textbf{SS\_GP coefficients:} The calibration procedure of Sect.~\ref{sub:radome-and-gust-pod-beta}
effectively removes the effects of possible biases in BDIF\_GP and QC\_GP
and instead replaces them with uncertainties arising from the coefficients
\{$e_{0},\, e_{1}$\} in (\ref{eq:betaFunctionForm}) and the ability of the selected
formula to represent the calibration data. For sideslip, the fit procedure used values
of heading and ground speed components determined from the gust-pod IRU, but wind components
determined from the radome system. This allowed better determination of sensitivity
coefficients than would have been possible from "bootstrapping" gust-pod
measurements using repeated iterations, because the wind measurements from the
radome system have lower uncertainty than those from 
the gust-pod system. However, this means for example that the
offset in sideslip or heading will be dependent on the values from the radome system. Adjustment using the circle maneuvers of Sect.~\ref{WS-var-circles} is not
possible for the gust pod because the wind measurements are compromised at the high
bank angles required for the circle maneuver, so other adjustment is necessary. Values of SS\_GP are at least as uncertain as those from SSRD because the SS\_GP calibration uses wind
measurements determined from SSRD, so the values in Table~\ref{tab:Elemental-h}. 0.03 and
0.002$^{\circ}$, are lower limits for the uncertainty in calibration coefficients from
the gust pod. However, the standard deviation between
sideslip measured by the radome (SSRD) and that 
measured by the gust pod (SS\_GP) is typically about 0.3$^{\circ}$, an indication
that the uncertainty in SS\_GP may be much larger.\footnote{For comparison, the standard
deviation in the difference between AKRD and AK\_GP is only 0.09$^{\circ}$.} This large
standard deviation might arise partly from different
turbulent components being measured at the radome and at the gust pod, but this
seems unreasonably high for that explanation because the standard deviation corresponds
to a standard deviation in the difference in lateral wind at the two locations of 1.25\,m/s. 
Instead, it appears that there is some source of error affecting SS\_GP and that the
elemental uncertainty assigned to the random error arising from application of the SS\_GP
calibration must be increased to 0.3$^{\circ}$ until this discrepancy between SSRD and
SS\_GP can be resolved.
\item \textbf{ADIFR: }See Sect.~\ref{sub:Airflow-angles} and Table~\ref{tab:Radome-system-measurements}.
The angle of attack has negligible effect on the horizontal
wind as long as the roll angle is small, so for measurements made
during straight-and-level flight this contribution to uncertainty
in horizontal wind is negligible.
\item \textbf{QCF and QC\_GP:} The values listed are the characteristics of the transducers.
Application of the calibration procedure based on comparison to the
laser air motion sensor (\cite{CooperEtAl2014}) led to an estimated standard
uncertainty of 0.1~m/s for steady flight conditions and 0.3~m/s for
fluctuating conditions, so this is used for the table entry pertaining to
the wind component in the direction of the longitudinal axis of the aircraft. 
For the lateral component, QC\_GP affects the calculated sideslip SS\_GP, but the procedure
to determine sensitivity coefficients removes any effect of bias in
QC\_GP by calibration in terms of the coefficients \{$c_{0},\, c_{1}$\}
so a potential bias in QC\_GP does not enter the lateral 
component of the horizontal wind but instead is replaced by possible bias in those
coefficients, as discussed for element 2. The effect of a random error
in QCF of 0.01~hPa for a typical value of QCF$\approx$100\,hPa is, 
from (\ref{eq:BetaFunctionForm}), to introduce
an uncertainty in sideslip of about 0.01\% or, because typical values of sideslip
are smaller in magnitude than 1$^{\circ}$, an error propagated to horizontal wind 
smaller than 0.001~m/s. This contribution is therefore neglected.
\item \textbf{HEADING (CTHDG\_GP):} The random error in heading can be evaluated by comparing
the two duplicate IRUs for the gust pod and the LAMS, as was done for pitch. In DEEPWAVE, 
these two systems differed in mean heading by about 1.3$^{\circ}$, evidently a result of 
being installed at different angles relative to the aircraft longitudinal axis.
The standard deviation in the difference in heading measurements from the two systems,
after excluding some additional flights (18, 22, 25) 
where there appeared to be problems with the measurement, was 0.3$^{\circ}$, so
the uncertainty associated with this measurement is much higher than that for the
radome-based system. The mean difference between the two measurements of heading,
averaged over flights, had a standard deviation of 0.17$^{\circ}$, so this may
be a reasonable estimate of bias, as entered into the table.\footnote{Special processing,
using a hybrid heading obtained by complementary filtering (cf.~Sect.~\ref{sec:comp-filter}) 
with CTHDG\_GP considered the "fast"  signal and THDG the "slow"  signal, with appropriate
adjustment for the discontinuity at 360$^{\circ}$ and with exclusion of data during and
for 1 min after turns, reduced the standard deviation of the difference in heading to
less than 0.10$^{\circ}$ with a similar reduction in estimated bias. The resulting heading variable
retains the high-frequency response of the gust-pod IRU, needed to address issues like
wing flex or vibration of the pod, but used the higher-quality measurement of heading from
the fuselage IRU for long-term updating. This can improve the measurement of horizontal
wind from the gust pod significantly, at the expense of having the measurements use
a reference measurement outside the instrument. If improvement in the measurement of
horizontal wind from the gust pod becomes important, such processing can be
used for special cases.}
\item \textbf{PITCH:} The contribution to uncertainty from the measurement of
pitch was discussed above in connection with measurement of the vertical
wind. However, in the case of horizontal wind, for level flight with negligible
roll an uncertainty in pitch makes negligible contribution to uncertainty in
either component of the horizontal wind.
\item \textbf{Horizontal Velocity Components of the Aircraft:} The 
measurement of horizontal wind is the sum of the relative wind and the
horizontal motion of the aircraft relative to the Earth, so uncertainty in
this component enters directly into uncertainty in the measured wind components. Comparison
among the different measurements of velocity components of the aircraft (\{GGVEW, CVEW\_GP,
CVEW\_LAMS\} and \{GGVNS, CVNS\_GP, CVNS\_LAMS\}) indicate that, for DEEPWAVE flights
with good IRU operation (flights 5, 8--14, 16, 20--21, 23--24, 26) the standard
deviations among these measurements are consistent with an uncertainty of 0.05\,m/s.
This characterizes some combination of bias and random error, so to be conservative
this value has been assigned to each in the table.
\item \textbf{PSF:} This measurement has the same effect on the wind measurement from
the gust pod that it has on the measurement from the radome-based system because
the same true airspeed measurement is used for both. See the discussion for the radome
system that follows Table~\ref{tab:Elemental-h}.
\item \textbf{ATX:} Temperature affects wind measured by the gust pod in the same way as
that measured by the radome-based system. See the discussion for the radome
system that follows Table~\ref{tab:Elemental-h}.
\item \textbf{PCOR:} The same correction to true airspeed is applied to wind
measured by the gust pod as that applied to measurements from the radome-based
system. See the discussion for the radome
system that follows Table~\ref{tab:Elemental-h}.
\end{enumerate}

For the gust-pod system, the uncertainties in the two components of the horizontal wind 
(lateral and longitudinal relative to the aircraft) are quite different.
For the lateral component of the wind, adding the elemental contributions to uncertainty in quadrature leads to a net bias estimate of 0.7\,m/s and a random uncertainty of 1.7\,m/s. The measurement of heading makes a dominating contribution to each, and the values used for
these estimates are the result of intercomparisons between units and are much higher
than the best specifications for the unit listed in Table \ref{tab:Gust-pod-measurements},
but as noted there the error can increase fairly significantly if not updated with
frequent course changes. For the longitudinal
component, the corresponding results are 0.3 and 0.2\,m/s. Here the dominant contribution
arises from the corrections applied to dynamic pressure to address the measured static
defect as determined from calibrations, just as for the radome-based system, because the
measurements of the longitudinal component of the wind are the same for both systems.




\subsection{Summary}


This subsection summarizes the net uncertainty in wind measurements
as developed in the earlier parts of this section. See Table~\ref{tab:Summary-of-uncertainty} for the key results.

\begin{table}
\noindent \begin{centering}
\begin{minipage}[t]{1\columnwidth}%

\protect\caption{Summary of uncertainty for measurements of wind from the GV. The two
entries for bias for the horizontal wind are first the component lateral
to the axis of the aircraft and second the component parallel to the
axis of the aircraft.\label{tab:Summary-of-uncertainty}}
\noindent \begin{center}
\begin{tabular}{>{\centering}p{3cm}ccc>{\centering}p{2.5cm}}
\toprule 
\textbf{Measurement} & \textbf{bias} & \textbf{random uncertainty} & \textbf{net uncertainty} & \textbf{notes}\tabularnewline
\midrule
\midrule 
vertical wind, radome & 0.1 & 0.04 & 0.1 & lower with pitch correction\footnote{With application of the
pitch-correction algorithm of Sect.~\ref{sub:Schuler}, Eq.~\ref{eq:pitch-error-wo-approx}, the bias estimate is 0.05\,m\,s$^{-1}$ and the net uncertainty is 0.06\,m/s.}\tabularnewline
\midrule 
horizontal wind components, radome & 0.4 , 0.3 & 0.2 & 0.4 & roll < 5$^{\circ}$ \footnote{Expect minor degradation in turns.}\tabularnewline
\midrule 
vertical wind, gust pod & 0.18 & 0.11 & 0.2 & roll < 5$^{\circ}$\footnote{Errors may be much larger in turns.}\tabularnewline
\midrule 
horizontal wind components, gust pod & 0.7, 0.3 & 1.7, 0.2 & 1.8, 0.4 & best conditions\footnote{Selected flights in DEEPWAVE; can be factor-of-2 more uncertain for
worst flights. Must qualify heading measurement by comparison to another
measurement to get the listed performance. Not valid in turns.}\tabularnewline
\bottomrule
\end{tabular}
\par\end{center}%
\end{minipage}
\par\end{centering}

\end{table}



\subsubsection{The radome-based system. }

The standard wind measuring system on the GV is called the radome-based
system and results in the basic wind measurements WDC, WSC, and WIC
representing the horizontal wind direction (degrees relative to true
north), horizontal wind speed (m/s) and vertical wind speed (m/s).
For this system, the estimated bias limit, random component of standard
uncertainty, and combined standard uncertainty are listed in Table~\ref{tab:Summary-of-uncertainty}.
The combined uncertainty is obtained by adding the estimate of bias
and the estimate of random error in quadrature, but this characteristic
can be questioned because the bias estimate does not have normal statistical
characteristics. It is preferable to use the estimates of bias and
of the random component of standard uncertainty separately when characterizing
a measurement. As an approximation, it is reasonable to consider about
0.4\,m/s as the uncertainty in each component of the measurement
of horizontal wind and 0.1\,m/s as the corresponding uncertainty
in vertical wind. In any specific direction, the uncertainty in horizontal
wind remains about 0.4\,m/s, so this is also the uncertainty in measured
wind speed. Translation to uncertainty in wind direction depends on
the magnitude of the wind speed: If the wind speed is $u$ and the
uncertainty in the component of the wind transverse to the wind speed
is $\delta_{u}$, the uncertainty in wind direction $\delta\xi$ is
about $\delta_{u}$/u. For example, for $u=20$\,m/s and $\delta_{u}=0.4$\,m/s,
$\delta\xi=0.02$ rad. or about 1\,$^{\circ}$.

There is some potential for improvement in the vertical wind if the
pitch-correction algorithm of Section\inputencoding{latin1}{~}\inputencoding{latin9}\ref{sub:Schuler}
is applied. The improvement in effect removes the bias contribution from pitch, so
it reduces the estimated bias to 0.05\,m/s and the standard uncertainty to about 0.06\,m/s. Further reduction
would require independent evidence that the calibration maneuvers
are flown where the average wind is smaller than the 0.05\,m/s value
assumed when obtaining these results, because this then is the dominant
remaining uncertainty. 

In the case of horizontal wind,
the leading uncertainty is that associated with heading, which could
be improved by implementation of a full Kalman filter to adjust the
heading or by replacement of the IRU with a higher-quality system
with inherent Kalman filtering. There are systems available with much
lower specified uncertainty that could reduce the uncertainty in lateral
wind significantly.


\subsubsection{The gust-pod system}

The vertical wind measured by the gust pod is surprisingly good, when
it is considered that the measurements are made under the wing of
the aircraft in a region of seriously distorted airflow. While of
lesser quality than the measurements from the radome, the measurements
based on the gust pod have estimated undertainty only about twice
that of the radome-based measurement. On the other hand, the measurements
of horizontal wind from the gust pod have significantly greater uncertainty
than those from the radome. The uncertainty approaches 2\,m/s even
in the selected best cases, and there are examples where the discrepancy
between similar measurements of heading becomes much larger than the
tabulated values and the associated uncertainty in horizontal wind
becomes even larger. Vertical and horizontal winds are both problematic
in turns and should not be used for roll angles exceeding about 5$^{\circ}$
in magnitude. The problem with measurements in turns arises because
the gust-pod system is not aligned with the longitudinal axis of the
aircraft so, in turns, the three attitude angles (pitch, roll, heading)
become intermixed. It may be that the appropriate angle transformations
can be found to handle this problem, but current processing leads
to obvious errors in turns.

Some support for this value of uncertainty in vertical wind from the gust pod
was provided by Fig.~\ref{fig:vw-gp-vs-wic}, where the two measurements of
vertical wind were compared for all measurements from one flight. The standard
error in the difference between the two measurements was 0.27\,m/s, while the
uncertainties in Table~\ref{tab:Summary-of-uncertainty} would suggest an
expected uncertainty in the difference of 0.22\,m/s. While this is slightly
lower than the measured difference, some of that
difference can arise from real differences in vertical wind at the two
locations on the aircraft and from timing differences, so the measured
standard error is in
reasonable agreement with the expectations from the uncertainty analysis.

Using measurements of horizontal wind from the gust pod is not
recommended. That system was designed to provide a back-up measurement that
could be anti-iced to remain operational in heavy cloud. In the case
of vertical wind, it appears that the system fills this back-up roll
well. However, the horizontal wind from the gust pod is much inferior
to that from the radome-based system and probably should be used only
with much caution. That would involve checking that the measurement
of heading from the gust-pod IRU provides measurements in reasonable
agreement with other units, considering installation differences in
orientation) and excluding turns. An additional restriction arises
from the fit restrictions used to determine the coefficients in the
equation representing angle of attack. Those restrictions were: true
airspeed (TASF) greater than 130\,m/s, absolute value of roll less
than 5$^{\circ}$, and altitude greater than 5000~m. Outside these
limits, extrapolation errors can lead to significant errors in the
measurements from the gust pod.

The straightforward way to improve the measurements from the gust
pod would be to improve the measurement of heading. It might be possible
to calculate a surrogate heading from the fuselage IRU and the known
installation offset of the gust-pod IRU, but this hasn't been investigated
yet and would require continued study beyond that reported here.

\subsubsection{Conclusion}

The first three lines in Table \ref{tab:Summary-of-uncertainty} claim
that wind can be measured with low uncertainty from a high-speed aircraft
such as the NCAR/NSF GV. This is particularly challenging at high speed
because the aircraft introduces flow distortions and pressure variations
over and near the fuselage that affect many of the sensors used to measure
wind. Calibration by comparison to a laser air-motion sensor has led to
improvement in the measurement of horizontal wind and is the basis for
achieving these tolerances. Calibration maneuvers, especially those involving
flying circles, have provided evidence for the claimed limits to uncertainty
and have refined some of the calibrations used to achieve these limits.
The fourth line in the table indicates disappointing performance
for the gust-pod measurement of the component of horizontal wind lateral to
the aircraft, so in general this measurement should
not be used for research without further improvement. However, the measurement`
of vertical wind from the gust pod provides a useful
back-up to the conventional measurement.


% \end{document}




\section{Sensitivity coefficients\label{sec:Calibrations}}

This section reviews the determination of ''sensitivity coefficients''
that provide parameterized measurements of the angles of the relative
wind (angle of attack and sideslip angle) in terms of measured quantities
like pressure differences between ports on the radome. These sensitivity
coefficients are essential for measurement of the relative wind, as
described in Sect.~\prettyref{sub:The-Relative-Wind}, Eq. (\ref{eq:relative wind}).
DEEPWAVE Flight RF15 on 3 July 2014 was devoted to calibration maneuvers,
and measurements from that flight, combined with similar calibration
maneuvers flown on RF11 at 40,000 ft, are used to determine sensitivity
coefficients for angle of attack (AKRD and AK\_GP) and for sideslip
(SSRD and SS\_GP). A larger data set, described below, is also used
to study the representativeness and uncertainty of the resulting sensitivity
coefficients. This section also discusses some aspects of the relative
timing of the measurements. 

% For reference: The data files used were those produced in the field% during the DEEPWAVE project using nimbus code on the ground station% at that time. Files used were those from RF15, RF11, RF14, and RF16.% These data files have been transferred to EOL storage as /scr/rafdata/DEEPWAVE/DEEPWAVErfxx.nc% where xx is the flight number. For backup purposes, there is also% a zip file of these data files saved as DWCal.zip.





\subsection{Angle of Attack\label{sub:Calibration-AOA}}


\subsubsection{Equations underlying the calibration}

The first-order expression for the vertical wind $w$ is

\begin{equation}
w=V\sin(\alpha-\phi)+w_{p}\label{eq:VWind}
\end{equation}


where $V$ is the true airspeed, $\alpha$ the angle of attack, $\phi$
the pitch, and $w_{p}$ the vertical motion or rate-of-climb of the
aircraft. The solution for the angle-of-attack is

\begin{equation}
\alpha=\phi+\arcsin\frac{w-w_{p}}{V}\label{eq:SolvedForAOA}
\end{equation}


If it is reasonable to assume for some period of flight that $w$
is zero, or that it averages to zero, then

\begin{equation}
\alpha^{*}=\phi-\arcsin\frac{w_{p}}{V}\label{eq:alphaWithwZero}
\end{equation}


can be used as a reference angle-of-attack to which to fit a parameterized
formula. This fit reference depends on measurements of pitch, rate-of-climb,
and true airspeed. Even in the presence of waves, fitting functions
of the radome measurements and other flight characteristics to this
reference should average any real effects of vertical wind as long
as the vertical wind over the flight segments used averages to zero.

The danger in this approach is that a particular data set may not
have negligible average mean wind. For example, if a flight spent
more time in the updraft regions in the ascending portion upwind of
the island and less in the downdraft region downwind of the island,
the mean measurement of vertical wind may not be negligible. The functions
used for representation of angle of attack always include an offset
term along with functions of measurements, so it may be appropriate
to adjust that offset if there is evidence that the mean vertical
wind should not be zero. Other steps can be taken to check that offset
coefficient, as discussed in subsequent sections. One compromise,
followed below, is to determine any coefficients other than the offset
term from comprehensive data sets but then revised the constant coefficient
in the fit on the basis of special periods expected to average to
zero vertical wind, like flight over the ocean well away from weather
disturbances or special calibration flights in conditions with apparently
level air motions. 

In the case of the radome, the relevant variables are $\phi$=PITCH,
$w_{p}=$GGVSPD, and $V=$TASX. The radome measures the pressure
difference (ADIFR) between top and bottom ports on the radome, and
this pressure is usually normalized by some measure of dynamic pressure
like QCF or QCXC or QCRC. The former is preferable because the use
of corrected QCXC requires the application of static-defect corrections
that themselves depend on $\alpha$, leading to circularity in the
calculation, and QCR and QCRC are sometimes affected by icing or freezing
of accumulated water even when ADIFR continues to function.

For the gust pod, the relevant variables are $\phi$=CPITCH\_GP, $w_{p}$=CVSPD\_GP,
and $V$=TASX. The gust-pod measurements differ from those measured
relative to the fuselage; for example, the pitch of the gust pod is
several degrees different from that of the fuselage because of the
way in which the gust-pod IRU is installed. However, the true airspeed
$V$ in (\ref{eq:alphaWithwZero}) is measured better by the fuselage
system, so TASX will be used for $V$. The equation with the appropriate
variables is then: 
\begin{equation}
\alpha^{*}=\mathrm{CPITCH\_GP}-\arcsin\frac{\mathrm{CVSPD\_GP}}{\mathrm{TASX}}\label{eq:AOAeq}
\end{equation}


''Calibration'' of the angle-of-attack (i.e., determining the sensitivity
coefficients) then requires determining a function $f(\{x_{i}\})$
of measured quantities that matches $\alpha^{*}$ determined from
(\ref{eq:AOAeq}). Possible terms $\{x_{i}\}$ in that function may
include ADIFR and related measurements of pressure and dynamic pressure
as well as Mach number, and powers and products of these terms. For
the Rosemount 858 sensor used with the gust pod, it is expected from
theory that one element of $\{x_{i}\}$ will be ADIF\_GP/QC\_GP.
Wind tunnel and theoretical studies predict how the pressure will
vary on a hemispheric surface with changing angles, but those don't
necessarily apply to the mounting location on the GV because there
is considerable flow distortion at the under-wing location of the
pod and that affects the pressure response to changes in flow angles.
Therefore, the best approach is to use the above approach for that
sensor also and determine a functional response that matches the calibration
data.


\subsubsection{Application to the radome\label{sub:radome-sensitivity}}

The best method for calibrating angle-of-attack is through the use
of speed runs. In this maneuver, the aircraft is slowed to a speed
near the lower range of its operating range, then accelerated to near
the upper limit, and then slowed again to normal cruise. If this is
done while flying a level track, the angle of attack will vary through
its normal range and the pitch will vary similarly. If there is no
vertical wind or if a fluctuating vertical wind averages to zero,
(\ref{eq:alphaWithwZero}) then can be used to provide a reference
angle $\alpha^{*}$ that serves as reference for the parameterized
fit. There were three speed runs during RF15, at the times 3:21--3:29,
4:15--4:23, and 5:01--5:11 UTC. On RF11, there was a similar speed
run flown from 10:30--10:40. For the purpose of this first determination
of sensitivity coefficients for angle-of-attack, only those periods
were used. In addition, because some of the measurements at minimum
speed deviated from the otherwise simple fits, only measurements with
true airspeed in excess of 130 m/s were used; this eliminated some
of the slowest parts of the speed runs, but that is a flight speed
not used in normal operation.

In the code being used (cf.\ WindUncertainty.Rnw), the line 
\begin{lyxcode}
AOAREF~<-~PITCH~-~asin(GGVSPD/TASX)~/~Cradeg
\end{lyxcode}
represents Eq.~(\ref{eq:alphaWithwZero}).

The sensitivity to the pressure difference between vertically separated
ports is the most important part of the calibration of angle of attack.
Secondary terms are sometimes needed to adjust the value to maintain
a correct zero. Therefore, the fit was done in two stages. First,
the four speed runs alone were used to determine the sensitivity to
the pressure ratio, and then a larger dataset was used to incorporate
a wider range of flight conditions to check that the fit determined
from the speed runs remained representative of the larger data set..
The first fit was to the following simplified equation:

\begin{equation}
\alpha^{*}=c_{0}^{*}+c_{1}^{*}\frac{\Delta p_{\alpha}}{q}\label{eq:AOAsens1term}
\end{equation}
A fit to the data is shown in Fig.~\ref{fig:AOA-fit-and-plot} and
tabulated in the following summary of the fit, which was produced
by the ``R'' call at the top of the listing. AOAREFC is $\alpha^{*}$
with correction to pitch as in Section \ref{sub:Schuler} and AQR=$\Delta p_{\alpha}/q$
.



\begin{knitrout}\footnotesize
\definecolor{shadecolor}{rgb}{0.969, 0.969, 0.969}\color{fgcolor}\begin{kframe}
\begin{verbatim}
## lm(formula = AOAREFC ~ AQR, data = Data2)
## [1] "Coefficients:"
##             Estimate Std. Error t value Pr(>|t|)
## (Intercept)    4.394   0.005672   774.6        0
## AQR           20.986   0.068937   304.4        0
## [1] "Residual standard error: 0.121, dof=1977"
## [1] "R-squared 0.979"
\end{verbatim}
\end{kframe}\begin{figure}

{\centering \includegraphics[width=\maxwidth]{figure/WU-AOA-fit-and-plot-1} 

}

\caption{The angle-of-attack determined from the fit to pressure measurements from the radome, as a function of the reference angle provided by Equation (\ref{eq:AOAsens1term}), for the combination of data from all four speed runs as listed in the text.\label{fig:AOA-fit-and-plot}}
\end{figure}


\end{knitrout}

This fit gave coefficients $c_{0}^{*}$ and $c_{1}^{*}$ equal to 
4.394 and 20.986.
The fit accounted for 97.9\%
of the variance and had a residual standard error of 0.12$^{\circ}$,
so it represents the speed-run data well. In Fig.~\ref{fig:AOA-fit-and-plot},
the thin gray line under the thick orange dashed line represents the
uncertainty in the fit result and shows that the uncertainty in representing
these data with this set of coefficients is negligible. The more significant
uncertainty, however, comes from the assumption that the vertical
wind is zero for these periods of the speed runs. Therefore, additional
studies will be used below for further evaluation of the uncertainty
introduced by this assumption.

This calibration would be affected by a time difference between the
IRU measurement of pitch and the data-system sampling of the pressures
involved, especially that from the radome. The IRU outputs measurements
with a time delay that can be variable and unknown but is expected
to be <0.1~s, and normal processing uses an adjustment of 60~ms
to compensate for this delay. To guard against this delay affecting
the calibration, fits like that leading to the above formula
and coefficients were repeated after shifting the measurement of
pitch by various time intervals, both forward and backward. For shifts
within about 0.5\,s, the results did not change significantly, although
the fit with adjustment of pitch \emph{backward} by about 0.5\,s
had the smallest standard error . Equivalently, a similar standard
error was obtained if the measurement of the pressure ratio from the
radome ($\delta p_{\alpha}$) were moved forward in time by a similar
amount. Neither of these shifts seems likely at this magnitude, and
the change in standard error was only about 0.002$^{\circ}$, so we
interpret this as indicating insensitivity of the calibration to small
shifts in time. That is likely a result of the calibration data being
from speed runs where the flight speed is gradually increased and
then decreased, so any effect of a lag partially cancels in such data
segments.

The calibration would also be affected by pitch errors such as are
discussed in Sect.~\ref{sub:Schuler}. Therefore the correction procedure
discussed in that section was applied to these data before finding
the preceding fit. In comparison to the fit without this correction,
the standard error was changed only negligibly and the two fit coefficients
changed by -0.007 and -0.004, respectively, so this correction also
had only minor effect on the fit, with changes comparable to or smaller
than the standard errors in these coefficients. This insensitivity
perhaps arises because the period of the calibration spanned a few
hours and so included enough time for the Schuler oscillation of pitch
to average over the dataset used for calibration.





Previous studies of the radome where low-altitude flight segments
were included indicated that an additional term was needed in the
calibration to adjust the zero for vertical wind for flight in the
boundary layer over the ocean. The standard calibration determined
in that way is that given in the RAF document on \href{https://drive.google.com/open?id=0B1kIUH45ca5Ab2Z6cld1M1cydjA&authuser=0}{Processing Algorithms}:

\begin{equation}
\alpha=c_{0}^{\prime}+\frac{\Delta p_{\alpha}}{q}(c_{1}^{\prime}+c_{2}^{\prime}M)\label{eq:AOAstandard}
\end{equation}
with coefficients \{$c^{\prime}$\} = \{4.604, 18.67, 6.49\}. The
last term provides some adjustment dependent on Mach number and corrects
an offset often seen in vertical wind at low airspeed.



Figure~\ref{fig:plot-comparison-to-standard} shows the angle of
attack determined from this fit vs.\ that from the standard formula.
The plotted shapes show the distribution in values for centered intervals
in the predicted angle-of-attack of (1/3) degree, where the blue shapes
and orange line are the results from the fit to the speed-run measurements
and the green shapes are the distributions that would result from
using the standard calibration. The standard calibration is close
close to that determined from the speed runs, but the RMS difference
between predicted values and fit values increases from 0.12$^{\circ}$
for the speed-run fit to 0.15$^{\circ}$ for the standard fit. The
difference is most evident for measurements at large angle-of-attack,
where the standard coefficients produce increasingly higher values
as the angle-of-attack increases.

\begin{knitrout}\footnotesize
\definecolor{shadecolor}{rgb}{0.969, 0.969, 0.969}\color{fgcolor}\begin{figure}

{\centering \includegraphics[width=\maxwidth]{figure/WU-plot-comparison-to-standard-1} 

}

\caption[The angle of attack determined from the simple fit to the pressure ratio from the radome, as a function of the angle-of-attack predicted for zero vertical wind from the formula in the text (blue shapes), and the corresponding distribution that would result from using the standard calibration (green shapes)]{The angle of attack determined from the simple fit to the pressure ratio from the radome, as a function of the angle-of-attack predicted for zero vertical wind from the formula in the text (blue shapes), and the corresponding distribution that would result from using the standard calibration (green shapes). The shapes show the distributions for measurements in centered bins at (1/3)-deg increments in the predicted angle-of-attack. The dashed orange line is the best fit to the speed-run measurements.\label{fig:plot-comparison-to-standard}}
\end{figure}


\end{knitrout}

% For% comparison, the next figure (Fig.\ 2) shows the angle of attack determined% from this fit vs.\ that from the standard processing being used for% preliminary field processing and onboard display during DEEPWAVE.% There is a clear discrepancy, which arises from flight at between% 8,000 and 9,000 m. This also causes a significant offset in WIC at% these altitudes. This is simply an error in the calibration in use,% which traces back to much earlier projects and does not match the% algorithm documents or other recommendations for this calibration.% It will be important to use the new calibration for processing to% eliminate the offset arising from this old calibration.% % \begin{figure}% {\centering \includegraphics{figure/GP-AOA-vs-old-AOA}% % }% % \protect\caption[The angle of attack determined from the simple fit to the pressure% ratio from the radome, compared to the standard variable AKRD used% for preliminary processing]{The angle of attack determined from the simple fit to the pressure% ratio from the radome, compared to the standard variable AKRD used% for preliminary processing. The points plotted in red are those measured% at altitudes between 8 and 9 km GPS altitude; all others are plotted% as blue symbols.\label{fig:AOA-vs-old-AOA}}% \end{figure}

Because the standard fit in use for the GV include an additional terms
representing dependence on Mach number, the benefit of using such
a fit for DEEPWAVE is worth considering. The primary reason for that
fit has been that otherwise the measurements of vertical wind at low
level tend to be biased, but most of the useful flight data from DEEPWAVE
was at intermediate or high levels so this may not be a concern for
this project. 

\begin{knitrout}\footnotesize
\definecolor{shadecolor}{rgb}{0.969, 0.969, 0.969}\color{fgcolor}\begin{kframe}
\begin{verbatim}
## lm(formula = AOAREF ~ AQR + AQRM, data = Data2)
## [1] "Coefficients:"
##             Estimate Std. Error t value
## (Intercept)    4.387   0.005402  812.09
## AQR           17.924   0.216747   82.70
## AQRM           4.295   0.288585   14.88
##              Pr(>|t|)
## (Intercept) 0.000e+00
## AQR         0.000e+00
## AQRM        1.445e-47
## [1] "Residual standard error: 0.115, dof=1976"
## [1] "R-squared 0.981"
\end{verbatim}
\end{kframe}
\end{knitrout}

Fitting (\ref{eq:AOAstandard}) to the speed-run data gave a standard
error that is reduced by <0.006$^{\circ}$, so for those measurements
this small improvement does not seem to justify adding another term
to the fit. This will be revisited later in this section with an expanded
data set.



The next step was to expand the data set to include flights from a
range of altitudes including up to FL450, and to see if it is necessary
to refine the fit to include additional terms like that in Eq.~\ref{eq:AOAstandard}
to constrain the zero of the angle-of-attack while preserving the
measured sensitivity as in (\ref{eq:AOAsens1term}). This expanded
dataset included RF14, with a FL450 leg from 11:30 to 12:20 UTC; RF16,
with a FL430 leg from 9:30 to 11:00 UTC; RF03, all measurements above 35,000\,ft flight altitude (to incorporate a long flight where fuel burn-off changed the angle of attack); 
and an expanded section from
RF11 (in addition to the speed runs) from 7:00 to 10:00 UTC, to include
a long leg at FL400.

Various fit equations were explored involving terms including the
Mach number, the air density, the measured pressures, and various
products and powers of these terms including terms multiplied by the
basic pressure ratio already included in (\ref{eq:AOAsens1term}).
None of these produced enough improvement over the single-term fit
to warrant their inclusion; the best improvement in the residual standard
error was about 2\%.%
\footnote{Additional terms that did improve the fit significantly were those
correlated with the terms in Eq.~\ref{eq:alphaWithwZero}, esp.~pitch,
but including such terms is not consistent with finding a fit that
would represent the angle-of-attack in conditions with non-zero vertical
wind. For example, including \textquotedbl{}PITCH\textquotedbl{} as
a term in the fit resulted in a coefficient for this term of about
0.4, and such a large correlation between resulting measurements of
angle-of-attack and pitch would bias the response to a true vertical
wind. As an extreme example, inclusion of a term based on the right
side of Eq.~\ref{eq:alphaWithwZero} leads to a perfect fit, but
use of such a fit would ensure that all measurements of vertical wind
would be zero. For this reason, pitch and vertical aircraft motion
were excluded from the candidate terms in the fit.%
}



The results from the two-coefficient fit to the expanded data set, using\\
\begin{equation}
\alpha=c_{0}+c_{1}\frac{\Delta p_{\alpha}}{q}\,\,\,,\label{eq:AOArecommended}
\end{equation}
led to best-fit coefficients \{$c_{0},\, c_{1}$\} = \{4.468,
21.481\} and a standard error of 0.12
for 17,715 measurements. The small increase in standard error in comparison
to the speed-run measurements is expected because this expanded data
set includes regions more likely to have non-zero vertical wind, which
contributes to this error. Expanding this fit to include a third coefficient
as in (\ref{eq:AOAstandard}) resulted in negligible improvement (0.001$^{\circ}$)
in the standard error, and introduction of a set of eight possible
dependencies while exploring for better fits only led to reduction
in the the standard error by 0.003$^{\circ}$, so more complicated
equations than (\ref{eq:AOArecommended}) do not appear to be needed.%
\footnote{This simpler representation applies to the normal research flight
levels of the DEEPWAVE project, levels above about 10,000 ft. Because
expanded representations were needed in other projects to represent
low-level flight data, and low levels are not included in this expended
data set, it may be necessary to revisit this calibration with more
terms if lower-level flight segments are to be analyzed.%
} Another fit considered with this expanded data set was to set the
slope parameter to the value obtained from the speed-run data, 20.986,
and then fit using the expanded data set to determine a value of the
offset coefficient that minimized the mean vertical wind. That gave
a value for the first coefficient of 4.43
and a standard error negligibly different from that for the two-coefficient
fit, so that is another indication that the speed-run dataset and
the expanded dataset give consistent results. 

The recommended calibration for DEEPWAVE, determined with correction
of the pitch errors as discussed in Sect.~\ref{sub:Schuler}, is
(\ref{eq:AOArecommended}) with these values for the coefficients:
\{$c_{0},\, c_{1}$\} = \{4.468, 21.481\}.


Evaluation of all the points in the expanded dataset using three fits,
that from the expanded dataset, from the speed runs alone, 
and using the \textquotedbl{}standard\textquotedbl{}
fit discussed above, gave only very small differences in the results.
In Fig.~\ref{fig:second-violin-plot} the values of angle-of-attack
obtained using the fit to data from the expanded dataset are compared
to those obtained using the fit to the speed-run dataset. 
The angle-of-attack evaluated from the expanded-dataset coefficients
resulted in values 
0.036%
$\pm0.007%
^{\circ}$ larger than those from
the speed-run coefficients and 
0.037%
$\pm0.036^{\circ}$ larger
than those from the standard coefficients, so using any of these formulas
would give about the same results.  

% Still another determination of the
% sensitivity coefficients was based on the full set of measurements from
% all flights in the DEEPWAVE project; 
% that led to coefficients \{4.437, 21.193\} and a difference
% in angle of attack relative to the reference data set above of 
% round(mean(Data$A4-Data$A3, na.rm=TRUE),3)%
% $\pmround(sd(Data$A4-Data$A3, na.rm=TRUE),3)^{\circ}$.
% One measure of uncertainty in the results from applying
% these fits is that the various fits differ by about 0.03$^{\circ}$
% in resulting angle-of-attack. 

\begin{knitrout}\footnotesize
\definecolor{shadecolor}{rgb}{0.969, 0.969, 0.969}\color{fgcolor}\begin{figure}

{\centering \includegraphics[width=\maxwidth]{figure/WU-second-violin-plot-1} 

}

\caption[Distributions in the angle-of-attack determined from the two-coefficient fit to the expanded dataset, plotted with the distributions that would result from use of the equation based on the speed-runs only]{Distributions in the angle-of-attack determined from the two-coefficient fit to the expanded dataset, plotted with the distributions that would result from use of the equation based on the speed-runs only. The pink color denotes regions covered by both distributions.\label{fig:second-violin-plot}}
\end{figure}


\end{knitrout}





 % The best fit to the expanded dataset, keeping $c_1^{*}$ fixed as given in (\ref{eq:AOAsens1term}), was% the following equation:% % \begin{equation}% \alpha=c_{0}+\frac{\Delta p_{\alpha}}{q}(c_{1}^{*}+c_{1}M+c_{2}M^{2}+c_{3}M^{3})+c_{4}M\label{eq:BestFitAOAradome}% \end{equation}% with coefficients \{$c_{i}$\}=\{4.1079717, 82.6157340, -135.8737038,% 77.4028078, 0.5531502\} and $c_1^*$ =21.872829. The results are shown in the following figure, and a summary of the fit is listed also. The residual error in angle of attack was 0.11$^\circ$ for the 17,168 points used in the fit, and all terms in (\ref{eq:BestFitAOAradome}) tested as having high significance. (The variables listed as AQRMx are $\frac{\Delta p_{\alpha}}{q}M^x$ where $M$ is the Mach number.)

The standard errors in the coefficients for the expanded dataset are
respectively 0.0035 and 0.046,
so the coefficients are tightly constrained by the fit. 
The correlation between error terms was very high and positive, but the second
coefficient is applied to a term ($\delta p_{\alpha}/q$ that is typically negative with representative magnitude of about -0.0037$^{\circ}$ so the correlated
uncertainties partially cancel, leaving a net uncertainty in angle of attack
of 0.0007$^{\circ}$ (for 0.99 correlation between the uncertainty terms) as a result of uncertainty in the fit.

The expanded dataset provided results consistent with those from the
speed runs alone but contained 17,721 measurements vs.~only 2,019
for the speed-run dataset, so this larger set provides an opportunity
to examine the consistency of results from subsets of the measurements.
For this purpose, randomly selected but exclusive subsets of the data
were selected repeatedly and the fit coefficients were determined
from each of these subsets. In one example, the dataset was divided
randomly into 100 exclusive subsets and the fit coefficients were calculated for
each of those subsets, and then this process was repeated 50 times.
The resulting standard deviations in the fit coefficients were 0.070
and 0.88. With resampling, the independent samples entering these
averages are fewer than 5000 but more than 100, suggesting that the
means for the coefficients are known with uncertainties of between 0.007 -- 0.001 for the first coefficient and 0.09--0.012 for the second. The estimates from
the fit (0.004, 0.053) are midway in these ranges, so the fit results 
are consistent when the data are partitioned into various subsets. The fit
results thus appear reasonably characterized by the estimated standard errors from the fit. 

Repeating this for different subset sizes $N$ indicated
that the standard deviations in the coefficients scaled in a manner
consistent with $\sqrt{N}$ behavior, suggesting that the results
are not biased by isolated outlier measurements such as would occur
from regions of extended non-zero vertical wind. The subsetting also produced
resulting values for the individual coefficients that were highly
correlated, with correlation coefficient around 0.99. This also
supports the partial cancellation of uncertainty as evaluated for the fits.

These estimates of uncertainty characterize how well the sensitivity
coefficients characterize the data, but the validity of the result is
dependent on the mean vertical wind being zero, as discussed in connection 
with Eq.~(\ref{eq:SolvedForAOA}). 
A non-zero mean vertical wind of 0.1~m/s will introduce bias into the first 
sensitivity coefficient of magnitude 0.02$^{\circ}$, 
a value much larger than the estimated uncertainty in angle of attack
resulting from the fit. This then is the major uncertainty affecting angle
of attack and hence vertical wind, and it is difficult to quantify except
by estimates of how large the vertical wind might be. For all 1-s measurements
of vertical wind in the DEEPWAVE project, the mean vertical wind was 0.03\,m/s
without correction for pitch errors and <0.01\,m/s with such correction, in both
cases with a standard deviation of about 0.5\,m/s. When measurements are partitioned
into groups within 1-5$^{\circ}$ longitude upwind and downwind of the island crest, 
using a slanted dividing line approximately matching the orientation of the island,
the two groups had respective mean vertical wind measurements of -0.026 and +0.032\,m/s,
so even over the South Island of New Zealand there was no significant mean vertical
wind at the altitude of research flights. It therefore seems reasonable to use
a limit of about 0.05\,m/s as applicable to the calibration process, and so a bias
limit on the first sensitivity coefficient of magnitude 0.01$^{\circ}$. This is then
the dominant uncertainty component entering the measurement of angle of attack.

One additional test was used to check the consistency of the measurements and to ensure that the project-mean vertical wind would be near zero for research-flight conditions.  All flights
that appeared to provide good data for the calibration procedure were combined into
one dataset. The excluded flights were 6 and 7 (where there were problems with plugged
lines in the radome), and 15 (the calibration flight, mostly at lower than research flight
altitude). Also, the period from 9:50 to 10:30 UTC on flight 23 was excluded because the radome measurements looked suspicious and might have been affected by blockage. Pitch corrections as discussed in the preceding section were applied to these flights, and then the following tests were used to exclude periods not significant for the calibration: a) measurements where the true airspeed was less than 130 m/s, to exclude periods of anomalously high angle-of-attack, esp. during takeoff and landing when flaps and/or landing gear might be deployed; (b) periods when the roll angle was less than -5 or more than +5$^{\circ}$, to exclude turns; and (c) periods of flight below 35,000 ft pressure altitude, to emphasize the altitudes most used during research flights. More than 400,000 measurements were available for the fit after these exclusions.


\begin{knitrout}\footnotesize
\definecolor{shadecolor}{rgb}{0.969, 0.969, 0.969}\color{fgcolor}\begin{kframe}
\begin{verbatim}
## lm(formula = AOAREF ~ AR, data = DataC)
## [1] "Coefficients:"
##             Estimate Std. Error t value Pr(>|t|)
## (Intercept)    4.435   0.001014    4373        0
## AR            21.166   0.013268    1595        0
## [1] "Residual standard error: 0.115, dof=439552"
## [1] "R-squared 0.853"
\end{verbatim}
\end{kframe}
\end{knitrout}

The result was fit coefficients 4.435 and 21.166.\label{coefficients-aoa-all-flt} These coefficients are similar to those determined for the subset data used previously, but they could be used instead if the goal is to minimize the offsets in mean vertical wind for the project because, for the constraints used, they will force a project-mean vertical wind of 0. A test of these coefficients for the segment of flight 12 away from the influence of South Island indeed led to a mean wind very close to zero, and flight-by-flight averages also are reasonably small with these coefficients.




\subsubsection{Application to the gust pod\label{sub:gust-pod-aoa}}

A similar approach was taken for the gust pod, but with variables
translated to apply to the gust-pod instead of the radome: $\Delta P_{\alpha}=$ADIF\_GP,
$q=$QC\_GP, $p=$PS\_GP, and the Mach number $M$ was that determined
from the uncorrected measurements $p$ and $q$ from the gust pod.
Apparently because of the under-wing location in disturbed airflow,
a fit involving more terms was needed to represent the reference measurements
from Eq.~\ref{eq:AOAeq}. After exploration of various possible terms,
the fit selected to represent the gust-pod was the following:

\begin{equation}
\alpha=b_{0}+\frac{\Delta p_{\alpha}}{q}(b_{1}+b_{2}M)+b_{3}\frac{q}{p}\label{eq:AOAGfunction}
\end{equation}


\begin{knitrout}\footnotesize
\definecolor{shadecolor}{rgb}{0.969, 0.969, 0.969}\color{fgcolor}\begin{kframe}
\begin{verbatim}
## lm(formula = AOAREF_GP ~ AQR_GP + AQRM_GP + RR2_GP, data = Data2)
## [1] "Coefficients:"
##             Estimate Std. Error t value
## (Intercept)  -0.9033   0.007859 -114.94
## AQR_GP        3.6025   0.068942   52.25
## AQRM_GP       4.2860   0.137571   31.16
## RR2_GP        1.3299   0.033658   39.51
##               Pr(>|t|)
## (Intercept)  0.000e+00
## AQR_GP       0.000e+00
## AQRM_GP     1.128e-173
## RR2_GP      4.153e-252
## [1] "Residual standard error: 0.099, dof=1975"
## [1] "R-squared 0.989"
\end{verbatim}
\end{kframe}\begin{figure}

{\centering \includegraphics[width=\maxwidth]{figure/WU-AOA-gust-pod-1} 

}

\caption{For the gust pod, the angle-of-attack determined from the fit as a function of the reference angle provided by (\ref{eq:AOAeq}), for the combined four speed runs. Compare to Fig.~\ref{fig:AOA-fit-and-plot}, the corresponding plot for the radome.\label{fig:AOA-gust-pod}}
\end{figure}


\end{knitrout}



The results of this formula are compared to the reference data in
Fig.~\ref{fig:AOA-gust-pod}. The best-fit coefficients were \{$b_{i}$\}
= \{\ensuremath{-0.903}, 3.602,
4.286, 1.330\}
and the square of the correlation was 0.989
with residual scatter (residual standard error) of 0.099.
The fit was thus even better than that obtained for the radome for
these same speed runs.



As for the radome, an all-project fit was obtained for the gust pod. In this case, different flights were excluded: 2, 3, 4, 11, 15, 17, 18, 19, 23, 24, 25 and the flight period from 8:05--8:40UTC on flight 1. In most cases, this was because the CMIGITS IRU providing gust-pod measurements of pitch seemed to have larger than normal deviations that appeared suspicious, so it was thought preferable to fit without those periods of suspicious measurements. The fit summary is as follows:\label{aoa-coefficients-GP-all-flt}
\begin{knitrout}\footnotesize
\definecolor{shadecolor}{rgb}{0.969, 0.969, 0.969}\color{fgcolor}\begin{kframe}
\begin{verbatim}
## lm(formula = AOAREF_GP ~ AQR_GP + AQRM_GP + RR_GP, data = DataC)
## [1] "Coefficients:"
##             Estimate Std. Error t value Pr(>|t|)
## (Intercept)  -0.8083   0.002366  -341.7        0
## AQR_GP        3.2478   0.030399   106.8        0
## AQRM_GP       5.3507   0.049025   109.1        0
## RR_GP         1.2820   0.007171   178.8        0
## [1] "Residual standard error: 0.163, dof=315554"
## [1] "R-squared 0.861"
\end{verbatim}
\end{kframe}
\end{knitrout}
These coefficients \{\ensuremath{-0.808}, 3.248, 5.351, 1.282\} may be preferable to the coefficients obtained above if it is desirable to minimize the all-project mean vertical wind for the gust-pod measurements.

\subsection{Sideslip Angle\label{sub:calibration-SS}}


\subsubsection{Equations underlying the calibration}

Calibration of the sideslip angle is more difficult, both because
the equations are more complicated and because the maneuver is very
hard to fly. Ideally, the sideslip maneuver should only change yaw
angle and heading without change in roll, altitude, or angle-of-attack,
but that is impossible to fly. It is practical, however, to minimize
roll and change in altitude, and that was how these maneuvers were
flown. The three sets of yaw maneuvers were at these times: 3:32:00--3:35:30,
4:31:00--4:33:30, and 5:26:40--5:29:30 UTC.\footnote{Because the yaw maneuvers
on rf11 (10:25:00 to 10:30:00) were flown less well, they were
not combined with these new measurements, but it does not appear that
the sideslip calibration has any significant dependence on altitude.}%
In the case of yaw maneuvers, the calibration is based on the expectation
that the horizontal wind remains constant. The first-order equations
for the east and north components of the wind, $u$ and $v$, are:

\begin{eqnarray}
u & = & -U_{a}\sin(\Psi+\beta)+u_{p}\nonumber \\
v & = & -U_{a}\cos(\Psi+\beta)+v_{p}\label{eq:horizWindeqs}
\end{eqnarray}
where $U_{a}$ is the true airspeed, $\Psi$ the heading, $\beta$
the sideslip angle, and $u_{p}$ and $v_{p}$ are the eastward and
northward ground-speed components of the aircraft. These two equations
lead to the following reference formula for $\beta$:

\begin{equation}
\beta^{*}=-\Psi+\arctan\left(\frac{u_{p}-u}{v_{p}-v}\right)\label{eq:beta-equation}
\end{equation}
where the second term represents a correction for the change in direction
of motion of the aircraft, which is difficult to avoid in the yaw
maneuver. The measurements thus provide $\beta^{*}$, an estimate
of the sideslip during the yaw maneuvers.

There is, however, a circular component in (\ref{eq:beta-equation})
because it involves the wind components and those require $\beta$
for their measurement when sideslip changes. To reduce the feedback
from this term, the horizontal wind components $u$ and $v$ were
low-pass-filtered with periods ranging from 5--60 s and the filtered
values were used in (\ref{eq:beta-equation}). Filtering made small
differences in the fit coefficients but increased the residual error
significantly, but 60-s filtering was still selected because that
is a period long in comparison to the yaw maneuvers so it should reduce
possible bias in the fit coefficients from use of the older sensitivity
coefficients and any associated fluctuations in the wind measurements
during maneuvers.


\subsubsection{Application to the radome-based and gust-pod systems\label{sub:radome-and-gust-pod-beta}}

For both systems, a relatively simple fit was sufficient, in the following
form:

\begin{equation}
\beta=e_{0}+e_{1}\frac{\Delta p_{\beta}}{q}\label{eq:betaFunctionForm}
\end{equation}
where $\Delta p_{\beta}$ is the pressure difference between horizontally
separated pressure ports and $q$ the dynamic pressure. For the radome,
$q=$QCF and $\Delta p_{\beta}=$BDIFR; for the gust-pod, $q=$QC\_GP
and $\Delta p_{\beta}=$BDIF\_GP. The resulting fit for the radome
is listed below:

\begin{knitrout}\footnotesize
\definecolor{shadecolor}{rgb}{0.969, 0.969, 0.969}\color{fgcolor}\begin{kframe}
\begin{verbatim}
## lm(formula = SSREF ~ BQR, data = DataV)
## [1] "Coefficients:"
##             Estimate Std. Error t value
## (Intercept)  0.09187   0.005428   16.92
## BQR         22.30224   0.170344  130.92
##              Pr(>|t|)
## (Intercept) 7.164e-50
## BQR         0.000e+00
## [1] "Residual standard error: 0.114, dof=441"
## [1] "R-squared 0.975"
\end{verbatim}
\end{kframe}\begin{figure}

{\centering \includegraphics[width=\maxwidth]{figure/WU-sideslip-fits-1} 

}

\caption{The sideslip attack determined from the fit, as a function of the reference angle provided by Equation (\ref{eq:beta-equation}), for the combination of data from all three yaw maneuvers listed in the text. The gray area underlying the dashed orange line denotes the standard-uncertainty range for the fit.\label{fig:sideslip-fits}}
\end{figure}


\end{knitrout}

The best-fit coefficients were \{e\} = \{0.092,
22.302\}, the squared correlation was 0.97
and the residual standard error was 0.11,
as listed above. The plotted measurements and this fit are shown in
Fig.~\ref{fig:sideslip-fits}. 
The sideslip offset that is represented by this term will be determined
later by other means; cf. Section \prettyref{par:Offset-in-Sideslip}.)




The standard calibration in use for the GV has coefficient $e_{1}$=21.155, so this fit has
slope about 5\% greater than the standard values. These maneuvers were flown with special
care to minimize altitude and roll changes, so this value may be preferable to the older values.
The standard error of the fit, about 0.1$^{\circ}$, is likely much larger than the error in
representing the values of sideslip via \eqref{eq:betaFunctionForm} because much of the 
variability likely arises from real fluctuations in the horizontal wind, which would
contribute this magnitude error for wind fluctuations of only about 0.4\,m/s. The respective uncertainties in the fit coefficients were about 0.005 and 0.17$^{\circ}$, the former
leading to a possible bias error and the latter to an error of 0.17/22.3 or less than 1\%
in the value of the sideslip. Measurements of sideslip are seldom larger than 0.2$^{\circ}$,
so a fractional error of 1\% is not significant and will be neglected here. The bias in 
sideslip is intertwined with possible bias in heading, so further study of this offset
will be deferred to Section \ref{par:Offset-in-Sideslip} where it will be determined by a better method that separates it from the offset in heading.


% \begin{figure}% {\centering \includegraphics[width=300pt]{figure/GP-sideslip}% % }% % \protect\caption[The sideslip attack determined from the fit, as a function of the% reference angle provided by Equation (9), for the combination of data% from all three yaw maneuvers listed in the text]{The sideslip attack determined from the fit, as a function of the% reference angle provided by Equation (9), for the combination of data% from all three yaw maneuvers listed in the text.\label{fig:sideslip}}% \end{figure}

For the gust-pod, the same approach was followed, giving a fit with
characteristics as listed below:

\begin{knitrout}\footnotesize
\definecolor{shadecolor}{rgb}{0.969, 0.969, 0.969}\color{fgcolor}\begin{kframe}
\begin{verbatim}
## lm(formula = SSREF_GP ~ BQR_GP, data = DataV)
## [1] "Coefficients:"
##             Estimate Std. Error t value Pr(>|t|)
## (Intercept)   -3.621    0.02624  -138.0  0.0e+00
## BQR_GP        12.184    0.13287    91.7 2.5e-289
## [1] "Residual standard error: 0.164, dof=441"
## [1] "R-squared 0.950"
\end{verbatim}
\end{kframe}\begin{figure}

{\centering \includegraphics[width=\maxwidth]{figure/WU-sideslip-gust-pod-1} 

}

\caption[As in the preceding figure but for the gust-pod system]{As in the preceding figure but for the gust-pod system.\label{fig:sideslip-gust-pod}}
\end{figure}


\end{knitrout}

The squared correlation of this fit was 0.95
and the residual standard error was 0.164$^{\circ}$.
The best-fit coefficients were \{-3.621,
12.184\}. The difference in first coefficient
vs the radome is a result of the offset in heading between the aircraft
longitudinal axis and the gust pod. The detailed report for the fit
is listed above, and the result of applying these sensitivity coefficients
to the measurements from the gust pod is shown in Fig.~\ref{fig:sideslip-gust-pod}.

An additional study is needed to separate a possible offset in sideslip
from a similar error in heading, because the approach used here assumes that
the heading is accurate. In the data files used for this study, an offset in
heading of -0.08$^{\circ}$ was imposed; without that offset, the offset in
sideslip (coefficient $e_0$) would be reduced by 0.08.
A method of determining these separate offsets
is developed later in this report (Section \ref{sub:Analysis-of-circle}), 
where a set of circles flown with constant roll angle 
are used to obtain a better estimate of the sideslip offset. That study
results in a corrected value for $e_0$.\marginpar{XXX}


\subsection{True Airspeed from the Gust Pod\label{sub:GP-TAS}}

% \subsection*{The Components of the Relative Wind}% % The gust pod is a new wind-sensing system so it is useful to document how the measurements are handled. The probe itself has five pressure-sensing holes located on a hemispheric cap pointed into the wind. The pressure difference between the two ports located upward and downward from centerline, measured as ADIF\_GP, is used in the calculation of angle-of-attack. The similar pressure difference between the two ports located toward starboard and toward port of centerline, measured as BDIF\_GP, are used in the calculation of sideslip. In addition, the probe has static ports located on the side of the cylinder carrying the hemispheric cap, and the difference between the pressure at the forward port and that at the static ports is measured as the dynamic pressure QC\_GP. The static pressure is also measured, as PS\_GP.% % In addition, there is an inertial reference unit mounted with the gust pod to measure its attitude angles (roll, pitch, heading) and to measure the three components of motion of the aircraft relative to the Earth. This unit receives GPS information and uses it to update its measurements via a Kalman filter.

The measurements ADIF\_GP and BDIF\_GP, normalized by QC\_GP, are
used with the calibrations in the preceding section to find the angle-of-attack
and sideslip. In addition, to find the relative wind, a measurement
of true airspeed (TAS) is needed. In an effort to allow the gust-pod
system to operate independently of the standard radome-based system,
it is desirable to determine the true airspeed directly from the gust-pod
measurements. However, the standard system has been calibrated to
low uncertainty by reference to the Laser Air Motion Sensor, so for
the purpose of calibration that true airspeed (TASX) will be used
as a reference. Once the gust-pod measurement of true airspeed is
calibrated, the gust-pod system still measures wind without future
reference to the standard system, but the calibration process links
the two systems to have a common reference for TAS.

Therefore, the next step needed to use the gust pod wind-sensing system
is to determine a functional relationship between gust-pod measurements
and the true airspeed. The approach taken is to fit the basic pressure
ratio $q/p$ as measured by the 858 probe to match the corresponding
pressure ratio provided by the conventional measurements of dynamic
and static pressure. The relationship represents the true ratio of
dynamic to static pressure, so using that ratio the normal calculation
of true airspeed can be used to determine the TAS measured by the
gust pod. For this purpose, all the measurements from the calibration
flight (RF15) were used qualified only by requiring TASX>130 m/s.



The measurements entering the resulting fit are as follows: $q$=QCXC
is the reference dynamic pressure, $p=$PSXC is the reference ambient
pressure, $q_{g}=$QC\_GP is the gust-pod dynamic pressure, $p_{g}$=PS\_GP
is the gust-pod static pressure, $M$ is the Mach number determined
from the reference measurements, $M_{g}$ the Mach number determined
from the gust-pod static and dynamic pressure, and $\Delta P_{\alpha,g}$=ADIF\_GP
is the pressure difference between top and bottom pressure ports on
the gust-pod sensor. The following equation was found to provide a
useful representation of the reference ratio $q/p$ in terms of quantities
measured only by the gust-probe system:

\begin{equation}
\frac{q}{p}=b_{0}+b_{1}\frac{q_{g}}{p_{g}}+b_{2}M_{G}\frac{q_{g}}{p_{g}}+b{}_{3}\left(\frac{q_{g}}{p_{g}}\right)^{2}+b_{4}\frac{\Delta p_{\alpha,g}}{q_{g}}+b_{5}M_{G}+b_{6}\left(\frac{\Delta p_{\alpha,g}}{q_{g}}\right)^2+b_{7}\frac{\Delta p_{\alpha,g}}{q_{g}}M_{G}\label{eq:TASfit}
\end{equation}
This complexity in the equation was needed to obtain a good fit (with
residual error translating to a typical error in true airspeed of
about 0.5\,m/s for typical DEEPWAVE research legs), and all terms
were indicated to be significant in the fit. The fit details are listed
below:

\begin{knitrout}\footnotesize
\definecolor{shadecolor}{rgb}{0.969, 0.969, 0.969}\color{fgcolor}\begin{kframe}
\begin{verbatim}
## lm(formula = B1 ~ B2 + BxM + I(B2^2) + AQR_GP + MachG + I(AQR_GP^2) + 
##     I(MachG * AQR_GP), data = DataV)
## [1] "Coefficients:"
##                   Estimate Std. Error t value
## (Intercept)         1.1940   0.031653   37.72
## B2                 27.2893   0.844564   32.31
## BxM               -29.0483   0.965719  -30.08
## I(B2^2)            13.6231   0.411740   33.09
## AQR_GP             -0.5936   0.004224 -140.54
## MachG              -7.5812   0.226389  -33.49
## I(AQR_GP^2)         0.1554   0.002811   55.27
## I(MachG * AQR_GP)   1.2607   0.007205  174.97
##                     Pr(>|t|)
## (Intercept)       3.365e-305
## B2                1.114e-225
## BxM               3.094e-196
## I(B2^2)           2.311e-236
## AQR_GP             0.000e+00
## MachG             5.543e-242
## I(AQR_GP^2)        0.000e+00
## I(MachG * AQR_GP)  0.000e+00
## [1] "Residual standard error: 0.002, dof=34524"
## [1] "R-squared 0.998"
\end{verbatim}
\end{kframe}
\end{knitrout}
\begin{lyxcode}


\begin{knitrout}\footnotesize
\definecolor{shadecolor}{rgb}{0.969, 0.969, 0.969}\color{fgcolor}\begin{figure}

{\centering \includegraphics[width=\maxwidth]{figure/WU-plot-qp-fit-1} 

}

\caption{The estimate of $q/p$ obtained from the right side of (\ref{eq:TASfit}), determined entirely from measurements provided by the gust pod, plotted against the standard measurement obtained from QCXC/PSXC. The coefficients are listed in the text.\label{fig:plot-qp-fit}}
\end{figure}


\end{knitrout}
\end{lyxcode}
The resulting fit is shown in Fig.~\ref{fig:plot-qp-fit}. With this
result for $G^{*}$ from (\ref{eq:TASfit}), the true airspeed can then be obtained
using the usual formula, with $G^{*}$ used in place of $q/p$:

\begin{equation}
V=\sqrt{2\left(\frac{(c_{p}-R_{a}}{R_{a}}\right)\left(1+G^{*}\right)^{\frac{R_{a}}{c_{p}}-1}\left(\frac{c_{p}}{c_{v}}R_{a}T\right)}\label{eq:TASformula}
\end{equation}
where $c_{p}$, $c_{v}$, and $R_{d}$ are the specific heat of air
at constant pressure, the specific heat of air at constant volume,
and the gas constant for air and $p$ is the static pressure, $q$
the dynamic pressure, and $T$ the absolute temperature. 

The resulting true airspeed is compared to the conventional measurement
in Fig.~\ref{fig:TAS-from-the-gust-pod-2}. The fit provides a reasonable
representation of TASX, with a standard error of about 0.4\,m/s.
This is not much larger than the estimated uncertainty in TASX (about
0.3\,m/s), so wind calculations might be based on this value if necessary. However,
there is serious danger of over-fitting with this many coefficients and variables,
even though the fit was improved significantly with each addition; indeed, the standard error in the $q/p$ fit was reduced by 50\% with the addition of the last two variables in
the fit. Because of the danger that this variable might not perform as well when 
used outside the range of this fit (which was TASX > 130, |ROLL| < 5, and GGALT > 5000), 
it still should be better to base wind measurements from the gust pod on the 
conventional TASX combined with the relative-wind angles from the gust pod. However,
if it is desirable to have a wind measurement that is completely independent of the
standard radome-based system, using true airspeed measured with this fit might
provide a useful alternative.

\begin{knitrout}\footnotesize
\definecolor{shadecolor}{rgb}{0.969, 0.969, 0.969}\color{fgcolor}\begin{figure}

{\centering \includegraphics[width=\maxwidth]{figure/WU-TAS-from-the-gust-pod-2-1} 

}

\caption[True airspeed obtained from the gust pod, plotted against corresponding measurements from the conventional measurement system (TASX)]{True airspeed obtained from the gust pod, plotted against corresponding measurements from the conventional measurement system (TASX)\label{fig:TAS-from-the-gust-pod-2}}
\end{figure}


\end{knitrout}

In particular, for the DEEPWAVE flights where the standard vertical
wind measurement was compromised by an obstruction in the pressure-sensor
lines of the radome, the best vertical wind will be that calculated
using (\ref{eq:AOAGfunction}) for the angle of attack but conventional
TASX for the true airspeed, because the pressure ports used to determine
TASX were not obstructed.

%\vfill\eject

\subsection{Summary of sensitivity coefficients\label{sub:CalSummary}}


\subsubsection{Angle of Attack}

Radome: Eq.~(\ref{eq:AOArecommended}), coefficients $c_{0,1}$ =\{4.468,
21.481\}.\footnote{See also the alternative coefficients \{4.435, 21.166\} discussed on page \pageref{coefficients-aoa-all-flt}.}\\
Gust pod: Eq.~(\ref{eq:AOAGfunction}), coefficients\{$b_{i=0,3}$\}
= \{\ensuremath{-0.903}, 3.602,
4.286, 1.330\}\footnote{See also the alternative coefficients \{\ensuremath{-0.808}, 3.248, 5.351, 1.282\} discussed on page \pageref{aoa-coefficients-GP-all-flt}.}


\subsubsection{Sideslip Angle\label{CalSSsummary}}
\begin{hangparagraphs}
Radome and gust pod: Eq.~(\ref{eq:betaFunctionForm}), coefficients
\{$e_{0}$, $e_{1}$\} = \\
\qquad{} \{0.092, 22.302\}
for the radome,%
\footnote{However, see the revision of the first coefficient that follows in Section \ref{sec:HWind}. The value listed in the concluding summary of that section, 
in Section \ref{sub:HWsummary}, 
should supersede the value listed here.}%
\\
\qquad{} \{\ensuremath{-3.621}, 12.184\}
for the gust pod.
\end{hangparagraphs}


\subsubsection{True Airspeed, Gust Pod}
\begin{hangparagraphs}
Equation (\ref{eq:TASformula}), with (\ref{eq:TASfit}) and coefficients
\{$b_{i=0,7}$\} = \\
\qquad{} \{1.194, 27.2893,\inputencoding{latin1}{
}\inputencoding{latin9}\ensuremath{-29.0483}, 13.6231, \ensuremath{-0.5936},
\ensuremath{-7.5812}, 0.1554, 1.2607\}.\end{hangparagraphs}

% \end{document}



\section{Studies of the Vertical Wind\label{sec:vertical-wind}}

\subsection{Overview}

The preceding sections provided extensive information on how vertical
wind is measured and how the radome-based system is calibrated. Here,
three additional topics not covered well there are discussed to complement
those earlier discussions. The sections here deal with the choice and
quality of the variable characterizing the vertical motion of the 
aircraft, some issues related to the relative timing of the
measurements entering the calculation of vertical wind, and a proposed
method for using detection of the Schuler oscillation of the IRU to
correct the measurement of pitch.


\subsection{The vertical velocity of the aircraft\label{sec:VerticalVelocity}}


\subsubsection{Available measurements}

There are several independent measurements of the vertical motion
of the aircraft:

\noindent \begin{center}
\begin{tabular}{|c|c|}
\hline 
\textbf{Variable}  & \textbf{Source}\tabularnewline
\hline 
\hline 
VSPD  & Honeywell IRU\tabularnewline
\hline 
VSPD\_A  & Avionics system (Honeywell IRU)\tabularnewline
\hline 
GGVSPD  & GPS receiver, possibly with OmniSTAR corrections\tabularnewline
\hline 
CVSPD\_GP  & CMIGITS IRU, gust pod\tabularnewline
\hline 
CVSPD\_LAMS  & CMIGITS IRU, LAMS pod\tabularnewline
\hline 
\end{tabular}
\par\end{center}

The standard vertical wind calculation depends on a measurement of vertical
motion of the aircraft. Past projects have used one of VSPD (from the IRU), VSPD\_A, VSPD\_G
or GGVSPD (from GPS, the first two alternate names for the value
provided by the avionics package) or, 
much earlier, WP3 from a baro-inertial
update loop using VSPD. Baro-inertial updating is no longer used because
the Honeywell IRU that now provides VSPD already incorporates such updating.
For the systems based
on the gust pod or LAMS, additional vertical-velocity measurements
are provided by their pod-mounted IRUs, respectively CVSPD\_GP and
CVSPD\_LAMS. Changing roll of the aircraft can cause these measurements
to differ from the fuselage-based measurements.

Because VSPD as provided by the Honeywell IRU 
is controlled against the known instability of IRU
measurements in the vertical by updating to a reference value provided
by pressure altitude, it has some drawbacks. Pressure altitude
is not really a measure of altitude but of pressure, so the reference
altitude can be biased. Also, the IRU-imposed baro-inertial
loop has unknown response characteristics and uncertainty, and the
value provided by the IRU has inherent filtering. The
advantage of an IRU variable over a GPS measurement has been, until
recently, better response at high frequency at the expense of absolute
accuracy. GPS measurements have now improved, esp.~with OmniSTAR,
so it is worth considering what variable or combination of variables
should be used in the calculation of vertical wind. 

For horizontal
wind, the standard solution in use since the early 1990s has been
a complementary-filter solution (discussed in Sect.~\ref{sub:comp-filter},
where the difference between IRU and
GPS measurements is low-pass filtered and the result is added to the
IRU measurement. This preserves the high-frequency response of the
IRU measurement but causes the low-frequency components to match the
GPS measurements, thus providing both absolute accuracy and valid
high-frequency measurements. It seems worthwhile to explore a similar
approach for the vertical wind, to see if there is an advantage to
a variable that is provided by an analogous complementary filter.

Some of the following refers to a variable GGVSPDB, a highest-resolution
version of GGVSPD. GGVSPDB was used in this evaluation to ensure that
the reduced resolution in GGVSPD did not influence the variance spectra,
but GGVSPD is the usual variable used in wind calculation. It has
sufficient resolution for that purpose, and there appears to be no
advantage to using the higher-resolution variable.

In standard processing, two vertical-wind variables are produced from the radome-based
system, WI and WIC. These are called, respectively, ``Wind Vector,
Vertical Gust Component'' and ``GPS-Corrected Wind Vector, Vertical
Gust Component''. They are calculated by adding the relative wind
measured by the radome system to the vertical motion of the aircraft,
either VSPD (WI) or GGVSPD (WIC). The former is directly from the
Honeywell IRU; the latter is produced by the GPS receiver and may
benefit from OmniSTAR accuracy (flagged by GGQUAL equal to 5).

The names are not really appropriate and should be changed, but they appear
in many past datasets. Neither is the gust component;
both are the full vertical wind including relative wind and 
aircraft motion. Also, WIC
is not GPS-corrected, it is completely based on the GPS and does not
use the IRU at all except for the attitude angles need to determine
the relative wind (the same for both). These calculations employ the
nimbus function ``gust'', which implements the algorithm described
in Sect.~\ref{sub:General-comments}.

In addition, the Honeywell IRU provides a measurement of vertical acceleration.
In a special calculation, this was integrated to get velocity and
the result compared to VSPD. The results of the integration and the
variance spectrum of that result were 
quite similar to VSPD,
provided that a feedback loop was used to avoid exponentially growing
errors from positive feedback, so there is no advantage to using
that integrated acceleration in place of VSPD. (It was hoped that
some of the filtering imposed on VSPD could be avoided.)


\subsubsection{Variance spectra for components affecting the vertical wind\label{sub:Variance-spectra-for-W-components}}

For evaluation of these measurements of aircraft velocity, it is useful
to compare the variance spectra among them and to evaluate the contribution made to the vertical wind.
For this purpose, one flight from DEEPWAVE, RF16 (4 July 2014) was
used because it provided a good example of relatively intense and
prolonged vertical motion, with the core 6 h period of the flight
from 6:30:00 -- 12:30:00 having a standard deviation in vertical wind
of 1.0\,m/s. Much of the variation in vertical wind was from waves,
so the field was not fully developed turbulence, but the signals to
be resolved had enough intensity that noise floors on the measurements
were not a problem. 

\begin{figure}
\noindent \begin{centering}
\includegraphics[width=0.9\textwidth]{VSPDspectraDW16} 
\par\end{centering}

\protect\protect\caption{\label{fig:Variance-spectra-VSPD}Variance spectra {[}units: m$^{2}$s$^{-2}$,
density functions per logarithmic interval in frequency expressed
in Hz{]} for various components entering vertical-wind calculations.
Data are from DEEPWAVE flight 16, 6:30:00--12:30:00 UTC. The thick blue trace
is that for the vertical wind variable WIC; others show either measurements
of the aircraft vertical motion (VSPD, GGVSPD)
or, as GUSTW, the relative-wind contribution to WIC.
The green diagonal lines show the slope expected for an inertial
subrange; the thicker green line labeled ``!0$^{-4}$'' corresponds
to the spectrum expected for an eddy dissipation rate of $1\times10^{-4}$\,m$^{2}$s$^{-3}$.
Other green reference lines are displaced by an order of magnitude
in eddy dissipation rate. The calculations are based on the all-poles or maximum-entropy
method, with 100 poles; cf.~\citet{Press:1992:NRC:148286}. The resolution used
was 0.0005, with results smoothed in 50 bins
in the logarithm of frequency. Total variance is 1.36\,m$^2$s$^{-2}$.}
\end{figure}


Figure \ref{fig:Variance-spectra-VSPD} shows variance spectra calculated
for the entire 6-h period, for a number of measurements entering the
vertical-wind
calculations.\footnote{This plot is an exception to the goals of "reproducible
analysis" explained in the introduction and Appendix B. The calculations used
an external spectral-analysis program, not R code, and the specific data are not
part of the data archive for the document.}%
The thick blue trace shows the spectrum for the standard
vertical wind
measurement WIC, which has a pronounced broad peak near 10 km wavelength,
a relatively small inertial subrange extending only to about 2 km,
and (after smoothing in 50 logarithmic intervals across the range
of the plot) uncertainty estimates that range from about 5\% at the
lowest frequencies to <1\% at the highest frequencies.

The two measures of vertical aircraft motion, GGVSPD (from the GPS receiver)
and VSPD (from the inertial reference unit), have important differences in
their properties. At high frequency, VSPD (orange line) decreases rapidly
with increasing frequency, so it makes negligible contribution to the
vertical wind (WI) at frequencies above 1 Hz. In contrast, GGVSPD (red line) 
does not show a similar steep drop in variance until after about 2 Hz,
and it continues to make an important contribution to WIC over the 
frequency range from about 0.3 to 3\,Hz. The absence of spectral
variance above about 0.3\,Hz in VSPD appears to be the result of
internal filtering of this signal in the inertial unit, and indeed
some limited information on the properties of the signals does indicate
filtering at about this frequency. Because the aircraft motion clearly
has components in this frequency range (based on the feel of the
ride in turbulence),
it appears that WI should not be used for frequencies above about 0.3\,Hz.

The inertial-system variable VSPD is also subject to pressure damping,
and this may account for the difference between GGVSPD and VSPD at
long wavelength. The aircraft normally flies with reference to the
pressure altitude, so in a region of varying altitude at a given
pressure there are fluctuations in the aircraft geometric altitude (red
line) not reflected in the pressure altitude (orange line). Both contribute
in ways that result in essentially the same spectral variance for
frequencies below 0.3 Hz, as shown by the near coincidence of the
blue and cyan lines, so the difference is not significant. Here also,
though, the red line representing the GPS measurement is a better
measure of how the vertical wind should be affected, because the
pressure-damped orange line will not show real vertical motion of
the aircraft as it remains on a pressure surface.

For these reasons, WIC is the preferable variable to use for
vertical wind. This also answers the question posed earlier regarding
the possibility of using a combination of measurements from inertial and 
GPS systems to obtain better frequency response. The IRU measurements
turn out to have poorer frequency response and lead to apparent
biases for frequencies around 1\,Hz, so this is not a useful
combination. Instead, it appears best to use the GPS-provided
measurement of vertical aircraft speed directly without further modification.

There is still some reason for concern about the spectral response
of GGVSPD at frequencies around 0.5--1\,Hz, because there is no
evidence that the GPS is providing valid response at this high rate.
The unit used provides measurements at 5\,Hz and these measurements
are interpolated and filtered to higher frequency, so the cut-off
apparent in Fig.~\ref{fig:Variance-spectra-VSPD} is a result of
that sampling and may remove a real signal at higher frequency.
This topic will be reconsidered in the next section when variance
spectra for the horizontal wind are examined.

The black trace (labeled GUSTW) is the vertical component of the
relative wind, calculated as in Sect.~\ref{sub:The-Relative-Wind}. 
The sum of GUSTW and GGVSPD determines WIC, and the
alternate measure of vertical wind WI is the sum of GUSTW and VSPD from
the inertial reference unit. 
GUSTW has higher spectral variance than WIC for frequencies below
0.3\,Hz, so at these frequencies the aircraft motion tends to 
counter the relative wind and reduced the measured variance.
This would be the case, for example, if there were negligible
vertical wind and the relative wind arose entirely from the
vertical motion of the aircraft. On the other hand, for
frequencies around about 0.5\,Hz the spectral variance in the
vertical wind exceeds that in the relative wind 
so the aircraft responds approximately in phase with the vertical gusts.

The high-frequency variance spectrum has a slope differing 
a small amount from the expected -5/3 spectrum, and that is 
cause for some concern, but the turbulence in this region was not
very well developed or consistent so this is probably not
cause for alarm. Very good agreement with expectations has been
seen in cases of boundary-layer measurements where well-developed
turbulence at small scales is expected, but in those cases the
intensity of turbulence at small scales was greater also. This is
reason to continue to be suspicious of the response around
0.5--1\,Hz. 
%The gust-pod spectrum actually looks better, closer to -5/3 and
% lying a little lower in spectral variance.

Some of the relative magnitudes shown in Fig.~\ref{fig:Variance-spectra-VSPD}
may change for different intensities of turbulence. However, some
tentative conclusions seem indicated by this figure and the discussion
above: 
\begin{enumerate}
\item 
% \begin{figure}
% \noindent \begin{centering}
% \includegraphics[width=0.9\textwidth]{WIxpectra} 
% \par\end{centering}
% 
% \protect\protect\caption{\label{fig:WIWICcomparison}Variance spectra for vertical-wind variables
% WI and WIC, from DEEPWAVE flight 16, 4 July 2014, 6:30:00--12:30:00. }
% \end{figure}
% 


Exclude the variable WI from future data archives. VSPD,
on which WI is based, has problems at both high and low frequency.
Providing this variable could lead to the mistaken expectation that this
is the uncorrected version of WIC, while the two variables actually use
different and independent measurements of the vertical motion of the
aircraft. The argument for using a variable
based on VSPD has been that this formerly represented the high-frequency contribution
better than a GPS variable. That is not the case any longer, because
GPS measurements have improved greatly and the presently
available IRU-provided measurements
are obviously filtered at high frequency. 
% Fig.~XXX shows that, while
% the variance spectra of WI and WIC are reasonably similar, the WI
% spectrum is above that for WIC for frequencies from about 0.007--0.3
% Hz, with WI often 30\% higher, and then falls below the WIC spectrum
% above 0.3 Hz. The high-frequency decrease in WI is apparently attributable
% to the drop in VSPD at high frequency, thought to be from imposed
% filters in the IRU providing this variable. WIC is reasonably consistent
% with a -5/3 spectrum for frequencies above 0.1 Hz, while WI is not.
% The difference in magnitude between 0.01--0.1 Hz is not evident: The
% variance spectra from VSPD and GGVSPDB have the same magnitude, coherence
% near unity and phase near zero, through this range. This needs further
% investigation unless the variable WI is no longer produced. 

\item VSPD does not have better frequency response than GGVSPDB at high
frequency, as might be expected; the reverse is the case, possibly
because of filters imposed on VSPD in the inertial unit itself. This
argues for direct use of GGVSPDB in the vertical-wind calculation,
at least when OmniSTAR corrections are available. 
\item The relative-wind variance spectrum shows higher variance than
does that of the vertical wind,
indicating that the contribution from aircraft motion counters the
relative-wind contribution at all but the highest frequencies. The
variable GGVSPDB is close to 180$^{\circ}$ out-of-phase with the
relative wind (GUSTW in Fig.~\ref{fig:Variance-spectra-VSPD}) for
frequencies smaller than 0.2\,Hz. At low frequency, 
the aircraft is flown to maintain altitude and so to counter
any vertical wind, so these altitude adjustments and any other
adjustments made while maintaining altitude contribute to the
relative-wind spectrum but not to the net measurement of vertical wind.
%the motion of
%the aircraft probably produces more of the measured
%vertical component of the relative
%wind than does the real wind, while
At intermediate frequency (near 0.1 Hz) the aircraft %perhaps
responds to the vertical wind so as to move out-of-phase with it,
exaggerating the measured vertical component of the relative wind.
There is a transition in phase from being above 180$^{\circ}$ to
being below 180$^{\circ}$ as the frequency increases through 0.05\,Hz,
perhaps representing a transition from where the pilots or autopilot
cause motions that produce the relative-wind vertical component to
where the vertical wind causes the aircraft response. 
\item At high frequency (>0.3 Hz), the gust component dominates over the
aircraft response, the coherence between them falls to values between
0.1--0.5, and the phase relationship becomes variable before settling
near zero at 0.5 Hz. 
\end{enumerate}

\begin{figure}
\noindent \begin{centering}
\includegraphics[width=0.9\textwidth]{WIXHR1} 
\par\end{centering}

\protect\protect\caption{\label{fig:WIXHR1}Variance spectra for the vertical wind (here, WIX)
and for the contributions to it made by the relative wind (GUSTW)
and the aircraft motion (GGVSPDB). Also shown for comparison is the
spectrum for the IRU-provided aircraft motion (VSPD, dashed line).
Data from DEEPWAVE flight 15, 3:40:00--3:55:00.}
\end{figure}

An example where there was very low turbulence is shown in 
Fig.~\ref{fig:WIXHR1}. The intensity of vertical-wind fluctuations
was quite small for this flight segment, with a standard deviation
in vertical wind of only about 0.2 m/s. Some features of this plot
are explained below: 
\begin{enumerate}
\item The measured vertical wind at high frequency (>1\,Hz) is a white-noise
spectrum with intensity that can be matched by generating a random-noise
signal with peak amplitude of 0.15--0.2 m/s, which would lead to a
random error of about (0.15--0.2)/$\sqrt{12}\simeq0.05$~m/s. The
noise is dominated by the relative-wind contribution; there is essentially
no contribution at these frequencies from the motion of the aircraft.
The noise arises almost entirely from the angle-of-attack contribution
to the relative wind, and specifically from the measurements ``ADIFR''
and ``QCR'', which exhibit noise spectra for frequencies above 1~Hz.
In ADIFR, the noise is at a level that would arise from a white-noise
signal with peak amplitude of 0.4~hPa or a random error of about
0.1~hPa. Similar noise is present in both QCR and QCF, so it may
be linked to the nature of the transducers used or to some effect
originating in the pressure lines connecting the transducers to the
ports. The specifications for the pressure transducer that measures
ADIFR assert a standard uncertainty of about 0.03 hPa, and as sampled
and digitized by the data system the resolution is about 0.002 hPa
($\pm70\,$hPa for $2^{16}$ range digital encoding). Thus the white-noise
spectrum indicates a random error about three times greater than expected
from the transducer characteristics. This needs further investigation
because it imposes an important limit on capability to measure low-intensity
turbulence. 
% \item GGVSPDB as plotted here was obtained by interpolating linearly between
% 1-Hz measurements and then applying a low-pass filter with cutoff
% frequency 1 Hz. Simple linear interpolation leads to a pronounced
% peak at 1~Hz and peaks also at 2, 4, ... Hz, so some better interpolation
% procedure is useful. Smoothing with Savitzky-Golay polynomias with
% a 25-point span might be best%
% \footnote{A test of this with 4th-order Savitzky-Golay polynomials spanning
% 25 points produced a signal without the phase shift that arises with
% a Butterworth filter near the cutoff frequency.%
% }, but that can't be done in the conventional processing chain of nimbus
% which necessarily handles data second-by-second. {[}XXX check what
% is actually done for filtering and if a higher-rate is available for
% GGVSPDB.{]} VSPD (from the IRU), on the other hand, is present in
% the 25-Hz data files at 25 Hz, but it either has already been filtered
% in the IRU or simply reflects, with GGVSPDB, the absence of contributions
% from aircraft motion at these frequencies. It falls below even filtered
% GGVSPDB, though, so it probably doesn't represent any real signal
% at frequency above about 1 Hz. 
\item The relative-wind contribution (GUSTW) and the aircraft-motion contribution
(GGVSPDB) both have peaks at about 0.05\,Hz, with canceling contributions
so that no peak occurs at that frequency in the resulting vertical
wind. This peak arises from the autopilot controlling the aircraft,
which has a noticeable oscillation about the set altitude with a period
of about 15--20~s. A regular oscillation with this period is quite
evident in Fig.~\ref{fig:AltOscillationFMS}. 

\begin{figure}
\noindent \begin{centering}
\includegraphics[width=0.8\textwidth]{GGALTB} 
\par\end{centering}

\protect\protect\caption{\label{fig:AltOscillationFMS}The GPS-measured altitude [m] during a flight
segment in very smooth air. Data from DEEPWAVE flight 15, 3:40:00 to
3:45:00 UTC, variable GGALTB.}
\end{figure}

\item The disagreement between VSPD and GGVSPDB even at low frequency is
another indication that the IRU-provided value (VSPD) should not be
used, because the GPS-measured value at low frequency is surely a
better measurement than is possible from the IRU. (OmniSTAR corrections
were present throughout this flight.) 
\item The difference between WIC and GUSTW near 0.3~Hz, seen also as a
possible effect in Fig.~\ref{fig:Variance-spectra-VSPD}, indicates
that both the relative wind and the aircraft motion are making contributions
that add at these frequencies. The spectrum of vertical wind here
may be suspect because it is not clear if the amplitude of this contribution
from the GPS measurements should be trusted at these frequencies.
{[}XXX exploration of high-frequency GGVSPDB might be useful if such
are available; they were not present in the 25-Hz file I used.{]} 
\end{enumerate}

\subsection{Timing of measurements}

The different measurements entering the calculation
of vertical wind should be sampled at the same time. That is particularly
difficult in the case of samples from the inertial reference system
and GPS because they produce sample streams according to their own
timing and not in response to requests from the aircraft data system.
The variables involved in calculating the vertical wind are: 
\begin{itemize}
\item Angle of attack, from transducers attached to radome ports via lines
that can introduce small lags. No lag is currently used in processing. 
\item Pitch, from the IRU, transferred to the aircraft data system after
some delay that must be removed in processing. The standard processing
has a ``delay'' of -60~ms; i.e., the measurement is advanced in
time by 80 ms so as to apply to a time later than when it is received.
It is hard to understand how this is justified. 
\item Other attitude angles (heading and roll), which enter in minor ways
if the aircraft is not flying a straight-and-level course. The timing
of these can probably be neglected for calculations of vertical wind,
but the standard variables are also given a time lag of -60 ms. 
\item True airspeed, measured using the pitot-tube measurement of dynamic
pressure, used also with a measurement of temperature. While no lag
is assumed for dynamic pressure, the reference total temperature for
DEEPWAVE is RTHR1 (tentatively), and a time lag of -1~s is used for
this measurement. (No lag is assumed for RTRL, another candidate for
the reference temperature.) 
\item The vertical speed of the aircraft, taken for the preferred vertical
wind variable from GGVSPDB, the variable produced by the Novatel GPS
receiver employing OmniSTAR corrections when possible. This variable
is only sampled at a rate of 5 Hz, and no time lag is used in processing. 
\end{itemize}
It is useful to try to determine appropriate lags from the data and
from appropriate maneuvers. For example, in pitch maneuvers (in which
the pitch is alternately increased and decreased with typically a
10-s period) if the timing of measurements of pitch and angle-of-attack
are not matched or if the measurement of vertical speed of the aircraft
is not timed correctly there will be a residual measured vertical
wind, so these maneuvers are particularly stringent tests of relative
timing of the signals.

An approximate formula for the vertical wind $w$ is\\
 
\begin{equation}
w=V\sin(\alpha-\theta)+w_{p}\label{eq:Weq}
\end{equation}
where $V$ is true airspeed, $\alpha$ is angle-of-attack, $\theta$
is pitch and $w_{p}$ is the vertical velocity of the aircraft. This
equation can be used to adjust relative timing among the signals to
minimize the variance in vertical wind during the pitch maneuvers.
A good example is that from DEEPWAVE flight 15, 3:15--3:18 UTC. As
initially processed using standard processing at the time of DEEPWAVE,
the standard deviation in vertical wind through the pitch maneuvers
was 0.4~m/s while the variance in vertical motion of the aircraft
was 5.6~m/s. A long-standing criterion for acceptable pitch maneuvers
is for less than 10\% of the imposed velocity to enter the vertical
wind, so by this criterion the test was successful. However, a plot
of the vertical wind shows a clear match to the imposed velocity,
so it is worthwhile to see if better results are possible.

Speed runs have been used to determine the sensitivity coefficients
for determining $\alpha$ from the measured pressure differences on
the radome, so the sensitivity coefficients should not be adjusted
on the basis of the pitch maneuvers. The relative timing of the measurements
entering Eq.~\ref{eq:Weq}, however, can be adjusted to determine
if the residual vertical wind can be reduced. The sensors producing
$V$ and $\alpha$ are located close together and are processed in
the standard manner by the data acquisition system, so these can be
assumed to determine the reference time except for possible delays
introduced by pressure lines. However, the other variables
$\theta$ and $w_{p}$ are determined by independent systems and may
have timing offsets from the standard measurements, so these are the
variables whose lags are explored here.

The approach was to shift the measurements forward or backward in
time and, using Eq.~\ref{eq:Weq}, recalculate the vertical wind.
The best result obtained in this way was to shift PITCH forward 0.04~s
and shift GGVSPDB backward 0.04~s. Almost as good was to shift PITCH
forward 0.08~s and shift GGVSPDB backward 0.04~s. This reset the
assumed lag in PITCH to zero while imposing a small lag of -0.04~s
in GGVSPDB. For the purpose of this study, to avoid the phase lag
produced by filters at the high-frequency limit, Savitzky-Golay filtering
was used for GGVSPDB with 4th-order polynomials applied over 25-measurement
intervals to smooth the original measurements. The resulting standard
deviation in WIX was 0.22~m/s, vs 0.23 with no shift in GGVSPDB,
so it may be preferable to leave the assumed lag for GGVSPDB at zero
to avoid the awkwardness of advancing the measurement in time.

With the assumed time lag in PITCH removed, the resulting
measurement of vertical wind during the pitch maneuver is shown
in Fig.~\ref{fig:pitchManeuvers}. The standard deviation
in measured wind through the pitch maneuvers is only 4\% of the imposed
vertical motion of the aircraft. 
\begin{figure}
\noindent \begin{centering}
\includegraphics[width=0.8\textwidth]{PitchManeuvers} 
\par\end{centering}

\protect\protect\caption{\label{fig:pitchManeuvers}Vertical wind measured during pitch maneuvers.
The variable VSPDX (which is GGVSPDB interpolated, filtered, and shifted)
shows the vertical motion imposed on the aircraft, and WIX shows the
resulting measurement of vertical wind. }
\end{figure}


There is little signal corresponding to the imposed vertical
motion or variations in pitch, and the measured standard deviation
in vertical wind is about the same as that for measurements just before
and just after the pitch maneuvers. It thus appears that the measuring
system is able to remove the effects of the pitch maneuvers with essentially
undetectable residual.

\subsection{Correcting pitch for the Schuler oscillation\label{sub:Schuler}}


%include("chunks/latex-header")


% <<calculate-pitch-correction>>=
% read_chunk ("chunks/pitch-correction-function.R")
% <<correct-pitch>>=
r% 
%@

%\section{Schuler correction for pitch\label{sec:Schuler}}


\subsubsection{Basis for the correction}

%Before discussing additional aspects of the measurement system like
%determination of sensitivity coefficients for the radome, it is necessary
%to explain and document a correction that will be used in that determination
%and in subsequent studies. 
The primary weakness in the measurement
of vertical wind normally is the error in the
measurement of pitch, which is provided 
by inertial reference systems and therefore has the uncertainty and
fluctuations associated with those systems. An inertial system aligns
during initialization to detect the local vertical direction and then
calculates the new vertical direction as the aircraft moves (changing
the local vertical direction) and accelerates (which can cause gyros
to precess). Any misalignment present at initialization persists
but also can oscillate and can cause errors in roll and pitch to mix
as the aircraft changes flight direction. The standard uncertainty associated
with this measurement is 0.05$^{\circ}$ for flight duration of a
few hours, and the error often increases during the flight as heading
errors and accelerometer biases affect the results.

The work of Schuler (\cite{Schuler1923}) demonstrated that coupling among
some of these error sources leads to limits on the growth of errors
and to simultaneous oscillations in some of the measurement errors. In
particular, an error in pitch leads to an error in horizontal acceleration,
and integration of that error in horizontal acceleration leads to
a position error that compensates for the original error in pitch.
However, when the error in pitch is reduced to zero, errors in position
and velocity have been accumulated and those lead to growth of the
error in pitch in the direction opposite to the original error. The
result is a Schuler oscillation having a period of $T_{Sch}=(R_{e}/g)^{0.5}/(2\pi)\approx5064\thinspace s$
or 84.4\,min, where $R_{e}$ is the radius of the Earth and $g$
the acceleration of gravity. The existence of this coupling allows
estimation of the pitch error if the error in horizontal acceleration
is known. That is the case in DEEPWAVE because high-quality measurements
of velocity are available from GPS and those measurements can be compared
to the uncorrected measurements from the inertial systems to measure
the error in velocity and, from its derivative, the error in acceleration.
The specified uncertainty for the IRU, 0.05$^{\circ}$, contributes
an uncertainty in measured vertical wind of about 0.2 m/s, and 
Table~\ref{tab:Elemental-w} lists this as the dominant contribution to uncertainty
in that measurement. Therefore, improving the measurement
of pitch can lead to important
reduction in the uncertainty associated with measurement of vertical
wind.

For that reason, this section documents a procedure that can be used to
correct for a major part of the error in the measurement of pitch by
correcting for the Schuler oscillation as determined from errors
 in the ground-speed components of the aircraft as measured by the
IRU. This applies only to the Honeywell IRU used with the radome-based
wind system because the IRUs used with the gust pod or the LAMS
already incorporate a Kalman-filter correction that applies a similar
adjustment to the measurements. 




\subsubsection{Simple illustration of Schuler oscillation}

Although the following is standard material, it is presented here
to make the discussion of the Schuler oscillation self-contained and
for tutorial purposes. Those familiar with the Schuler oscillation
should skip this subsection.

Consider first the case of steady flight to the north. If, at some
starting point at rest, there is a pitch error $\delta\theta$, that
will cause gravity to be resolved into a NS component, as shown in
this diagram where a pitch error of $\delta\theta$ produces an erroneous
northward acceleration of $g\thinspace\sin(\delta\theta)\simeq g\delta\theta$:

\begin{knitrout}\footnotesize
\definecolor{shadecolor}{rgb}{0.969, 0.969, 0.969}\color{fgcolor}\begin{figure}

{\centering \includegraphics[width=\maxwidth]{figure/WU-plot-illustrating-pitch-coupling-1} 

}

\caption[Illustration of how a pitch error of magnitude ]{Illustration of how a pitch error of magnitude $\delta\Theta$ leads to a false measurement of horizontal acceleration of magnitude $g\delta\Theta$. The blue line represents the true level plane tangent to the surface of the Earth and $g$ the magnitude and direction of the gravitational acceleration.\label{fig:plot-illustrating-pitch-coupling}}
\end{figure}


\end{knitrout}

The result is that this erroneous acceleration produces a false northward
velocity and, integrated, a northward error in position. However,
the error then causes the integrated position to become too far north,
where the calculated direction toward the center of the earth becomes
biased in the opposite direction, and at some point the calculated
offset from being too far north equals the pitch bias, as shown in
the next figure:

\begin{knitrout}\footnotesize
\definecolor{shadecolor}{rgb}{0.969, 0.969, 0.969}\color{fgcolor}\begin{figure}

{\centering \includegraphics[width=\maxwidth]{figure/WU-plot-change-gravity-w-motion-1} 

}

\caption[Illustration of how a position error, denoted here as an error in latitude ]{Illustration of how a position error, denoted here as an error in latitude $\delta\lambda$, leads to an error in the direction of the vertical axis and so to a false horizontal acceleration.\label{fig:plot-change-gravity-w-motion}}
\end{figure}


\end{knitrout}

However, at the point where the error in latitude cancels the error
in pitch, there is an accumulated error in velocity, so the integrated
solution for position overshoots the equilibrium position by an amount
such that the error in pitch becomes the negative of the original
error, at which point the integrated velocity is zero but the new
error in pitch now causes a reversal of the position error. The result
is that the errors in pitch, velocity and position all oscillate with
a period equal to the Schuler period, $T_{s}\simeq5064$\,s, determined
from $T=(R_{e}/g)^{0.5}/(2\pi)$.

Here are the equations predicting how this oscillation will occur:

\begin{equation}
v_{n}=\frac{dx_{n}}{dt}=R_{e}\frac{d\lambda}{dt}\label{eq:vn}
\end{equation}


\begin{equation}
\frac{d^{2}x_{n}}{dt^{2}}=a_{n}\label{eq:an}
\end{equation}


where $x_{n}$ is the north coordinate of the position and $a_{n}$
is the northward acceleration. However, if $a_{n}=a_{n}^{*}+\delta a_{n}$
where $a_{n}^{*}$ is the true northward acceleration of the aircraft
and $\delta a_{n}$ is the erroneous acceleration that results from
pitch and displacement errors, then

\begin{equation}
\delta a_{n}=g(\delta\lambda-\delta\theta)\label{eq:delta-an}
\end{equation}
with $\lambda$ the latitude, $\delta\lambda=\delta x_{n}/R_{e}$
the error in latitude, and $\delta\theta$ the error in pitch. Then\\
\begin{equation}
\frac{d(\delta v_{n})}{dt}=-g\delta\theta\label{eq:delta-vn-dot}
\end{equation}
\begin{equation}
\frac{d(\delta\theta)}{dt}=\frac{\delta v_{n}}{R_{e}}=-\frac{1}{g}\frac{d^{2}(\delta v_{n})}{dt^{2}}\label{eq:delta-theta-dot}
\end{equation}
which has the solution:\\
\begin{equation}
\delta v_{n}=V_{n}\cos(\omega t+\zeta_{n})\label{eq:delta-vn-solution}
\end{equation}
where $\omega=\sqrt{g/R_{e}}\simeq0.00124$ is the Schuler angular
velocity. The error in the north component of the velocity therefore
oscillates with the Schuler period and a phase $\zeta_{n}$. Integrating
in time gives\\
\begin{equation}
\delta x_{n}=\int\delta v_{n}dt=\frac{V_{n}}{\omega}\sin(\omega t+\zeta_{n})\label{eq:delta-xn-solution}
\end{equation}


\textsf{\textcolor{black}{The key to developing a correction to the
pitch angle is that both $\delta x_{n}$ and $\delta v_{n}$ are observable
because reference measurements are available from GPS. For example,}}

\begin{eqnarray}
\delta x_{n} & = & R_{e}(\mathrm{LAT-GGLAT)}\nonumber \\
\delta v_{n} & = & \mathrm{VNS-GGVNS}\label{eq:observed-errors}
\end{eqnarray}
These observations can determine $A_{n}$ and $\zeta_{n}$ in (\ref{eq:delta-vn-solution}).
From these, the error in pitch can be found from (\ref{eq:delta-vn-dot}):\\
\begin{equation}
\delta\theta=-\frac{1}{g}\frac{d(V_{n}\cos(\omega t+\zeta_{n})}{dt}=\frac{V_{n}\omega}{g}\sin(\omega t+\zeta_{n})\label{eq:solution-delta-theta}
\end{equation}


Analogous equations lead to a similar coupling between the roll angle
and the east component of the ground velocity:

\begin{equation}
\delta\phi=\frac{V_{e}\omega}{g}\sin(\omega t+\zeta_{e})\label{eq:solution-delta-phi}
\end{equation}


where $\phi$ is the roll angle and $V_{e}$ and $\zeta_{e}$ are
determined from fits to the observed error $\delta v_{e}=\mathrm{VEW-GGVEW}$.

Thus the observed errors in the components of the aircraft ground speed
can be used to find corrections to be applied to the measurements
of pitch and roll. 


\subsubsection{Illustrative example}

An example from the second ferry flight of DEEPWAVE is shown in Fig.~\ref{fig:v-errors-straight-leg}.
The heading for most of this flight is close to southbound and steady,
so to a reasonable approximation the errors in pitch and heading will
be given by the respective derivatives in the error terms $\delta v_{n}$
and $\delta v_{e}$. 

\begin{knitrout}\footnotesize
\definecolor{shadecolor}{rgb}{0.969, 0.969, 0.969}\color{fgcolor}\begin{figure}

{\centering \includegraphics[width=\maxwidth]{figure/WU-v-errors-straight-leg-1} 

}

\caption[Errors in the NS and EW components of ground speed as determined by comparison to GPS, for DEEPWAVE flight ff02, a ferry flight starting on 1 June 2014 and traveling from Hawaii to Pago-Pago]{Errors in the NS and EW components of ground speed as determined by comparison to GPS, for DEEPWAVE flight ff02, a ferry flight starting on 1 June 2014 and traveling from Hawaii to Pago-Pago.\label{fig:v-errors-straight-leg}}
\end{figure}


\end{knitrout}

The errors can be determined directly from the time-derivatives of
the error terms in (\ref{eq:delta-vn-dot}) and the analogous equation
for roll, restated as:

\begin{eqnarray}
\delta\theta & = & -\frac{1}{g}\frac{d(\delta v_{n})}{dt}\label{eq:delta-theta}\\
\delta\phi & = & -\frac{1}{g}\frac{d(\delta v_{e})}{dt}\label{eq:delta-phi}
\end{eqnarray}


Therefore, rather than fitting variations like that shown in Fig.~\ref{fig:v-errors-straight-leg}
to sine or cosine functions, it is possible to obtain an estimate
of the time derivatives of the velocity-error terms over some shorter
interval and then use that derivative in (\ref{eq:delta-theta}) or
(\ref{eq:delta-phi}) to find the errors in pitch and roll. It is
important to note, though, that this gives errors in the local reference
frame with axes eastward, northward, and upward, here called the l-frame,
and these errors will then need to be transformed to the aircraft
reference frame to get errors in the measured quantities. The choice
made here is to estimate the derivatives using Savitzky-Golay polynomials,
as shown in Fig.~\ref{fig:Savitzky-Golay-estimates} . Savitzky-Golay
polynomials were chosen because they are computationally efficient,
don't introduce a time shift, and can provide derivatives directly.
A rather long averaging period of 1009 s, or about 1/5 of a Schuler
oscillation, was used to reduce noise in the result, and interpolation
filled some gaps in the measurements.

\begin{knitrout}\footnotesize
\definecolor{shadecolor}{rgb}{0.969, 0.969, 0.969}\color{fgcolor}\begin{figure}

{\centering \includegraphics[width=\maxwidth]{figure/WU-sg-poly-smoothing-1} 

}

\caption[Deduced error in pitch and roll angles for DEEPWAVE ferry flight ff02]{Deduced error in pitch and roll angles for DEEPWAVE ferry flight ff02.\label{fig:sg-poly-smoothing}}
\end{figure}


\end{knitrout}

The result is that the pitch error is limited to about 0.01$^{\circ}$
in magnitude for most of this flight, except for the final descent,
and the roll error is limited to less than about 0.015$^{\circ}$for
the same period. This is evidence for low uncertainty in the pitch
measurement for this flight, well below the specification of 0.05$^{\circ}$.
Section \ref{sub:Application-to-research} contains further discussion
of the errors from the research flights, where the estimated errors
can be larger. 


\subsubsection{Transformation of attitude angles}

In a reference frame called the $l$-frame or ENU frame, where the
coordinate axes are local-east, local-north, and upward, the preceding
subsection showed that the pitch and roll errors are related, via (\ref{eq:delta-theta})
and (\ref{eq:delta-phi}), to the time-derivatives of the errors in
horizontal velocity. Pitch and roll as used in these equations will
be the respective errors in platform alignment%
\footnote{The inertial system used is a strap-down system, so there is no actual
motion of the ``platform''. Instead, from measured rotations and
accelerations, the system calculates the expected orientation if there
were a true stabilized platform. The errors referenced here are those
relative to that calculated platform orientation.%
} in the north-south and east-west directions, so these angles must
be transformed to account for the orientation of the aircraft when
it is not flying straight-and-level to the north. Coordinates in the
body or $b$-frame of the aircraft are obtained from those in the
ENU or $l$-frame by applying three rotations to account for the heading,
pitch, and roll of the $b$-frame. This transformation leads to pitch
errors in the body frame of the aircraft (where measured pitch and
roll are measured and where the pitch measurement affects the calculated
vertical wind) that are mixtures of pitch and roll errors in the $l$-frame,
with the mixture dependent primarily on the heading. A positive pitch
error for northbound level flight will be a negative pitch error for
southbound level flight, and for eastbound flight an $l$-frame roll
error becomes a $b$-frame pitch error while an $l$-frame pitch error
become a negative $b$-frame roll error. 

Consider a unit vector $\mathbf{b}^{(l)}$ representing the orientation errors in pitch
and roll in the $l$-frame, with components \{$\sin\delta\phi,\,\sin\delta\theta,\,\sqrt{1-\sin^{2}\delta\phi-\sin^{2}\delta\theta}$\}
or, because the errors are always small, approximately \{$\delta\phi,\,\delta\theta,\,1$\}.
The three-angle transformation of this vector from the $l$-frame
to the $b$-frame is then represented by the following matrices, with
\{$\phi,\,\theta,\,\psi$\} denoting \{roll, pitch, heading\}:

\begin{equation}
R_{l}^{b}=\begin{bmatrix}\cos\psi\cos\phi+\sin\psi\sin\phi\sin\theta & -\sin\psi\cos\phi+\cos\psi\sin\phi\sin\theta & -\cos\theta\sin\phi\\
\sin\psi\cos\theta & \cos\psi\cos\theta & \sin\theta\\
\cos\psi\sin\phi-\sin\psi\sin\theta\cos\phi & -\sin\psi\sin\phi-\cos\psi\sin\theta\sin\phi & \cos\theta\cos\phi
\end{bmatrix}\label{eq:transform-l-to-b}
\end{equation}
 If the roll and pitch angles are small,

\begin{eqnarray}
\mathbf{b^{(b)}=}R_{l}^{b}\mathbf{b^{(l)}} & \approx & \left[\begin{array}{ccc}
\cos\psi & -\sin\psi & 0\\
\sin\psi & \cos\psi & 0\\
0 & 0 & 1
\end{array}\right]\begin{bmatrix}\delta\phi\\
\delta\theta\\
1
\end{bmatrix}=\begin{bmatrix}\cos\psi\delta\phi-\sin\psi\delta\theta\\
\sin\psi\delta\phi+\cos\psi\delta\theta\\
1
\end{bmatrix}\label{eq:b-vector-in-b-frame}
\end{eqnarray}


which leads to $\delta\theta^{(b)}$ and $\delta\phi^{(b)}$, the
pitch and roll errors in the $b$-frame:

\begin{eqnarray}
\delta\theta^{(b)} & \simeq & \sin\psi\delta\phi+\cos\psi\delta\theta\label{eq:final-answer}\\
\delta\phi^{(b)} & \approx & \cos\psi\delta\phi-\sin\psi\delta\theta\nonumber 
\end{eqnarray}
In turns, the roll angle is no longer negligible, so in general the
full transformation matrix \eqref{eq:transform-l-to-b} should be used.  Because this
matrix imposes rotations
by the measured pitch
and roll angles, the unit vector representing the errors in pitch and roll
after transformation will have components from which the true pitch and roll
can be determined, so subtracting the corrected pitch from the measured pitch
gives the error in measured pitch: \\
\begin{equation}
\delta\theta^{(b)}=\theta-\arctan\frac{b_{2}^{(b)}}{b_{3}^{(b)}}\label{eq:pitch-error-wo-approx}
\end{equation}
This pitch error should then be subtracted from the measured pitch
to obtain a corrected value of the pitch for use in the calculation
of vertical wind.


\subsubsection{Application to research flights\label{sub:Application-to-research}}

The research flights have frequent changes in heading, with mixing
of the roll and pitch errors but also accelerations that affect those
errors and introduction of new errors from heading errors. The corrections
to pitch therefore appear much less systematic than was the case for
the ferry flight. An example, DEEPWAVE flight 1, is presented here.
Figure \ref{fig:processing-1} shows the measured errors in ground-speed
components, and Fig.~\ref{fig:processing-2} shows the deduced pitch
and roll errors. There are instances where the pitch error abruptly
reverses sign; those are cases where the flight direction changes
by about 180 deg. During turns, the full transformation leads to a
result significantly different from the small-angle-approximation
result, as shown by the orange line in Fig.~\ref{fig:processing-2},
but when not turning the full-transformation results replicate the
small-angle-approximation results (blue line), as indicated by the
orange dashed line overlapping the blue line. The proposed solution
is to use the full transformation for processing a corrected pitch
variable, to be named ``PITCHC'', to be used for calculation of
the vertical wind. In straight-and-level flight, the needed corrections
are about $\pm$0.03$^{\circ}$ at some times, and this error can
lead (for true airspeed of 220 m/s) to an error in vertical wind of
about $\pm0.1$\,m/s. Correction for this error thus should lead
to a significant reduction in the uncertainty associated with the
measured vertical wind.

\begin{knitrout}\footnotesize
\definecolor{shadecolor}{rgb}{0.969, 0.969, 0.969}\color{fgcolor}\begin{figure}

{\centering \includegraphics[width=\maxwidth]{figure/WU-processing-1-1} 

}

\caption[Measured errors in ground-speed components for DEEPWAVE flight 1]{Measured errors in ground-speed components for DEEPWAVE flight 1. VEW and VNS are the east and north components of the ground speed measured by the inertial system, and GGVEW and GGVNS are the corresponding components measured independently by the GPS system.\label{fig:processing-1}}
\end{figure}


\end{knitrout}

\begin{knitrout}\footnotesize
\definecolor{shadecolor}{rgb}{0.969, 0.969, 0.969}\color{fgcolor}\begin{figure}

{\centering \includegraphics[width=\maxwidth]{figure/WU-processing-2-1} 

}

\caption{Errors in pitch and roll determined from the measured errors in ground-speed components, after transformation to the reference frame that is the body frame of the aircraft. The orange line labeled 'Full Xfm' uses (\ref{eq:b-vector-in-b-frame}) and (\ref{eq:pitch-error-wo-approx}), while the blue line uses the approximate result (\ref{eq:final-answer}). The limits $\pm$0.03 correspond to roll angle of $\pm 30^{\circ}$ after division by 1000, so the regions with vertical black lines are ones with significant roll.\label{fig:processing-2}}
\end{figure}


\end{knitrout}


\subsubsection{Tests of the correction}

Two tests were used to test if these pitch corrections made any significant
difference in the measurements of vertical wind. First, wind measurements
made before and after level course reversal were compared to see if
correcting the pitch reduces the difference in measurements on the two legs
before and after turns. Reduction would be expected because if there is
a pitch error it would reverse sign between the two legs, increasing their
absolute difference. Second, flight-average and project-average vertical
wind measurements were compiled without and with the pitch correction.

The following is a tabulation of five instances where the flight track
reversed course and remained at the same altitude. A number of other
candidates were excluded because conditions were too variable along
the legs to produce a small-uncertainty estimate of the vertical wind.
In each case, flight periods of about 5 min (sometimes
adjusted in times of strong wind to give similar-length segments flown
upwind and downwind) are listed before and after the turn, but excluding
the turn, to represent approximately overlapping flight segments where
it would be expected that the vertical wind would be the same. 

\begin{minipage}[t]{1\columnwidth}%
\hskip1in%
\begin{tabular}{|c|c|c|}
\hline 
\textbf{Flight} & \textbf{Times before turn} & \textbf{Times after turn}\tabularnewline
\hline 
\hline 
1 & 81000--81600 & 82400--83000\tabularnewline
\hline 
2 & 122500--123000 & 123700--124300\tabularnewline
\hline 
19 & 83930--84430 & 85130--85630\tabularnewline
\hline 
21 & 85100--85500 & 90300--90730\tabularnewline
\hline 
21 & 95630--100130 & 100800--101300\tabularnewline
\hline 
\end{tabular}%
\end{minipage}



%  The average absolute value of the
% difference before correction was smaller than 0.1\,m/s, so the wind 
% measurements were already in very good agreement for these pairs of legs
% and not much improvement could be expected. 
% However, application of the
% pitch-correction algorithm did reduce this average to about 36\% of the
% uncorrected value, as shown in Fig.\ \ref{fig:reverse-course-w-comparison}.

\begin{knitrout}\footnotesize
\definecolor{shadecolor}{rgb}{0.969, 0.969, 0.969}\color{fgcolor}\begin{figure}

{\centering \includegraphics[width=\maxwidth]{figure/WU-reverse-course-w-comparison-1} 

}

\caption[Absolute difference in vertical wind for flight segments before and after level course-reversal maneuvers]{Absolute difference in vertical wind for flight segments before and after level course-reversal maneuvers. The top panel shows the uncorrected measurements and the bottom panel shows the result of applying the pitch correction developed in this subsection.\label{fig:reverse-course-w-comparison}}
\end{figure}


\end{knitrout}

The difference between average vertical wind measurements for each 
pair of legs was calculated before and after applying the pitch-correction 
algorithm developed in this subsection.
The measurements were in good agreement without any pitch correction,
with an average absolute value of the difference between opposing legs of 
0.09\,m\,s$^{-1}$. The pitch correction kept the averages quite small and  
improved the agreement, reducing the mean value of the difference to 
0.03\,m\,s$^{-1}$. More significant that this reduction
is that the values were so small even before correction. This is
a good indicator of the low uncertainty of the pitch measurement even without correction. A course-reversal difference of 0.1\,m\,s$^{-1}$ would result from a pitch error 
of less than 0.015$^{\circ}$, so this suggests that the inertial system is
performing significantly better than its specified uncertainty (0.05$^{\circ}$).
After correction, the mean difference suggests an error of less
than 0.005$^{\circ}$ for the corrected values.

Table\ \ref{tab:vw-by-flight} 
lists the mean vertical wind with and without pitch 
correction for each of the DEEPWAVE flights. For each flight, all 
measurements above 25,000\,ft with true airspeed above 130\,m/s 
 and roll between $-5$ and $5^{\circ}$ were included to emphasize normal
research conditions. Any missing measurements were also excluded 
from the averages, and measurements from flights 6, 7, and 15 are not
included because the first two were cases where the conventional
vertical-wind measuring system malfunctioned and flight 15 was a
flight devoted to calibration with little upper-level flight and
frequent turns including circles.



\begin{center}
\begin{table}[H]
\begin{centering}
\begin{tabular}{|c|c|c|}
\toprule
\textbf{flight} & \textbf{mean WIC} & \textbf{mean corrected WIC}\tabularnewline
\midrule 
\midrule 
1 & \ensuremath{-0.01} & \ensuremath{-0.02}\tabularnewline
\midrule 
2 & 0.03 & 0.02\tabularnewline
\midrule 
3 & \ensuremath{-0.07} & \ensuremath{-0.07}\tabularnewline
\midrule 
4 & \ensuremath{-0.09} & \ensuremath{-0.14}\tabularnewline
\midrule 
5 & 0.03 & \ensuremath{-0.02}\tabularnewline
\midrule 
8 & 0.01 & \ensuremath{-0.03}\tabularnewline
\midrule 
9 & 0.02 & 0.01\tabularnewline
\midrule 
10 & 0.15 & 0.13\tabularnewline
\midrule 
11 & 0.08 & 0.08\tabularnewline
\midrule 
12 & 0.07 & 0.04\tabularnewline
\midrule 
13 & 0.02 & \ensuremath{-0.01}\tabularnewline
\midrule 
14 & 0.08 & 0.08\tabularnewline
\midrule 
16 & 0.07 & 0.05\tabularnewline
\midrule 
17 & 0.02 & 0.01\tabularnewline
\midrule 
18 & \ensuremath{-0.07} & \ensuremath{-0.07}\tabularnewline
\midrule 
19 & 0.08 & 0.08\tabularnewline
\midrule 
20 & 0.07 & 0.06\tabularnewline
\midrule 
21 & 0.11 & 0.08\tabularnewline
\midrule 
22 & 0.02 & 0\tabularnewline
\midrule 
23 & 0.07 & 0.07\tabularnewline
\midrule 
24 & 0.03 & 0.03\tabularnewline
\midrule 
25 & \ensuremath{-0.02} & \ensuremath{-0.02}\tabularnewline
\midrule 
26 & 0.03 & 0.03\tabularnewline
\bottomrule
\end{tabular}
\par\end{centering}
\protect\caption{The average vertical wind for each flight, before and after
application of the pitch-correction algorithm developed in this subsection.
The data restriction applied was that the true airspeed be above 130\,m/s,
absolute value of the roll less than $5^{\circ}$, and
the flight level above 25,000\,ft (to emphasize normal research measurement
conditions in the DEEPWAVE project). Flights 6, 7, and 15 are also missing, as
explained in the text.\label{tab:vw-by-flight}}
\end{table}

\par\end{center}


The mean value of the vertical wind, for all flights combined, was 
0.03\,m/s
for the uncorrected measurements and 0.02\,m/s for the corrected measurements, with standard deviations of 
0.06\,m/s, 
so both are well within expected tolerances.
The pitch correction has little effect on
these mean measurements or the overall standard deviation. 
However, the small offset obtained
with the pitch corrections applied does not indicate that the
measurements are only uncertain within these limits, because
most flights are long compared to the Schuler-oscillation period
of about 84 min. Flights will average over an oscillating
correction and that average may be small compared to the
correction applied. The standard deviation
of the applied correction is 0.02$^{\circ}$ when
calculated for the entire project.
That indicates that the correction
to vertical wind arising from application of the pitch-correction
algorithm introduces changes with standard deviation of about
0.09\,m\,s$^{-1}$ 
project-wide. Studies of individual flights 
show that this varies significantly from flight to flight.
This uncertainty, however, is a significant contributor to the
uncertainty in vertical wind. 
Without pitch correction, measurements of vertical wind 
will have an error with typical period of the Schuler
oscillation that, for measurements spanning much shorter periods,
will appear as a slowly varying bias.

The correction procedure developed here is not 
applied in normal processing of data files because
%suited to application during
%normal processing of data files because it uses a fit to the
it fits to the
entire sequence of ground-speed measurements to find the
corrections while the normal processor is sequential and
has no access to future measurements while processing. 
To apply these corrections, an additional processing
step is required. A program has been developed to add values of
pitch and vertical wind after correction, and this was used
for all the analyses in this report including the determinations
of sensitivity coefficients in Sect.~\ref{sec:Calibrations}.
% \end{document}
% \subsubsection{Unresolved Questions}
% \begin{enumerate}
% \item To obtain the result shown in Fig.~\ref{fig:pitchManeuvers}, it
% was necessary to use Savitzky-Golay polynomials for smoothing. It
% needs to be investigated if the situation is different from GGVSPDB
% measurements at 5 Hz interpolated and filtered for 25-Hz output, as
% would be the case for high-rate wind measurements. 
% \item It appears unresolved if GGVSPDB underestimates the contribution to
% the variance around 0.3 Hz. VSPD clearly makes no contribution here.
% The contributions from the two pod-mounted IRUs (with Kalman filtering
% to use GPS measurements) show slightly higher variance than GGVSPDB
% at these frequencies. \end{enumerate}
% 
% \vfill\eject

% \end{document}




\section{Studies of the Horizontal Wind\label{sec:HWind}}





\subsection{Analysis of circle maneuvers\label{sub:Analysis-of-circle}}

\subsubsection{Data Used}

\begin{knitrout}\footnotesize
\definecolor{shadecolor}{rgb}{0.969, 0.969, 0.969}\color{fgcolor}\begin{figure}

{\centering \includegraphics[width=\maxwidth]{figure/WU-circle-tracks-1} 
\includegraphics[width=\maxwidth]{figure/WU-circle-tracks-2} 

}

\caption[An example of circle flight pattern, from DEEPWAVE flight 15, 3]{An example of circle flight pattern, from DEEPWAVE flight 15, 3:38:30--3:54:30 UTC. Left side: normal flight track referenced to ground coordinates; right side, flight track plotted in a Lagrangian reference frame drifting with the horizontal wind.\label{fig:circle-tracks}}
\end{figure}


\end{knitrout}
During the DEEPWAVE project, several circle patterns were flown to help
characterize and check the wind measurements. In these maneuvers, a constant roll
angle was maintained so that the flight track drifted with the wind, and that
drift alone provides a measurement of mean horizontal wind that is dependent
only on the measurement of position from the GPS. The example used here is
from DEEPWAVE flight 15, 3:38:30--3:54:30 UTC, during which two circles were flown 
with left roll and then two with right roll, as shown in Fig.~\ref{fig:circle-tracks}.
% \noindent 
% \begin{figure}[btp]
% \begin{centering}
% \includegraphics[width=0.49\textwidth]{Circle1}\includegraphics[width=0.49\textwidth]{Circle1Drift}
% \par\end{centering}
% 
% \protect\caption{An example of a circle flight pattern, from DEEPWAVE flight 15, 3:38:30--3:54:30
% UTC. Left side: normal flight track; right side, flight track plotted
% in a reference frame drifting with the horizontal wind.\label{fig:CircleFlightPatterns}}
% \end{figure}

\subsubsection{Constraints arising from the assumption that the wind is steady}

The circle maneuver is a stringent test of the wind measurements because, in steady
conditions, the measured wind should remain constant around the circles. From the
patterns of deviations with orientation in the turn, it is possible to detect an
error in true airspeed or an offset in heading or sideslip. A time offset in
measured ground speed from the GPS relative to the IRU can be detected from
an apparent change in required heading offset that changes sign with the 
direction of the turn. In addition, for circles
flown with steady bank angle, the drift of the ground track as detected by GPS
measures the wind without any reference to the measuring system on the aircraft,
so this provides a standard against which the wind measurements can be checked.

For these patterns, the wind should be relatively steady and non-turbulent
and the roll angle should be constant. For this flight segment,
the left-turn circles had a roll angle of \ensuremath{-26.92}$\pm$0.17
and the right-turn circles had roll angle 27.08$\pm$0.12,
while the mean true airspeed for these circles was 153.5$\pm$0.5.
Most of the standard deviation in true airspeed arose from the normal
fluctuations created by the flight management system, as discussed
in Sect.~\ref{sub:Variance-spectra-for-W-components} (cf.~Fig.~\ref{fig:AltOscillationFMS}).
The steadiness of these measurements indicates that this maneuver was
flown with good precision and symmetrically, so the following is is a good illustration of
what can be learned from this maneuver.

\subsubsection{Basic equations}

The quantity used in all these tests is the relative wind in Earth-based
coordinates, with east and north components given approximately by \{$V\sin\psi^{\prime}$,
$V\cos\psi^{\prime}$\} 
where $\psi^{\prime}=\psi+\beta\cos\phi-\alpha\sin\phi$ with $\psi$
the heading, $\beta$ the sideslip angle, $\phi$ the roll angle,
and $\alpha$ the angle of attack. In the circle maneuver, $\phi\approx27^{\circ}$
so the last two terms in the expression for $\psi^{\prime}$ do not simplify
with the small-angle approximation. If the corresponding components
of the horizontal wind are $v_{x}$ and $v_{y}$, the components of
the ground speed of the aircraft are

\begin{eqnarray}
\label{eq:dead-reckoning}
\begin{split}
v_{p,x} & = & V\sin\psi^{\prime}-v_{x}\\
v_{p,y} & = & V\cos\psi^{\prime}-v_{y}
\end{split}
\end{eqnarray}
and the difference between the motion of the aircraft expressed as \eqref{eq:dead-reckoning} and
the ground-speed components measured by GPS ($v_{g,x},\, v_{g,y}$)
is\\
\begin{eqnarray}
\label{eq:v-errors}
\begin{split}
\delta v_{x} & = & V\sin\psi^{\prime}-v_{x}-v_{g,x}\\
\delta v_{y} & = & V\cos\psi^{\prime}-v_{y}-v_{g,y}
\end{split}~~~~~.
\end{eqnarray}


If error terms for true airspeed ($\delta V$) and for the adjusted heading angle ($\delta\psi^{\prime}$) are
introduced so that $V=V_{m}+\delta V$ and $\psi^{\prime}=\psi_{m}^{\prime}+\delta\psi^{\prime}$ 
where subscript $m$ refers to the measured quantity, and if it is assumed that the wind components \{$v_{x}$ and $v_{y}$\}
are steady around the circles, then estimates for the four fit parameters
\{$\delta V,\,\delta\psi^{\prime},\, v_{x},\, v_{y}$\} can be found
by minimizing the errors given by \eqref{eq:v-errors}. 

Once the average wind direction ($\lambda$) and wind speed ($v$) have been determined either
by the above fit or from the mean of measurements around the circles,\footnote{With equal weighting for measurements with all orientations
relative to the wind, averaging measurements should give correct values even if
there are errors in the individual terms}  the error $\delta v_{m}$ in the measurement
of wind speed ($v_{m}$) can be expressed as\\
\begin{equation}
\delta v_{m}=v_{m}-v=-\delta V\cos\xi-V\delta\xi\sin\xi\label{eq:delta-v-m}
\end{equation}
where $\xi=\psi^{\prime}-\lambda$ is the angle between the relative
wind (in the direction $\psi^{\prime}$ which is the heading adjusted for the roll angle by components from the sideslip and angle of attack) and the direction x of the wind relative to the Earth ($\lambda$). 

This equation is justified as follows. Because the relative wind transformed to an Earth reference frame is added to the ground-speed vector to find the wind, an error $\delta V$ in true airspeed leads to an error of the same magnitude in the component of measured wind along the longitudinal axis of the aircraft. The measured wind therefore changes by 
$-2\delta V$ when the aircraft changes from an upwind to a downwind flight direction, with the negative sign arising from the convention that wind direction is specified as the direction from which the wind blows. The error in true airspeed thus can be determined from the difference between wind measured while flying upwind and that measured while flying downwind. For other flight directions, this error projects to the wind direction as $-\delta V\cos(\psi^{\prime})$. Similarly, when flying crosswind the wind measurement is determined by the sum of the ground-speed component along the wind direction and the component of the relative wind lateral to the aircraft. If the wind is from the port side of the aircraft (e.g., for an east flight direction with wind from the north), a positive heading error leads to a relative wind component opposing the wind and so to a measurement error of $-V\delta\psi^{\prime}$. For wind from the starboard side of the aircraft, the sign reverses, and for other angles the error contribution is $-V\delta\psi^{\prime}\sin(\psi^{\prime})$. 

Fitting to minimize the deviations expressed by \eqref{eq:delta-v-m} can then give estimates of the three fit parameters $\delta V$, $\delta\psi^{\prime}$, and $v$, or a fixed value can be used for $v$ as given by the mean of measurements or by a previous fit to \eqref{eq:v-errors}. These fits should give consistent results, but a fit to \eqref{eq:delta-v-m} is particularly illustrative because plots of the error as a function of flight direction relative to the wind clearly reveal the magnitude and source of the deviations. The expected pattern is shown in Fig.~\ref{fig:sine-plot}.
\begin{knitrout}\footnotesize
\definecolor{shadecolor}{rgb}{0.969, 0.969, 0.969}\color{fgcolor}\begin{figure}

{\centering \includegraphics[width=\maxwidth]{figure/WU-sine-plot-1} 

}

\caption{Illustration of the expected variation in measured wind speed with flight angle relative to the wind direction, for assumed errors of $\delta V=1$ m/s in true airspeed and $\delta\psi^{\prime}=0.3^{\circ}$ in heading and for an assumed true wind speed of 10 m/s. The assumed true airspeed is 155 m/s. The red arrows show the differences in measured wind speed between 90 and 270$^{\circ}$ directions and between 0 and 180$^{\circ}$ directions.\label{fig:sine-plot}}
\end{figure}


\end{knitrout}



\subsubsection{Finding the wind from the GPS ground track}

From the definitions \eqref{eq:v-errors}, the mean wind can be found by adjusting the fit parameters $v_x$, $v_y$, $\delta V$ and $\delta \psi$ to minimize the error measure $\chi^2=\sum(\delta v_x^2+\delta v_y^2)$. The results, with wind converted to wind direction $\lambda$ and wind speed $v$, are shown in Table~\ref{tab:GSmin} for the full circle maneuver and also separately for the right-turn circles and left-turn circles.

\begin{center}
\begin{table}[H]
\begin{centering}
\begin{tabular}{cccccc}
\toprule 
 & $\lambda\,[^{\circ}]$ & $v$~{[}m/s{]} & $\delta V$~{[}m/s{]} & $\delta\psi^{\prime}\,[^{\circ}]$ & residual error {[}m/s{]}\tabularnewline
\midrule
\midrule 
all turns & 222.8 & 17.7 & 0.4 & -0.05 & 1.0\tabularnewline
\midrule 
mean of measurements & 222.8 & 18.0 &  &  & \tabularnewline
\midrule 
left turns & 222.9 & 18.3 & 0.5 & -0.05 & 1.0\tabularnewline
\midrule 
left-turn measurements & 222.8 & 18.5 &  & & \tabularnewline
\midrule 
right turns & 222.6 & 17.2 & 0.3 & -0.06 & 0.4\tabularnewline
\midrule 
right-turn measurements & 222.8 & 17.5 &  & & \tabularnewline
\bottomrule
\end{tabular}
\par\end{centering}

\protect\caption{The best-fit parameters that minimize the errors given by \eqref{eq:v-errors} for the first circle maneuver from DEEPWAVE flight 15, 3:38:30--3:56:30 UTC, at an altitude of about 4.1 km (13,500 ft). The lines "mean of measurements", "left-turn measurements" and "right-turn measurements" are based on the wind measurements from the aircraft data system.\label{tab:GSmin}}
\end{table}

\par\end{center}

There is an apparently significant difference of about 1~m/s in mean wind speed, both measured and resulting from the fits, between the right-turn circles and the left-turn circles. This is apparently a real difference between wind conditions in the regions where the circles were flown and is supported by the difference between maximum and minimum ground speed for the two sets of circles. Conditions are more uniform for the right-turn circles, as reflected in the lower residual error for those circles, so more weight will be given here to the results from the right-turn circles. 

The fits point to a very small heading error that is consistent for the two turn
directions. That result is very sensitive to the timing of signal acquisition; any
difference in timing between the ground-speed components from the GPS and the heading
measurement from the IRU will produce an indicated error in heading that changes sign with turn direction. Shifting either pitch or ground-speed components by 40\,ms changes the indicated errors in heading by 0.07$^{\circ}$, so these lags are very sensitive to the assumed delay. The results here were obtained for a heading shift of $-50$\,ms, slightly
different from the conventional shift of $-80$\,ms, and with no shift imposed on the GPS measurements of ground-speed components. Also, no offset was applied to heading. 
For documentation, the attributes for the heading variable as processed for this study are listed here:

\begin{knitrout}\footnotesize
\definecolor{shadecolor}{rgb}{0.969, 0.969, 0.969}\color{fgcolor}\begin{kframe}
\begin{verbatim}
## [1] "attributes for variable"
## [1] "_FillValue: -32767"
## [1] "units: degree_T"
## [1] "long_name: IRS Aircraft True Heading Angle"
## [1] "standard_name: platform_orientation"
## [1] "valid_range: c(0, 360)"
## [1] "actual_range: c(0.00287811458110809, 359.997497558594)"
## [1] "Category: Analog"
## [1] "SampledRate: 25"
## [1] "TimeLag: -50"
## [1] "TimeLagUnits: milliseconds"
## [1] "DataQuality: Preliminary"
## [1] "CalibrationCoefficients: c(0, 1)"
## [1] "modulus_range: c(0, 360)"
\end{verbatim}
\end{kframe}
\end{knitrout}


Two other circle patterns were flown on this same flight, at about 4:35:00--4:53:00 and 5:30:00--5:53:00 UTC. Similar tables for those circle maneuvers are shown in Tables.~\ref{tab:GS2min} and \ref{tab:GS3min}. In these three cases, indicated airspeed corrections from individual circles varied from $-0.6$ to 0.6\,m/s and angular corrections (combining heading and sideslip errors) varied from 0.01 to 0.23$^{\circ}$, with mean values of 0.2$\pm 0.2$\,m/s and 0.13$\pm 0.03^{\circ}$. 





\begin{center}
\begin{table}[H]
\begin{centering}
\begin{tabular}{cccccc}
\toprule 
 & $\lambda\,[^{\circ}]$ & $v$~{[}m/s{]} & $\delta V$~{[}m/s{]} & $\delta\psi^{\prime}\,[^{\circ}]$ & residual error {[}m/s{]}\tabularnewline
\midrule
\midrule 
all turns & 230.7 & 22.8 & 0.5 & -0.01 & 0.6\tabularnewline
\midrule 
mean of measurements & 230.4 & 22.2 &  &  & \tabularnewline
\midrule 
left turns & 231.3 & 22.6 & 0.5 & -0.02 & 0.6\tabularnewline
\midrule 
left-turn measurements & 231.2 & 22 &  & & \tabularnewline
\midrule 
right turns & 230.1 & 23 & 0.6 & -0.01 & 0.4\tabularnewline
\midrule 
right-turn measurements & 229.6 & 22.5 &  & & \tabularnewline
\bottomrule
\end{tabular}
\par\end{centering}

\protect\caption{The best-fit parameters that minimize the errors given by \eqref{eq:v-errors} for the second circle maneuver from DEEPWAVE flight 15, 4:35:00--4:53:00 UTC, at an altitude of about 6.1 km (20,000 ft). The lines "mean of measurements", "left-turn measurements" and "right-turn measurements" are based on the wind measurements from the aircraft data system.\label{tab:GS2min}}
\end{table}

\par\end{center}



\begin{center}
\begin{table}[H]
\begin{centering}
\begin{tabular}{cccccc}
\toprule 
 & $\lambda\,[^{\circ}]$ & $v$~{[}m/s{]} & $\delta V$~{[}m/s{]} & $\delta\psi^{\prime}\,[^{\circ}]$ & residual error {[}m/s{]}\tabularnewline
\midrule
\midrule 
all turns & 253.7 & 7 & \ensuremath{-0.2} & -0.04 & 1.3\tabularnewline
\midrule 
mean of measurements & 253.7 & 7.2 &  &  & \tabularnewline
\midrule 
left turns & 256.6 & 7.1 & \ensuremath{-0.6} & -0.14 & 1.4\tabularnewline
\midrule 
left-turn measurements & 256.4 & 7.6 &  & & \tabularnewline
\midrule 
right turns & 250.9 & 6.9 & 0.2 & 0.05 & 0.8\tabularnewline
\midrule 
right-turn measurements & 251.1 & 6.8 &  & & \tabularnewline
\bottomrule
\end{tabular}
\par\end{centering}

\protect\caption{The best-fit parameters that minimize the errors given by \eqref{eq:v-errors} for the third circle maneuver from DEEPWAVE flight 15, 5:30:30--5:53:00 UTC, at an altitude of about 9.1 km (30,000 ft). The lines "mean of measurements", "left-turn measurements" and "right-turn measurements" are based on the wind measurements from the aircraft data system.\label{tab:GS3min}}
\end{table}

\par\end{center}



\subsubsection{Minimizing the variation in measured wind speed in circle maneuvers\label{WS-var-circles}}

The fit results in Tables~\ref{tab:GSmin}--\ref{tab:GS3min} and also the mean measurements of the wind in those tables provide reference angles for the wind direction, so those angles can be used in fits that minimize \eqref{eq:delta-v-m}. 
Because there is significant variation among the circles, the tabulated values
for fitted values for the appropriate set (all turns, left turns, or right turns) will be used as the value of $\lambda$ when fitting. 
An appropriate error function to minimize to find values for the parameters $\delta V$, $\delta\psi^{\prime}$, and $v$ in \eqref{eq:delta-v-m} is $\chi_2^2=\sum{\delta v_m^2}$. 

Tables \ref{tab:Circlemin}--\ref{tab:Circle3min} show the fit results for the three circles:

\begin{center}
\begin{table}[H]
\begin{centering}
\begin{tabular}{cccccc}
\toprule 
 &  $v$~{[}m/s{]} & $\delta V$~{[}m/s{]} & $\delta\psi^{\prime}\,[^{\circ}]$ & residual error {[}m/s{]}\tabularnewline
\midrule
\midrule 
all turns  & 18 & 0.5 & -0.15 & 0.6\tabularnewline
\midrule 
left turns & 18.5 & 0.7 & -0.20 & 0.2\tabularnewline
\midrule 
right turns &  17.5 & 0.2 & -0.10 & 0.2\tabularnewline
\bottomrule
\end{tabular}
\par\end{centering}

\protect\caption{The best-fit parameters that minimize the errors given by \eqref{eq:delta-v-m} for the first circle maneuver from DEEPWAVE flight 15, 3:38:30--3:56:30 UTC.\label{tab:Circlemin}}
\end{table}

\par\end{center}


\begin{center}
\begin{table}[H]
\begin{centering}
\begin{tabular}{cccccc}
\toprule 
 &  $v$~{[}m/s{]} & $\delta V$~{[}m/s{]} & $\delta\psi^{\prime}\,[^{\circ}]$ & residual error {[}m/s{]}\tabularnewline
\midrule
\midrule 
all turns  & 22.3 & 0.7 & -0.08 & 0.3\tabularnewline
\midrule 
left turns & 22 & 0.8 & -0.07 & 0.2\tabularnewline
\midrule 
right turns &  22.5 & 0.7 & -0.08 & 0.2\tabularnewline
\bottomrule
\end{tabular}
\par\end{centering}

\protect\caption{The best-fit parameters that minimize the errors given by \eqref{eq:delta-v-m} for the second circle maneuver from DEEPWAVE flight 15, 4:35:00--4:53:00 UTC.\label{tab:Circle2min}}
\end{table}

\par\end{center}




\begin{center}
\begin{table}[H]
\begin{centering}
\begin{tabular}{cccccc}
\toprule 
 &  $v$~{[}m/s{]} & $\delta V$~{[}m/s{]} & $\delta\psi^{\prime}\,[^{\circ}]$ & residual error {[}m/s{]}\tabularnewline
\midrule
\midrule 
all turns  & 7.1 & \ensuremath{-0.5} & 0.17 & 1.0\tabularnewline
\midrule 
left turns & 7.5 & \ensuremath{-1.1} & 0.37 & 0.7\tabularnewline
\midrule 
right turns &  6.8 & 0.2 & 0.00 & 0.5\tabularnewline
\bottomrule
\end{tabular}
\par\end{centering}

\protect\caption{The best-fit parameters that minimize the errors given by \eqref{eq:delta-v-m} for the third circle maneuver from DEEPWAVE flight 15, 5:30:00--5:53:00 UTC.\label{tab:Circle3min}}
\end{table}

\par\end{center}


















Figures \ref{fig:first-circles}--\ref{fig:third-circles} show the mean measurements of wind speed as a function of the angle between the mean wind direction and the adjusted heading representing the motion of the aircraft relative to the air. The two turn directions are shown separately in each figure because they often require a different mean wind speed, as also shown in Tables~\ref{tab:GSmin}--\ref{tab:GS3min} where the fit results are tabulated. The fits to \eqref{eq:delta-v-m}, found by  minimizing the defined $\chi^2$, are also shown in this figures (orange lines).



\begin{knitrout}\footnotesize
\definecolor{shadecolor}{rgb}{0.969, 0.969, 0.969}\color{fgcolor}\begin{figure}

{\centering \includegraphics[width=\maxwidth]{figure/WU-first-circles-1} 

}

\caption{Measured wind speed from the left-turn and right-turn circles in the circle pattern shown in Fig.\ \ref{fig:circle-tracks}, as a function of $\xi$, the difference between the adjusted heading and the mean wind direction. Orange lines are the results of fitting \eqref{eq:delta-v-m} to the measurements, with results as listed in Table~\ref{tab:Circlemin}.\label{fig:first-circles}}
\end{figure}


\end{knitrout}
\begin{knitrout}\footnotesize
\definecolor{shadecolor}{rgb}{0.969, 0.969, 0.969}\color{fgcolor}\begin{figure}

{\centering \includegraphics[width=\maxwidth]{figure/WU-second-circles-1} 

}

\caption{Measured wind speed from the left-turn and right-turn circles in the second circle pattern as a function of $\xi$, the difference between the adjusted heading and the mean wind direction. Orange lines are the results of fitting \eqref{eq:delta-v-m} to the measurements, with results as listed in Table~\ref{tab:Circle2min}.\label{fig:second-circles}}
\end{figure}


\end{knitrout}
\begin{knitrout}\footnotesize
\definecolor{shadecolor}{rgb}{0.969, 0.969, 0.969}\color{fgcolor}\begin{figure}

{\centering \includegraphics[width=\maxwidth]{figure/WU-third-circles-1} 

}

\caption{Measured wind speed from the left-turn and right-turn circles in the third circle pattern  as a function of $\xi$, the difference between the adjusted heading and the mean wind direction. Orange lines are the results of fitting \eqref{eq:delta-v-m} to the measurements, with results as listed in Table~\ref{tab:Circle3min}.\label{fig:third-circles}}
\end{figure}


\end{knitrout}

Some of the fits, notably the left-turn circles from the third maneuver, don't conform very well to the expected sinusoidal error pattern. The residual error about the fit was the largest of the circles, with the right-turn circles having almost as large a residual error, which may indicate that conditions were not sufficiently uniform to use these maneuvers.
However, if all six are averaged, the mean values for the indicated error in airspeed is
0.25$\pm$0.32\,m/s, and the indicated error in adjusted heading $\psi^{\prime}$ is 
\ensuremath{-0.01} $\pm$ 0.09$^{\circ}$, where the indicated ranges are the standard deviations estimated for the mean values. These are useful constraints on the uncertainty associated with these key contributors to uncertainty in measured wind.





% \begin{center}
% \begin{table}[H]
% \begin{centering}
% \begin{tabular}{cccccc}
% \toprule 
%  & $\bar{v_{d}}\,[^{\circ}]$ & $\bar{v_{s}}$~{[}m/s{]} & TAS~{[}m/s{]} & $\delta\psi\,[^{\circ}]$ & residual error {[}m/s{]}\tabularnewline
% \midrule
% \midrule 
% all turns & format(round(bestWD,1), nsmall=1) & round(bestWS,1) & round(bestFit[1],1) & round(bestFit[4],2) & format(round(rmsC,1), nsmall=1)\tabularnewline
% \midrule 
% mean of measurements & format(round(mwd,1), nsmall=1) & format(round(mws,1), nsmall=1) & round(mtas,1) &  & \tabularnewline
% \midrule 
% left turns & format(round(bestWDL,1), nsmall=1) & round(bestWSL,1) & round(bestFitL[1],1) & round(bestFitL[4],2) & format(round(rmsL,1), nsmall=1)\tabularnewline
% \midrule 
% left-turn measurements & format(round(mwdL,1), nsmall=1) & round(mwsL,1) & round(mtasL,1) & & \tabularnewline
% \midrule 
% right turns & format(round(bestWDR,1), nsmall=1) & round(bestWSR,1) & round(bestFitR[1],1) & round(bestFitR[4],2) & format(round(rmsR,1), nsmall=1)\tabularnewline
% \midrule 
% right-turn measurements & format(round(mwdR,1), nsmall=1) & round(mwsR,1) & round(mtasR,1) & & \tabularnewline
% \bottomrule
% \end{tabular}
% \par\end{centering}
% 
% \protect\caption{The same results as shown in Table\ \ref{tab:GSmin} but with a time shift applied to the GPS measurements of ground track to advance the measurements by 840\,ms.\label{tab:GSminShift}}
% \end{table}
% 
% \par\end{center}
% 
% The error in heading shown in Table\ \ref{tab:GSmin} approximately reverses sign for right-turn vs left-turn circles. This error would arise if there is a timing error between the measurement of heading and that of ground speed, here obtained from the GPS system (variables GGVEW and GGVNS) of about 0.46\,s, because the turn rate in these turns is about 1.8$^{\circ}$/s so the
% indicated errors of about 1.5$^{\circ}$ suggest that, for both turn directions, advancing the GPS ground-speed measurements by about 0.8\,s would remove the errors from all the circles. Fitting to minimize the standard error of the fits gave a minimum chisquare for a shift of nbest samples or nbest$\times$40 = 40*nbest\,ms, as shown in Table\ \ref{tab:GSminShift}. With this shift, the indicated errors in heading are all quite small and consistent with expected errors in heading (<0.05$^{\circ}$. The residual error for the right-turn circles, 0.3\,m/s, is also very good; variations of this magnitude were present in the wind field and hence in TAS, so this is as low as could be expected.






% The preceding fit used a constant true airspeed, but it is also possible
% to fit in the same way for an assumed error in true airspeed, by using
% $V=V_{m}+\delta V$ where $V_{m}$ is the measured value and $\delta V$
% is an assumed error in that measurement. There is some small variation
% in measured true airspeed during the maneuver, perhaps created by
% the normal oscillation that results from the flight management system
% and is discussed elsewhere in this report, so this approach may be
% preferable. However, the resulting best-fit values were the same as
% those shown in Table~\ref{tab:GSmin}, to the level of significance
% listed in that table.
% 
% 
% \paragraph{Offsets in TAS and heading}
% 
% An alternate way of determining the offsets in airspeed and heading,
% which illustrates the value of the circle maneuver for developing
% these constraints, is to plot the dependence of measured wind speed
% $v_{s}$ on the heading. The expected variation is for $v_{s}$ to
% change by $2\delta V$ from upwind to downwind flight and by $2V\delta\psi$
% from crosswind-right to crosswind-left flight direction (i.e., 90$^{\circ}$
% right of downwind vs.~90$^{\circ}$ left). The net effect is to produce
% a variation in $v_{s}$ given by:
% 
% \begin{equation}
% v_{s}=\bar{v_{s}}+\delta V\cos\theta+V\delta\psi\sin\theta\label{eq:SinCosDep}
% \end{equation}
% where $\theta$ is the difference between the heading and the wind
% direction.\footnote{As developed later, if there is non-zero sideslip
% the heading and the angle $\theta$ should be corrected by adding $\delta\beta\cos(\phi)$ where $\phi$ is the roll. Also, an error in sideslip will contribute to $\delta\psi$.}%
% Figure \ref{fig:sine-cosine-plot} illustrates the expected
% dependence that would result from errors of $\delta V$ = 1~m/s and
% $\delta\psi=0.3^{\mbox{\ensuremath{\circ}}}$. The plot is constructed
% so that $0^{\circ}$ corresponds to downwind flight and the difference
% between values at 0 and 180$^{\circ}$corresponds to $2\delta V$,
% while the difference from $90$ to 270$^{\circ}$ represents $2V\delta\psi$. 



% It is possible to determine $\delta V$ and $\delta\psi$ by fitting
% (\ref{eq:SinCosDep}) to observations. The measurements will be shown
% separately for the left-turn circles and the right-turn circles because
% a significant difference appears between them as discussed earlier. 
% Figures~\ref{fig:left-turn-circles}
% and \ref{fig:right-turn-circles} show the measurements, and the results
% of the fits are shown in Table \ref{tab:SinCosCoefficients}.%
% \footnote{The values for wind speed are slightly higher than those listed in
% Table~\ref{tab:GSmin}, but the fit is slightly different and preferable
% in Table~\ref{tab:SinCosCoefficients} because measured variations
% in $V$ are included.}%
% 
% 
% \noindent \begin{center}
% \begin{table}
% \noindent \begin{centering}
% \begin{tabular}{ccccc}
% \toprule 
%  & mean wind {[}m/s{]} & $\delta V$~{[}m/s{]} & $\delta\psi\,[^{\circ}]$ & residual error {[}m/s{]}\tabularnewline
% \midrule
% \midrule 
% left-turn circles & format(round(cf1[1],1), nsmall=1) & format(round(cf1[3],1), nsmall=1) & format(round(cf1[2]/(mtas*pi/180),2), nsmall=2) & round(summary(fm1)$sigma,2)\tabularnewline
% \midrule 
% right-turn circles & format(round(cf2[1],1), nsmall=1) & format(round(cf2[3],1), nsmall=1) & format(round(cf2[2]/(mtas*pi/180),2), nsmall=2) & round(summary(fm2)$sigma,2)\tabularnewline
% \bottomrule
% \end{tabular}
% \par\end{centering}
% \protect\caption{Fit results for the left-turn and right-turn circles as fitted by
% (\ref{eq:SinCosDep}).\label{tab:SinCosCoefficients}}
% \end{table}
% \par\end{center}
% A difference of about $0.4^{\circ}$ in heading offset translates
% to a change in wind speed of about 1 m/s at representative GV flight speeds, so small errors
% can be quite significant. Because the wind is more nearly uniform and the pattern of measurements is closer to a pure sinusoidal pattern for the right-turn circles, and because Fig.\ \ref{fig:gs-only-plot} shows better consistency for the two right-turn circles vs the left-turn circles, here the right-turn circles will be emphasized. They indicate that the wind measurements (with time-shifting as discussed above) are quite good, with indicated errors of only 0.2\,m/s in true airspeed and 0.03$^{\circ}$ in heading. For the right-turn circles, the significant departures are for parts of the circles nearly into the wind direction, where the circles start and end, so it may be that those small deviations arise from imperfect flight patterns for those times. Without those parts of the curve, the indicated sinusoidal variation in Fig.\ \ref{fig:right-turn-circles} would be even smaller.
% 
% However, there is a complication in regard to heading that needs further exploration, so the next section will discuss that complication before offering a concluding assessment regarding uncertainty. 
% 

\paragraph{Offset in Sideslip\label{par:Offset-in-Sideslip}}

In the preceding, the error $\delta\psi^{\prime}$ was discussed as an error
in adjusted heading, but that adjustment includes the sideslip so 
the error could also be one in sideslip. These errors
are difficult to separate, and normal sideslip calibration (Sect.~\ref{sub:calibration-SS}
even with reverse-heading maneuvers does
not provide a separation. Furthermore, heading errors may change
during a flight because error terms undergo a Schuler oscillation
and are also affected by horizontal accelerations such as occur persistently
in turns like those in the circle maneuver. 

The error term determined
as in the above tables should be represented by $\delta\psi^{\prime}$ given by

\begin{equation}
\delta\psi^{\prime}=\delta\psi+\cos\phi\delta\beta\label{eq:dpsiprime}
\end{equation}
where $\phi$ is the roll angle and $\delta\beta$ is the offset in
sideslip.\footnote{The angle of attack was determined by separate
calibration in Sect.~\ref{sub:radome-sensitivity} so that
contribution to $\psi^{\prime}$ is not included in the error term.} 
Because the dependence in (\ref{eq:dpsiprime})
is on the cosine of the roll which is an even function, left and right
turns are affected in the same way and also cannot distinguish the two
terms in the equation. Even though three different altitudes were used for
these circle maneuvers, the GV flight management system maintains nearly the same
roll angle for all three maneuvers so changes in roll also cannot be used
to distinguish the contributions from heading offset and sideslip offset.

One approximate test is to compare the sideslip measurement in left vs right
turns. The measurements of sideslip%
\footnote{The sensitivity coefficients as determined in Sect.~\ref{sub:calibration-SS}
have been used to construct this figure.%
} are shown in Fig.~\ref{fig:plotSS}. In the turns,
some sideslip is introduced as the aircraft configuration remains
slightly nose-up during the turn, and that sideslip should reverse
sign by symmetry when the flight pattern changes from left-turn to right-turn
circles. The lift required to maintain altitude would be
the same if the roll angles were opposite, as they nearly are for
these maneuvers (0.45$^{\circ}$ larger for right turns), and the
angle of attack is also close to the same, so the expected sign reversal
in sideslip can be used to estimate the offset in sideslip. 


\begin{knitrout}\footnotesize
\definecolor{shadecolor}{rgb}{0.969, 0.969, 0.969}\color{fgcolor}\begin{kframe}
\begin{verbatim}
## [1] " mean for maneuver: 0.01; left turns -0.30 and right turns 0.30"
\end{verbatim}
\end{kframe}\begin{figure}

{\centering \includegraphics[width=\maxwidth]{figure/WU-plotSS-1} 

}

\caption[Measurement of sideslip during the first circle maneuver from DEEPWAVE flight 15, with left-turn circles from 3]{Measurement of sideslip during the first circle maneuver from DEEPWAVE flight 15, with left-turn circles from 3:40:00 -- 3:46:00 UTC, followed by a straight segment and then right-turn circles 3:47:30 -- 3:53:40 UTC.\label{fig:plotSS}}
\end{figure}


\end{knitrout}

The mean value for the full maneuver, and the average of 
the right-turn and left-turn circles, both indicate that there is no offset 
in the sideslip. 
% This is a forced result, though, because the sensitivity
% coefficients for sideslip have been adjusted from a preliminary version of
% this plot to give symmetry, and an adjustment of 0.07$^{\circ}$ was required. 
However, this approach is sensitive to any variations in pitch, and there were small
but significant variations in pitch in the maneuvers.\footnote{This was first noticed in
circle maneuvers by the C-130, where pitch variations are more significant during circles and this approach is necessary.} The alternative is to take into account the influence of all angles on vertical wind. The transformation equations from \cite{Bulletin23} (see also \cite{NCAR_OpenSky_TECH-NOTE-000-000-000-064}),
with some small-angle simplifications, lead to this equation for the
vertical wind:

\begin{equation}
u_{z}=w_{p}+V(\sin\phi\tan\beta+\cos\phi\tan\alpha-\sin\theta\label{eq:vwind-eq-for-SS}
\end{equation}
where $w_{p}$ is the vertical motion of the aircraft, $V$ its true
airspeed, $\phi$ the roll angle, $\beta$ the sideslip angle, $\alpha$
the angle of attack, and $\theta$ the pitch angle. If it is assumed
that the vertical wind is zero, in the circle maneuver this equation predicts that the sideslip
angle will be $\beta^{*}$ given by\\
\begin{equation}
\beta^{*}\approx\frac{\theta-\alpha\cos\phi-(w_{p}/V)}{\sin\phi}\,\,\,.\label{eq:beta-from-other-measurements} 
\end{equation}
This is not useful in straight flight where the roll is near zero, 
but it provides a valid equation for
sideslip in the case of steady turns. The key assumption is that the vertical wind is zero;
a vertical wind of 0.1\,m/s will typically increase the deduced sideslip from this
equation by about 0.06$^{\circ}$, so it is important that the circle maneuver be flown
where there is no mean updraft. It is then possible to determine
the offset in sideslip by comparing this prediction to the measured
sideslip $\beta_{m}$:\\
\begin{equation}
\delta\beta=\beta_{m}-\beta^{*}\,\,.\label{eq:predicted-error-in-sideslip}
\end{equation}

\begin{knitrout}\footnotesize
\definecolor{shadecolor}{rgb}{0.969, 0.969, 0.969}\color{fgcolor}\begin{kframe}
\begin{verbatim}
## [1] "mean error in sideslip: 0.084 +/-0.001"
\end{verbatim}
\end{kframe}\begin{figure}

{\centering \includegraphics[width=\maxwidth]{figure/WU-sideslip-error-histogram-1} 

}

\caption{Distribution of measurements of sideslip error determined from (\ref{eq:predicted-error-in-sideslip}) for all measurements in turns from the three circle maneuvers.\label{fig:sideslip-error-histogram}}
\end{figure}


\end{knitrout}

This result, indicating that 0.084$^{\circ}$
should be subtracted from the sideslip measurements,\footnote{The value is the measurement error so the required correction is the negative of that value.} is different from the indication from Fig.\ \ref{fig:plotSS}, where no offset was required.
The uncertainty in the estimated mean of this result is very low, and this result is more general because it uses measured values of the pitch, so this value will be used to correct the sideslip offset. This indicates that the sideslip sensitivity coefficients should be \{$e_{0}$,\, $e_{1}$\} = \{0.008, 22.302\}. Because this result is dependent on the vertical wind being zero, an uncertainty of at least 0.03$^{\circ}$ should be assigned to the first coefficient to recognize that the vertical wind might typically be $\pm$0.05\,m/s in a region such as this.% %msrmt 0.1, but biased/unreliable in high-bank turn?

%This result was obtained from files where a heading offset of -0.08$^{\circ}$ had been imposed. 
The results of the circle analyses point to a combined error from heading and sideslip of
\ensuremath{-0.01}$\pm$0.09$^{\circ}$,
so the indicated heading error 
calculated from \eqref{eq:dpsiprime} is \ensuremath{-0.09}$\pm$0.09$^{\circ}$. The required heading correction is then the negative of this value, and the result of this calibration
is to make compensating adjustments in the offsets for heading and sideslip.

% As a consistency check, we repeat the fit determining TAS and heading offsets after adjusting the sideslip measurements by the indicated correction. 
% 
% With this shift applied, the dependence of the left-turn and right-turn
% circles is shown in Fig.~\ref{fig:combined-circles}. The mean value of sideslip
% for the circle maneuvers is yrm; for left turns yrL and for right turns yrR. 



% The best-fit coefficients are as shown in the following table, where
% the values for left-turn-only and right-turn-only measurements are
% repeated:
% 
% \noindent \begin{center}
% \begin{tabular}{|c|c|c|c|c|}
% \hline 
%  & mean wind {[}m/s{]} & $\delta V$~{[}m/s{]} & $\delta\psi\,[^{\circ}]$ & residual error {[}m/s{]}\tabularnewline
% \hline 
% \hline 
% left-turn circles & round(cf1[1],1) & format(round(cf1[3],1), nsmall=1) & format(round(cf1[2]/(mtas*pi/180.),2), nsmall=2) & round(summary(fm1)$sigma,2)\tabularnewline
% \hline 
% right-turn circles & round(cf2[1],1) & format(round(cf2[3],1), nsmall=1) & format(round(cf2[2]/(mtas*pi/180.),2), nsmall=2) & round(summary(fm2)$sigma,2)\tabularnewline
% \hline 
% combined & round(cfc[1],1) & format(round(cfc[3],1), nsmall=1) & round(cfc[2]/(mtas*pi/180.),2) & round(summary(fmc)$sigma,2)\tabularnewline
% \hline 
% \end{tabular}
% \par\end{center}
% 
% The airspeed correction as applied differently to the two turn directions
% leads to one aspect of consistency in the measurements: The mean TAS,
% originally higher in the left turns but after correction higher in
% the right turns, conforms to the expected difference required by the
% higher absolute value of the roll angle in right vs left turns. If
% the lift $L$ is proportional to $V^{2}$ as expected, and $L\propto V^{2}\cos\phi$
% remains constant, the (from the total derivative of this expression)
% 
% \[
% \frac{2}{V}\delta V=-\frac{\sin\phi}{\cos\phi}\delta\phi
% \]
% \[
% \delta V=-\frac{V\tan\phi}{2}\delta\phi\simeq+0.11
% \]
% (i.e., an increase in roll of round(rmeanR+rmeanL,1)$^{\circ}$,
% as is the case for the right turns, leads to an expected increase
% in $V$ of round(dv,2)  while the corrected values for V
% for the left and right turns, respectively, are round(tasmL+cf1[3],2)
% and round(tasmR+cf2[3],2). The difference between the uncorrected
% measurements, round(tasmR-tasmL,2), has the opposite sign,
% while after correction the difference conforms to expectations for
% an increased roll angle.



% One correction that could lead to more consistency in the measurements
% is to impose a time shift in the measured ground-speed components
% from the GPS. The delayed measurements will be offset in different
% directions as the direction of the turns changes. For this reason,
% a time shift was introduced in the GPS-measured ground-speed components
% and the wind components were recalculated. The minimum-residual-error
% solution was for a shift of 240~ms; i.e., the measurements GGVNS
% and GGVEW were applied 240 ms ahead of where they were originally
% recorded in the data files.%
% \footnote{No offset was introduced in that original processing for these components.
% For THDG, the heading variable, an offset was introduced of 80~ms
% to account for the expected transmission time of the measurement from
% the IRU to the data system, so the best-fit lag between heading and
% ground speed is 160 ms. If the lag in THDG is changed, the lag in
% GGVEW and GGVNS should be changed also.%
% } In addition, the mean wind speed for left turns and separately for
% right turns was subtracted from the measured wind speed to compensate
% for the real difference in wind speed determined above, so the results
% could be combined in one plot. The sdandard residual error from the
% fit to the shifted measurements after this adjustment was reduced
% to round(summary(fms)$sigma,1)~m/s, which seems reasonable
% because (as shown in Fig.~\ref{fig:plot-time-shift}), the residuals
% could arise from real fluctuations in wind or in the precision of
% the circular flight track. The best-fit results were $\delta V=$format(round(cfs[3],1), nsmall=1)
% m/s and $\delta\psi=$round(cfs[2]/(mtas*pi/180.)-0.07,2)$^{\circ}$. 






 


\subsubsection{Summary\label{sub:HWsummary}}

The results obtained from analysis of the circle maneuvers are these:
 
\begin{enumerate}
\item The circle maneuvers indicate that the measured true airspeed (TASX) with current LAMS-based pressure corrections is accurate to within expected uncertainty. The fits indicate an error in true airspeed of 
0.25$\pm$0.32\,m/s,
%0.2$\pm 0.3$\,m/s, 
which is within the expected ($\pm$0.3\,m/s) uncertainty limits for TAS deduced using the LAMS calibration (\cite{CooperEtAl2014}). The uncertainty includes the calibrated value,
so it appears better to make no adjustment to true airspeed.
\item The indicated error in sideslip is 0.08$^{\circ}$, so the first sensitivity coefficient should be reduced by this amount. That leads to sensitivity coefficients for the radome of  \{0.008, 22.302\}. Cf.\ the discussion in Section\ \ref{sec:Calibrations} and the summary section \ref{CalSSsummary} at the end of that section.
It appears reasonable to consider that the bias in this measurement may be uncertain by
about 0.03$^{\circ}$ because the mean offset is determined with this standard deviation from the combination of the available circle maneuvers and this is also the sensitivity to 0.05\,m/s vertical wind during the maneuvers.
\item With the preceding sensitivity coefficients, an offset should be introduced to
heading of magnitude \ensuremath{-0.06}$\pm$0.09$^{\circ}$. The evidence
from the circle fits is that the combined bias associated with the heading and sideslip offsets is 
\ensuremath{-0.01}$\pm$0.09$^{\circ}$.
%0.02$\pm 0.09^{\circ}$, 
The heading and sideslip offsets are coupled, so the two offsets indicated by the circle fits should be used together.
\item The ground-speed measurements GGVEW and GGVNS from the Novatel GPS need to coincide with the inertial measurements. An assumed lag for pitch of 50~ms, along with no
offset in GGVEW and GGVNS, was indicated by the comparison between the two directions
used for the circle maneuvers.
\end{enumerate}





\subsection{The complementary filter\label{sub:comp-filter}}

Wind measurements combine a measurement of relative wind with a measurement
of aircraft motion to determine the air motion relative to the ground.
The aircraft motion has long been measured by an IRS, and recently
also by a GPS. These have complementary strengths: The IRS provides
very good information on short-term motion but drifts with a characteristic
period of more than an hour, while the GPS provides good absolute
accuracy but sometimes is unable to receive the GPS signals and (except
in differential-GPS mode) can have short-term errors that make short
segments of the track look jagged. To take advantage of the strengths
of each, a complementary-filter calculation was developed and implemented
in the 1980s, but it was never documented publicly. This section is
partly an attempt to remedy that and partly a suggestion to make some
minor changes to how it is implemented. This discussion is complementary
to the information in ProcessingAlgorithms.pdf, section 3.4, and contains
additional detail as well as notes regarding implementation of changes.

To accomplish this combining of measurements, a low-pass filter, $F_{L}(\{\mathrm{GVNS,GVEW\}})$,
is applied to the GPS measurements of ground speed, \{GVNS,GVEW\},
which are assumed to be valid for frequencies at or lower than the
cutoff frequency $f_{c}$ of the filter. Then the complementary high-pass
filter, denoted ($1-F_{L}$)($\{\mathrm{VNS,VEW\}}$), is applied
to the IRS measurements of ground speed, \{VNS,VEW\}, which are assumed
valid for frequencies at or higher than $f_{c}$. Ideally, the transition
frequency would be selected where the GPS errors (increasing with
frequency) are equal to the IRS errors (decreasing with frequency).
The filter used is a three-pole Butterworth low-pass filter, coded
following the algorithm described in \citet{Bozic1980}, p. 49. The digital filter used is
recursive, not centered, to permit calculation during a single pass
through the data. If the cutoff frequency lies where both the GPS
and INS measurements are valid and are almost the same, then the detailed
characteristics of the filter in the transition region (e.g., phase
shift) do not matter because the complementary filters have cancelling
effects when applied to the same signal. The transition frequency
$f_{c}$ was chosen to be (1/600) Hz (but this value can be overridden
via the ``defaults'' file). The Butterworth filter was chosen because
it provides flat response away from the transition. The resulting
variables for aircraft motion, \{VNSC,VEWC\}, are then each the sum
of two filtered signals, calculated as described in the following
box:\\
 %
\fbox{%
\begin{minipage}[t]{0.95\textwidth}%
VEW = IRS-measured east component of the aircraft ground speed\\
 VNS = IRS-measured north component of the aircraft ground speed\\
 GVEW = GPS-measured east component of the aircraft ground speed\\
 GVNS = GPS-measured north component of the aircraft ground speed\\
 $F_{L}()$ = three-pole Butterworth low-pass recursive digital filter\\
 \\
 \rule[0.5ex]{1\linewidth}{1pt}

\[
\{\mathrm{VNSC}\}=\{\mathrm{VNS}\}+F_{L}(\mathrm{\{GVNS\}-\{\mathrm{VNS\}})}
\]
\[
\{\mathrm{VEWC}\}=\{\mathrm{VEW\}+}F_{L}(\mathrm{\{GVEW\}-\{\mathrm{VEW}\})}
\]
%
\end{minipage}}

This is straightforward and effective when both sets of measurements
(IRS and GPS) are available. The approach in use becomes more complicated
when the GPS signals are lost, as sometimes happens in sharp turns.
Then some means is needed to avoid sudden discontinuities in velocity
(and hence wind speed), which would introduce spurious effects into
variance spectra and other properties dependent on a continuously
valid measurement of wind. To extrapolate measurements through periods
when the GPS measurements are not available, a fit is determined to
the difference between the best-estimate variables \{VNSC, VEWC\} and
the IRS variables \{VNS,VEW\} for the period before GPS reception
was lost, and that fit is used to extrapolate through periods when
GPS reception is not available. The procedure is described in section
3.4 of ProcessingAlgorithms.pdf.

The following provides more documentation of the fit procedure used
to determine the Schuler oscillation. The errors are assumed to result
primarily from this oscillation, so the three-term fit is of the form
$\Delta=c_{1}+c_{2}\sin(\Omega_{Sch}t)+c_{3}\cos(\Omega_{Sch}t)$
, where $\Omega_{Sch}$ is the angular frequency of the Schuler oscillation
(taken to be $2\pi/(5067\, s))$ and $t$ is the time since the start
of the flight. A separate fit is used for each component of the velocity
and each component of the position (discussed below under LATC and
LONC). The fit matrix used to determine these coefficients is updated
each time step but the accumulated fit factors decay exponentially
with about 30-min decay constant, so the terms used to determine the
fit are exponentially weighted over the period of valid data with
a time constant that decays exponentially into the past with a characteristic
time of 30 min. This is long enough to determine a significant portion
of the Schuler oscillation but short enough to emphasize recent measurements
of the correction. The procedures for accumulating the matrices for
the fit are as follows:\\
 \\
 %
\fbox{%
\begin{minipage}[t]{0.9\columnwidth}%
Define $u_{G}$ as the aircraft eastward velocity measured by the
GPS and $u_{I}$ the corresponding velocity measured by the IRS, so
that the difference is

\[
\delta u=u_{G}-u_{I}
\]
If $\Omega_{S}$ is the Schuler oscillation period, with $\Omega_{S}=2\pi t/T_{s}$
where $T_{s}=5040$ s, $\tau_{u}$ is the time constant for the update
(1800 s), $t$ is the time from the start of the flight, and the measurement
matrix is $A_{ij}$, then updated terms of the measurement matrix
each sample period ($A_{i,j}^{\prime}$) are (for $\delta u$):

\[
A_{0,1}^{\prime}=A_{0.1}(1-\frac{1}{\tau_{u}})+\delta u
\]
\[
A_{1,1}^{\prime}=A_{1,1}(1-\frac{1}{\tau_{u}})+\delta u\,\sin(\Omega_{S}t)
\]
\[
A_{2,1}^{\prime}=A_{2,1}(1-\frac{1}{\tau_{u}})+\delta u\,\cos(\Omega_{S}t)
\]


The matrix components $A_{j,0}$ apply to the northward velocity component
and so are represented by the same equations with $\delta u$ replaced
by $\delta v$. Similar matrices are calculated for latitude $\theta$
and longitude $\phi$, based on the differences $\delta\theta$ and
$\delta\phi$ between GPS and IRS measurements. The information matrix
$H_{ij}$ is calculated via

\[
H_{i,j}=H_{i,j}(1-\frac{1}{\tau_{u}})+V_{i,j}
\]


where $V_{0,0}=1$, $V_{0,1}=V_{1,0}=\sin(\Omega_{S}t)$, $V_{0,2}=V_{2,0}=\cos(\Omega_{S}t)$,
$V_{1,1}=\sin^{2}(\Omega_{S}t)$, $V_{1,2}=V_{2,1}=\sin(\Omega_{S}t)\cos(\Omega_{S}t)$,
and $V_{2,2}=\cos^{2}(\Omega_{S}t)$. When the fit is needed, the
matrix $H_{i,j}$ is inverted and the result multiplied by the measurement
matrix $A_{i,j}$ to get the fit coefficients $C_{ij}$ to use for
predicting the results for $\delta u$, $\delta v$, $\delta\theta$,
and $\delta\phi$ via equations like $\delta u=C_{0,1}+C_{1,1}\sin(\Omega_{S}t)+C_{2,1}\cos(\Omega t)$.%
\end{minipage}}
% \end{document}

\appendix


%% LyX 2.1.3 created this file.  For more info, see http://www.lyx.org/.


% define format for exercises and examples:
\newcount\exernumber \exernumber=1
\newcount\examnumber \examnumber=1
\def\beginexercise{\bigskip\goodbreak\hrule\nobreak\smallskip\nobreak\hrule\nobreak\bigskip\sl\nobreak
{\bf\noindent EXERCISE \arabic{chapter}.\the\exernumber:\enspace}\advance\exernumber by 1}
\def\endexercise{\nobreak\bigskip\nobreak\hrule\nobreak\smallskip\nobreak\hrule\nobreak\bigskip\goodbreak\rm}
\def\beginexample{\bigskip\goodbreak\nobreak\hrule\nobreak\smallskip\nobreak\hrule\nobreak\bigskip\sl\nobreak
{\bf\noindent EXAMPLE \arabic{chapter}.\the\examnumber:\enspace}\advance\examnumber by 1}
\def\endexample{\endexercise}


\section{Appendix: Conventions for uncertainty analysis}

So that this document might serve as a model for future analyses of
uncertainty, this appendix documents some of the conventions followed
here and suggested for standardized use.


\subsection{Why perform analyses of uncertainty?}

When measurements are made to test scientific theories, provide input
to models, or characterize nature, they are only useful if accompanied
by some sense of their reliability. A key use of uncertainty analysis
is to provide this sense, in as quantitative terms as can be justified.
A quoted value should be considered incomplete unless accompanied
by some sense of the associated uncertainty. A target is to estimate
confidence limits to be associated with measurements or to be propagated
to final scientific results. Although it is usually impossible in
a strict statistical sense to provide formal estimates of confidence
limits, this target still underlies approaches to uncertainty analysis.
If those who make measurements don't characterize their reliability,
others must make their own (probably less informed) evaluations.

There are additional benefits of analyzing measurement uncertainty.
If an uncertainty analysis is done before an experiment, it may suggest
ways to refine the experiment to minimize critical uncertainty contributions,
and it should be possible to judge if the desired uncertainty is attainable.
An uncertainty analysis also highlights the dominant sources of error
and so can guide efforts to improve instruments.


\subsection{Error, accuracy, and uncertainty}

The \emph{error} in a measurement is the difference between the measurement
and the correct value of the measurand. A measurement
is of little use unless there is some way of estimating how large
this error may be. This estimate is called the \emph{uncertaint}y.\footnote{Results are sometimes classified according to their use: \textit{indication}
based only on primary measures such as sample means or correlation
coefficients; \textit{determination} based on primary and secondary
statistics, so that some estimate of uncertainty is obtained; and
\textit{inference}, in which a specific mathematical model is used
to assess uncertainty quantitatively. Often, a considerable amount
of information about the underlying distribution must be known (or
assumed) before statistical inference is possible. Experimental results
are usually appropriately quoted as determinations.} The uncertainty usually can be estimated in some way from knowledge
of the performance of an instrument or from calibrations, intercomparisons,
or statistical analysis of repeated measurements of the same quantity. 

The term \emph{accuracy} is often used erroneously where \emph{uncertainty}
would be appropriate. \emph{Accuracy} is determined by the
presence or absence of error, not uncertainty; a measurement may by
chance be accurate and still have a large uncertainty. \emph{Measurement}
\emph{uncertainty} is the correct term for an estimate of the limits
to the experimental error; it is incorrect to refer to this as the
measurement \emph{accuracy,} although that is unfortunately common
usage. \emph{Accuracy} is sometimes used to refer to error, not uncertainty,
but because accuracy is an absolute term even this usage is best avoided.
A measurement will either be accurate or not.  


\subsection{Standards for evaluating uncertainty}

Many different measures are used to characterize measurement error,
often making it difficult to determine which interpretation should
be associated with a quoted uncertainty. However, there is now an
established international consensus, defined by the International
Organization for Standardization (ISO) and by the US National Institute
of Standards and Technology (NIST), and this or modified forms have
also been adopted by many engineering societies. Acceptance of this
methodology followed decades of debate within engineering societies
and among international groups, and finally reached standardization
through the recommendations of the International Committee on Weights
and Measures. The two key publications defining these standards are
the \href{http://www.iso.org/iso/iso_catalogue/catalogue_ics/catalogue_detail_ics.htm?csnumber=50461}{Guide to the Expression of Uncertainty in Measurement}
(often referred to as the ``GUM'') and NIST Technical Note 1297
{[}1994 revision{]}, \href{http://www.nist.gov/pml/pubs/tn1297/index.cfm}{\textquotedblleft Guidelines for Evaluating and Expressing the Uncertainty of NIST Measurement Results.\textquotedblright}
The latter is available in full from the web link and will be the
primary reference followed in this document.\footnote{While the methodology described here is consistent with recommendations
from those publications, it seems appropriate in addition to advocate
separate estimation of the uncertainty associated with systematic
errors because the validity of such estimates often depends on judgment
and so is much harder to defend than in the case of random error. } 


\subsection{Classification of sources of error and of uncertainty}

\emph{\label{sec:Classification-of-sources}Errors} are often classified
as ``systematic'' or ``random,'' the former arising from consistent
and repeatable sources (like an offset in calibration) and the latter
from fluctuations about the measurand that are expected to average
to zero in a repeated series of measurements. The former are also
called ``biases'' when they arise from characteristics of an instrument.
It is straightforward to differentiate these error classes by this
test: Random errors are reduced when an experiment is repeated many
times and the results averaged together, while systematic errors remain
the same. Systematic errors can be reduced by better equipment or
better calibration or better experimental procedures. Figure \ref{fig:IllustrationOfTerms}
illustrates these terms.

\begin{figure}[H]
\noindent \begin{centering}
\includegraphics[width=11cm]{A-figure1}
\par\end{centering}

\protect\caption{\emph{\label{fig:IllustrationOfTerms}Illustration of the separate
effects of bias errors and random errors. The true population mean
is $\mu$, but an instrument is used that has a bias $b$ and measures
with random error (in each observation) $\sigma$. The resulting estimate
of the mean, obtained from $\overline{x}$, is in error because of
the separate contributions of the bias error $b$ and the random error
in the measurement of the mean, in this case $(\overline{x}-a)$.
The precision of the instrument is $\sigma$, so the estimated random
error in the mean is $\sigma/{N}^{1/2}$. The actual error in an experiment
is the difference between the true value $\mu$ and the measured value
$\overline{x}$. The histogram represents a frequency distribution
measured in a particular experiment, with mean shown as the solid
line labeled $\overline{x}$. In a large number of observations, it
would be expected that the results would tend toward the smooth dashed
curve with mean $a=\mu+b$. The measured standard deviation is $s$,
but the limiting value for a large number of measurements is expected
to be $\sigma$.} }


\end{figure}


It is awkward that most of the mathematical treatments of errors deal
with random errors, while most errors encountered in experimental
research are instead systematic errors. Digitization noise\footnote{I.e., the error that results when a continuously varying measurement
is measured by a digital instrument that must round the measurement
to the nearest digital value.} and the errors introduced when counting finite numbers of events\footnote{When, for example, the average measurement might be the possibly fractional
value $x$ but the actual value must be an integer.} are among the few good examples of random errors in modern experiments.\footnote{The prevalence of systematic error is a particularly compelling reason
to follow the methodology advocated here, because that approach features
parallel treatment of systematic and random errors and focuses attention
on their different characteristics. These separate error sources should
be investigated and treated in different ways, and should be reported
separately.} Analyses of uncertainty are made more difficult when most sources
of uncertainty are Type-B. Evaluation of standard uncertainty for
such sources is often subjective, based on judgment, and hard to quantify
or defend rigorously. Repeated calibrations, intercomparisons among
different instruments, and long-term stability of the calibrations
can all provide information on Type-B uncertainty.

Error contributions thought to be random may really be systematic,
and evaluating their associated uncertainty via Type-A methods may
not reveal that dependence. An example often cited as a possible source
of random error is a dependence of an instrument on line voltage,
causing fluctuations in the response function of an instrument during
an experiment. However, line voltage fluctuations are seldom random,
and are probably biased in a particular direction relative to the
conditions at the time of calibration, so it is likely that in a given
experiment or series of experiments such fluctuations will introduce
a bias. Furthermore, such errors are likely to be correlated in time,
so the usual procedure of assuming random error contributions to be
independent for different measurements will not be valid. Estimating
the associated uncertainty via standard Type-A methods can thus be
misleading in such a case. Close inspection of other common sources
of error shows that they are often biases, and this increases the
importance of estimating the associated uncertainty appropriately.
Other examples will be given in later sections.


\subsection{Recommended Guidelines}

These are central features of the methodology recommended and used
here:
\begin{enumerate}
\item Components introducing uncertainty are classified into two categories,
Type-A and Type-B (as defined in section \ref{sec:Classification-of-sources}),
and \emph{standard uncertainties} are evaluated for each component.
The estimated coverage associated with these evaluations is, in the
case of Type-A components, that corresponding to one standard deviation.
This is not quantifiable in a manner that can be defended rigorously
in the base of Type-B errors, but estimating a standard uncertainty
remains a goal.
\item To obtain the combined evaluation of uncertainty resulting from the
net effects of many uncorrelated sources, the standard uncertainties
are combined in quadrature,\footnote{i.e., $s^{2}=\sum_{i}s_{i}^{2}$.}
and the number of degrees of freedom in the combined uncertainty is
estimated from the Welch-Satterthwaite equation (cf. (\ref{eq:WS})).
For cases with correlations among components, methods that treat these
correlations must be used, as specified in a subsequent section. A
complete uncertainty report should also include an estimate of the
number of degrees of freedom associated with the result.
\item If some standard uncertainties are asymmetrical, the positive and
negative values should be combined separately to obtain separate upper
and lower composite values. 
\item The recommended uncertainty to be reported is the combined standard
uncertainty, evaluated to represent a single standard deviation. Other
estimates (e.g., that covering a specified level of confidence) can
be obtained readily from this, provided that the number of degrees-of-freedom
in the result is also reported. 
\item \emph{{[}Not part of the standard:{]}} The uncertainty report should
also include separate estimates of precision and bias in the result. 
\end{enumerate}
An uncertainty report will normally include a tabulated list of sources
of uncertainty, which should have separate sections for distinct influences
like those arising from calibration, data collection, and data analysis.
It is also useful to include associated estimates of precision, degrees
of freedom, and bias for each contribution shown in the table. Such
tabulations make it possible to isolate major sources of error, to
consider the validity of other investigators' estimates of error sources,
and to repeat the analyses for a new case when only one of the contributions
has changed. 

An important aspect of this methodology is that the degrees of freedom
associated with cited estimates should be calculated and quoted. This
becomes important when the number of degrees of freedom in the result
is small, so that error limits and propagated errors have non-Gaussian
character. Even if it is assumed that the individual measurements
are distributed according to a Gaussian error distribution, the true
standard deviation for an average of $n$ samples, $\sigma_{n}$,
is not known and must be estimated from the observations. The test
statistic $t=(\overline{x}-\xi)/S_{n}$ (where $\overline{x}$ is
the average of $n$ measurements, $\xi$ is the true value of $x$,
and $S_{n}$ is the estimated standard deviation of the average $\overline{x}$
about $\xi$, determined from $S_{n}=[\sum_{i=1}^{n}(x_{i}\overline{x})^{2}/(n(n-1))]^{\frac{1}{2}}$)
will not be Gaussian distributed. The appropriate distribution for
such averages is the Student-t distribution. The difference between
the Gaussian and Student-t distributions is generally insignificant
when the number of degrees of freedom\footnote{In the case of an average of n values, the number of degrees of freedom
is $n-1$.} exceeds about thirty, but for small sample sizes the differences
can be quite important. For this reason, when $n<30$, the confidence
limits used should be taken from the Student-t distribution rather
than from the normal distribution.

Figure \ref{fig:Student's t distribution} shows the relationship
between the 95\% confidence limits and the t statistic for the Student-t
distribution. 
\begin{figure}
\begin{centering}
\includegraphics[width=5in,keepaspectratio]{A-figure2}
\par\end{centering}

\protect\caption{\emph{\label{fig:Student's t distribution}Confidence limits for the
Student's t distribution.}}
\end{figure}
 If the final number of degrees of freedom is larger than thirty,
the range to select for the precision error limit in the result is
easily determined by use of the estimated standard deviation in the
result, the approximate 68\% confidence limit. Otherwise, it is necessary
to know the effective number of degrees of freedom in the final result,
as shown in Fig.~\ref{fig:Student's t distribution}. The Welch-Satterthwaite
formula provides an estimate: 
\begin{equation}
n_{r}={\frac{{[\sum_{i}S_{Y,i}^{2}]^{2}}}{{\sum_{i}{S_{Y,i}^{4}}/n_{i}}}}\label{eq:WS}
\end{equation}
 where n$_{r}$ is the number of degrees of freedom in the final result,
$S_{Y,i}$ is the standard deviation in $Y$ that would result from
error source $i$ alone, and $n_{i}$ is the number of degrees of
freedom in that source of uncertainty.

The format advocated and followed here for an analysis of uncertainty
includes these components:\\
 %
\fbox{\begin{minipage}[t]{1\columnwidth}%
\begin{enumerate}
\item \textit{Description of the measurement system with discussion of the
limits within which the analysis to be presented is valid.} For example,
the uncertainty in measurements of wind for a research aircraft might
be specified for straight-and-level flight within three hours of takeoff
(because of drift of the inertial navigation system), perhaps within
some altitude range. This description should discuss the calibration
procedures, tests to characterize measurement uncertainty, data processing,
and propagation of uncertainty to derived quantities.
\item \textit{Tabulation} \textit{and classification of} \textit{the elemental
sources} \textit{of uncertainty. }An example will be shown in Table
2.1. Each elemental source should be listed with its associated standard
uncertainty $u_{i}$ and, for Type-A sources, the number of degrees
of freedom ($n_{i}$). It is also convenient to tabulate the effect
of the error source on the final measurement $Y$ by including entries
for $u_{i}(\partial Y/\partial x_{i})$ in the tables, where $u_{i}$
is the standard uncertainty in the uncertainty-component $x_{i}$.
This simplifies propagation to the final result, although special
treatment is still needed in cases where the contributions are correlated.
The sources should be classified into groups contributing to calibration,
data acquisition, and data processing, or into similarly meaningful
groups for the instrument under consideration.
\item \textit{Discussion of} \textit{each elemental source} \textit{of uncertainty
in the table(s)} \textit{along with a} \textit{description of} \textit{the
basis for the evaluation. }These discussions should reflect the evidence
for the tabulated values.
\item \textit{The resulting} \textit{combined standard uncertainty,} \textit{combining
all sources into one value. }It is also useful to combine contributions
to random and systematic error (or precision and bias) into separate
composite values. 
\item \textit{Summarize} \textit{the results and the uncertainty limitations
of the measurement.} It is helpful here to highlight the main sources
of error and possible actions that could improve the measurements. \end{enumerate}
%
\end{minipage}}


\subsection{Estimating uncertainties}

Like errors, estimates of \emph{uncertainty} are also classified into
two categories, ``Type-A'' (evaluated by statistical methods) and
``Type-B'' (evaluated by other means). 


\subsubsection{Type-A evaluation}

A Type-A evaluation of uncertainty is based on statistical analysis
of repeated measurements or knowledge of the statistical character
of the observations. Standard statistical measures and approaches
including the standard deviation, analysis of variance, propagation
of error, etc., can provide the needed estimate. For example, the
standard?deviation?$s_{i}$ in repeated measurements of the same quantity
leads to an estimate of?standard?uncertainty??$u_{i}=s_{i}$.?Statistical
means can also provide the required estimate of the degrees of freedom
associated with the standard uncertainty.

Two important points need to be made in regard to Type-A evaluations:
\begin{enumerate}
\item ``Type-A'' refers to how the estimate is obtained, not to the type
of error. Type-A evaluations often result in estimates of systematic
error. For example, a random error affecting calibration of an instrument
can subsequently produce a systematic error when the instrument is
used, but the uncertainty can still be estimated via a Type-A evaluation.
This uncertainty component is therefore Type-A, even though the associated
error being characterized is systematic. This has sometimes been called
``fossilization'' of error: The random error in the calibration
procedure becomes a bias when that calibration is used. 
\item Variability in a measurement may result from random measurement error,
but it also may result from variability in the quantity being measured.
Variations in a measurement arising from true variation in the quantity
being measured cannot be used to estimate random measurement error,
although they may place upper limits on that error. When using a standard
devistion in repeated measurements to estimate standard uncertainty,
it is necessary to correct for any contribution from natural variability.
\end{enumerate}
A particularly clear?example of a Type-A evaluation leading to a proper
estimate of precision is that where the measurement consists of counting
discrete events, such as cloud droplets or particles. The uncertainty
in such measurements is expected to be characterized by Poisson statistics
if the events occur at times determined from random distribution of
the droplets in space. 


\subsubsection{Type-B evaluation}

\noindent A Type-B?evaluation?of?standard?uncertainty?is?more?dependent?on?judgment
and experience so it is harder to?defendthan Type-A evaluations. Some
guidance can be obtained from the following, but it must be acknowledged
that these are imperfect and not quantitative, so another analyst
with different judgment and experience could well disagree with the
estimate. In place of statistical measures, information obtained from
intercomparisons with other instruments, performance against standards,
?repeated calibrations, stability of the measurements, and specifications
of components can all contribute to Type-B estimates of uncertainty.
Nevertheless, it is useful to attempt to make Type-B evaluations that
are as far as possible comparable to Type-A evaluations. For example,
it is a goal that the ``coverage'' of the estimate be comparable
to a standard deviation. The following may provide some guidance when
developing such estimates: 
\begin{itemize}
\item If the error source is expected to be within the limits $\pm a$ 50\%
of the time, then $u_{j}\approx1.5a$.?
\item If it is expected to be within those limits about 2/3 of the time,
then $u_{j}\approx a$. 
\item If the quantity is expected to be within those limits 100\% of the
time, but equally probable anywhere in this range, then use $u_{j}\approx a/\surd3$. 
\item If?the?limits?are interpreted as 3-standard-deviation limits, then
$u_{j}\approx a/3$. (The NIST TN and GUM provide other examples?also.)?
\end{itemize}

\subsection{The composite or net uncertainty}

The tabluated standard uncertainties should then be combined to a
single standard uncertainty, $u_{c}$, which incorporates all sources
of uncertainty. ?Where possible, degrees of freedom should also be
provided. The?recommended?uncertainty?to?quote?with?results?is the
standard uncertainty?$u_{c}$; This?is?a?departure?from?earlier?practice,?favoring?2?standard
deviations or 95\% confidence limits.\footnote{``Confidence limits'' should only refer to Type-A evaluations; the
term ``coverage probability'' is sometimes used to emphasize the
difference between Type-B evaluations and those obtained via statistical
analysis.} NIST continues to accept such estimates also, and uses the term\emph{
expanded uncertainty} (symbol $U$) such that $U=2u_{c}$.


\subsection{Propagation of uncertainty estimates}

The interesting quantities for research are often derived from the
basic measurements by calculations that combine many measurements
into final quantities, transform the measurements, apply filters,
or otherwise convert the fundamental measurements into derived quantities.
In such cases, the uncertainty characteristics of the derived quantities
can become quite complicated and difficult to understand without a
prescribed methodology, and serious errors in interpretation can result.
For example, some attempts to derive correlations between radar reflectivity
($Z$) and rainfall ($R$) have been distorted by the problem that
both are based on different calculations from the same characteristics
of the drop size distributions, and hence there is a natural correlation
between the two that arises purely from correlated error sources.
If data sources are used that provide imprecise estimates of $Z$
and $R$, a correlation will appear that is purely the result of these
correlated error contributions and does not reflect a natural correlation
between radar reflectivity and rainfall rate. It would be a serious
error to use the correlation determined in this way to estimate rainfall
from radar reflectivity.

The following develops a consistent approach, often called ``error
propagation,'' that makes it possible to determine the uncertainty
characteristics in derived quantities if the characteristics of the
fundamental measurements are known. Let $\{x\}$ = $\{x_{\ell},\ell=1,L\}$
be a set of measured quantities with known measurement uncertainties.\footnote{Brackets denote multidimensional quantities and bold-face symbols
denote matrices.} Consider derived quantities $\{Y\}$ = $\{Y_{m},~m=1,M\}$, each
of which is a function of the measured quantities $\{x\}$: 
\begin{equation}
Y_{m}=Y_{m}(x_{1},x_{2},\dots x_{L})~or~{\textbf{Y}}={\textbf{Y}}({\textbf{x}}).
\end{equation}
 The mean values of $x_{\ell}$, $\overline{x_{\ell}}$, are then
the ``best'' values for $x_{\ell}$ in the sense that they minimize
the squares of the deviations from these best values. In the same
sense, the ``best'' values for $Y_{m}$ are the values $Y_{m}(\overline{x_{1}},\overline{x_{2}},\dots\overline{x_{L}})$.\footnote{These values do not necessarily minimize the sum of the squared deviations
from $\{\overline{Y}\}$, the derived quantities, unless the relationship
to the measured quantities is linear.}

The one-standard-deviation uncertainties in $\{Y\}$ are those that
represent the range over which $\{Y\}$ can vary while $\{x\}$ remain
within one-standard-deviation of their measured values. For small
deviations, a first-order Taylor expansion relates deviations in $\{Y\}$
to deviations in $\{x\}$: 
\begin{equation}
Y_{m}(x_{1},x_{2},\dots\,\,x_{L})=Y_{m}(\overline{x_{1}},\overline{x_{2}},\dots\,\,\overline{x_{L}})+\sum_{k=1}^{L}{\frac{{\partial Y_{m}(x_{1},x_{2},\dots\,\,x_{L})}}{{\partial x_{k}}}}\Bigr|_{\overline{x}}(x_{k}-\overline{x_{k}}).
\end{equation}
 The variance in $Y_{m}$ is then obtained by averaging over the $N$
measurements, indicated by index $i$: 
\begin{equation}
V_{Y_{m}Y_{m}}={\frac{{1}}{{N}}}\sum_{i=1}^{N}\bigl(Y_{m}(\{x\}_{i})-Y_{m}(\{\overline{x}\})\bigr)^{2}
\end{equation}


\begin{equation}
~~~~={\frac{{1}}{{N}}}\sum_{i=1}^{N}\Bigl[\bigl(\sum_{j=1}^{L}{\frac{{\partial Y_{m}}}{{\partial x_{j}}}}\Bigr|_{\overline{x}}(x_{i}-x_{j})\bigr)\bigl(\sum_{k=1}^{L}{\frac{{\partial Y_{m}}}{{\partial x_{k}}}}\Bigr|_{{\overline{x}}}(x_{k}-x_{i})\bigr)\Bigr]
\end{equation}
\begin{equation}
~~~~~=\sum_{j}\sum_{k}{\frac{{\partial Y_{m}}}{{\partial x_{j}}}}\Bigr|_{\overline{x}}{\frac{{\partial Y_{m}}}{{\partial x_{k}}}}\Bigr|_{\overline{x}}\bigl({\frac{{1}}{{N}}}\sum_{i}(x_{ji}-\overline{x_{j}})(x_{ki}-\overline{x_{k}})\bigr)\,\,\,.\label{eq:DerivedCoverianceMatrix}
\end{equation}


The matrix elements 
\begin{equation}
H_{jk}^{-1}={\frac{{1}}{{N}}}\sum_{i=1}^{N}(x_{ji}-\overline{x_{j}})(x_{ki}-\overline{x_{k}})
\end{equation}
 entering (\ref{eq:DerivedCoverianceMatrix}) are the variances and
covariances of the measured quantities, so ${\textbf{H}}^{-1}$ is
called the \textit{covariance matrix} or the \textit{error matrix.}
If the relationship between $\{Y\}$ and $\{x\}$ is linear or is
assumed linear (as in the first-order Taylor expansion) over the range
of fluctuations, then this matrix is particularly useful for determining
the variances in derived quantities because those variances can be
expressed as 
\begin{equation}
V_{Y_{m}Y_{n}}=\langle(Y_{m}-\overline{Y_{m}})(Y_{n}-\overline{Y_{n}})\rangle
\end{equation}
\begin{equation}
~~~~~=\sum_{j=1}^{L}\sum_{k=1}^{L}{\frac{{\partial Y_{m}(x)}}{{\partial x_{j}}}}\Bigr|_{x=\overline{x}}{\frac{{\partial Y_{n}(x)}}{{\partial x_{k}}}}\Bigr|_{x=\overline{x}}H_{jk}^{-1}
\end{equation}
 or, in matrix notation, 
\begin{equation}
V=T^{t}H^{-1}T\label{eq:refa}
\end{equation}
 where $T_{mj}$ = $\partial Y_{m}/\partial x_{j}$ is the element
of the (column) matrix of derivatives of the derived quantity $Y_{m}$
with respect to the measured quantity $x_{j}$ and the superscript
$t$ denotes the transpose matrix. This general form is valid for
any correlations among the original measurements (which will be represented
by off-diagonal elements of $H$) and properly represents the correlations
among dependent variables.

\beginexample
\htmlrule
\html{\textbf{Example 2.1:}} A thermocouple can be used to measure
temperature, because a junction between two metals will produce a
voltage difference in the two metals that is dependent on (and nearly
proportional to) the temperature of the junction. A common experimental
set-up is shown in Fig.~\ref{fig:ThermocoupleExperiment}.
\begin{figure}[H]
\begin{centering}
\includegraphics[width=4.5in]{A-figure3}
\par\end{centering}

\protect\caption{\emph{\label{fig:ThermocoupleExperiment}Experimental configuration
for measuring temperature with a thermocouple. Junctions J1 and J2
are junctions between copper and constantan wire, so the voltage V1
is a measure of the temperature difference between $T$ and $T_{ref}$.
A thermistor R$_{t}$ measures the bath temperature via the voltage
V2, and so provides a reference temperature to be added to the temperature
difference measured by the thermocouple.}}
\end{figure}
 The thermocouple junctions both produce voltage differences, dependent
respectively on the temperature $T$ and on the reference bath temperature
$T_{ref}$. The reason for using this arrangement is that both the
wires leading to the instrument measuring the voltage $V$ are then
copper wires, and can connect to copper junctions at the voltmeter
without introducing additional contact potentials such as would result
if the constantan wire were connected directly to the voltmeter. The
uncertainty in $T$ is then caused by two sources: (a) the uncertainty
in the measurement of $\Delta T$ = $T-T_{ref}$, and (b) the uncertainty
in $T_{ref}$. Often, a thermistor is used to measure the temperature
of the reference bath (or of a metal block used in the same way).

If a thermistor is used to determine the temperature of the reference
junction, as shown, there are two voltages that must be measured to
determine the unknown temperature $T$: $V_{1}$, produced by the
pair of thermocouples, and $V_{2}$, produced by the thermistor. These
are related to the temperature difference $\Delta T$=($T-T_{ref}$)
and to $T_{2}$, the temperature of the thermistor junction, by functions
$Y_{1}$ and $Y_{2}$, which often are almost linear relationships:
\begin{equation}
\Delta T=Y_{1}(V_{1})=a_{1}V_{1}
\end{equation}
\begin{equation}
T_{2}=Y_{2}(V_{2})=a_{2}V_{2}.
\end{equation}
 Then the first two fundamental quantities affecting the measurement,
in the earlier notation, are $x_{1}=V_{1}$ and $x_{2}=V_{2}$.

If $V_{1}$ and $V_{2}$ are measured by the same voltmeter, part
of the uncertainty in $V_{2}$ will be correlated with that in $V_{1}$
because bias in the voltmeter will affect both measurements in the
same way. This will be reflected in off-diagonal terms in the error
matrix, representing correlations between errors in $V_{1}$ and $V_{2}$.

There will also be an error in the measurement of $T$ introduced
by the assumption that $T_{ref}=T_{2}$, because the temperature bath
or constant-temperature block may not be uniform in temperature. Another
function $Y_{3}=x_{3}=T_{ref}-T_{2}$ can be introduced to account
for this error source, which probably will be a systematic error.
The measurement $T$ is then determined from 
\begin{equation}
T=\Delta T+(T_{ref}-T_{2})+T_{2}=Y_{1}(V_{1})+Y_{2}(V_{2})+Y_{3}.
\end{equation}


Suppose that the voltmeter has a precision of $S_{i}$ and a systematic
error of $B_{i}$ when measuring $V_{i}$, and that the random errors
are uncorrelated but the bias errors are always the same (as might
occur for a calibration error). If the only sources of error are these
random and systematic errors and a non-zero value of $Y_{3}$, the
error matrix for the random component of the uncertainty is 
\begin{equation}
H_{r}^{-1}=\left(\begin{matrix}S_{1}^{2}\end{matrix}\right)
\end{equation}
 and the bias component is 
\begin{equation}
H_{s}^{-1}=\left(\begin{matrix}(B_{1}^{2}\end{matrix}\right)
\end{equation}
 when expressed in terms of the fundamental quantities $x_{1}$, $x_{2}$,
and $x_{3}$ representing the two measurements and the unmeasured
difference between $T_{ref}$ and $T_{2}$.

The sum of these matrices can be used in (\ref{eq:refa}) to evaluate
the variance in the measured temperature: 
\begin{equation}
V_{TT}=\begin{pmatrix}a_{1}\end{pmatrix}\begin{pmatrix}S_{1}^{2}+B_{1}^{2}\end{pmatrix}\begin{pmatrix}a_{1}\cr a_{2}\cr1\cr\end{pmatrix}
\end{equation}
\begin{equation}
~~~~~=a_{1}^{2}S_{1}^{2}+a_{2}^{2}S_{2}^{2}+(a_{1}B_{1}+a_{2}B_{2})^{2}+B_{3}^{2}.
\end{equation}
 The first two terms show that the random contributions add to the
net variance in quadrature, as expected for independent error sources.
The next term shows that the bias contributions, however, add linearly.
This results because a bias error affects measurements of $\Delta T$
and $T_{2}$ in the same way, so the error enters the final result
additively. \endexample
\htmlrule


\subsection{Monte Carlo techniques}

Sometimes the functional relationships are so complex or non-linear
that the preceding analytical formulas are unwieldy. In such cases,
an alternative is to employ what is conventionally called a \textit{Monte
Carlo} technique. In this approach, the measured quantities are varied
randomly in ways that represent the experimental uncertainties, and
the calculations leading to the final answer are repeated with these
artificial quantities. This is done repeatedly, and the variances
and covariances in the resulting final answers are calculated. Random
number generators are available on computer systems that generate
variables having zero mean, unity variance, and a Gaussian probability
distribution. Correlated fluctuations can be represented by defining
linear combinations of such independent variables. In cases where
the error propagation is especially complex (e.g, where the final
answer might depend on non-linear fits to the input data), Monte Carlo
techniques may be the only feasible way of determining the uncertainty
in the final result.


\subsection{Reference Material}

This section provides, for reference, specific definitions of some
of the other terms used in the analysis of uncertainty. This information
is included here because this terminology is sometimes in conflict
with nonscientific usage and is not always used consistently even
in scientific papers.

With many of these terms, it is necessary to distinguish between the
characteristics of a \textit{parent} distribution and the estimates
of those characteristics obtained from a specific sample from the
parent population. For example, one may want to estimate characteristics
of the parent population from measurements taken on only a specific
subset from that population. A common convention, followed here, is
to use Greek letters for population characteristics and Roman letters
for sample characteristics. Thus, for example, $\overline{x}$ will
denote the average of a set of measurements, but $\mu$ will denote
an average characteristic of the underlying population.

\textit{Precision} is a measure of reproducibility or scatter in the
results, without regard for the accuracy of the result. It is a measure
of random error only; systematic errors will not affect the precision
of a result, although they do affect the accuracy.

The \textit{mean} of a set of measurements $\{x_{1},x_{2},\dots x_{N}\}$
is the average: 
\begin{equation}
\overline{x}={\frac{{1}}{{N}}}\sum_{i=1}^{N}x_{i}.
\end{equation}
 The \textit{expectation value} of a quantity is the value expected
if averaged over the entire parent population, and will be denoted
by angle brackets: $\langle\rangle$. For example, the mean in the
parent population that corresponds to the sample mean x is 
\begin{equation}
\mu=\langle x\rangle={\textrm{lim}}_{N\rightarrow\infty}(\overline{x}).
\end{equation}


There is an important distinction to be made between the standard
deviation characterizing the random error of a measurement and the
standard deviation characterizing a set of accurate observations and
hence reflecting physical reality. The latter is often encountered
in experimental research, and pertains to the natural variability
in the parameter being measured; e.g., the temperature may really
vary when measured over a path through the atmosphere. The former
represents the precision with which a constant value of that parameter
could be measured in a particular experiment. For example, in experiments
using airborne instrumentation variance spectra for measured variables
seldom show evidence of noise except at low levels that correspond
to digitization noise. This indicates that random measurement errors
seldom contribute significantly to the uncertainty in such a measurement.
However, there usually is high natural variability that causes repeated
sets of measurements in presumably identical conditions to vary significantly,
and the standard deviations among repeated measurements of, for example,
fluxes of water vapor are large. This standard deviation reflects
natural variability, not the random error in the measurement. It results
from the variability of particular samples about the underlying population
mean, and that variability would still characterize measurements from
error-free sensors.

The \textit{median} is the value that divides the population into
equal halves; i.e., half the members lie above and half below the
median. The \textit{most probable value} is that observed most frequently,
sometimes referred to as the \textit{mode} of a distribution. As an
example, the expected distribution of time intervals between randomly
occurring events is 
\begin{equation}
N(t)=N_{0}e^{-t/\tau}
\end{equation}
 where $N(t)$ is the number of events per time interval that occur
at time t, $N_{0}$ is the total number of events, $t$ is the time,
and $\tau$ is a time constant characterizing the process. For this
distribution, the mean time is $\tau$, the median time is $\tau$ln(2),
and the mode occurs for t=0.

A \textit{deviation} $\delta$ is the difference between a specific
measurement or value and the mean. The \textit{standard deviation}
$\sigma$ is the ``root-mean-square'' value of the deviations, obtained
from 
\begin{equation}
\sigma=\Bigl[{\textrm{lim}}_{N\rightarrow\infty}\bigl({\frac{{1}}{{N}}}\sum_{i=1}^{N}(x_{i}-\mu)^{2}\bigr)\Bigr]^{\frac{{1}}{{2}}}.
\end{equation}
 For a sample of measurements, the conventional estimate $s$ of the
population standard deviation $\sigma$ is 
\begin{equation}
s=\Bigl[{\frac{{1}}{{N-1}}}\sum_{i=1}^{N}(x_{i}-\overline{x})^{2}\Bigr]^{\frac{{1}}{{2}}}.
\end{equation}
 The \textit{variance} is the average of the squares of the deviations,
or the square of the standard deviation.

If $x_{j}$ is a possible observation, the observed fraction of observations
having the value $x_{j}$ is $P(x_{j})=N(x_{j})/N$ where $N$ is
the total number of observations and $N(x_{j})$ is the number having
value $x_{j}$. The underlying population distribution function is
then 
\begin{equation}
\Phi(x_{j})={\textrm{lim}}_{N\rightarrow\infty}P(x_{j}).
\end{equation}
 The preceding quantities can then be expressed in terms of the distribution
function; for example, the mean is 
\begin{equation}
\mu=\sum_{j=1}^{N}x_{j}\Phi(x_{j})
\end{equation}
 and the variance is 
\begin{equation}
\sigma^{2}=\sum_{j=1}^{N}(x_{j}-\mu)^{2}\Phi(x_{j})=\Bigl(\sum_{j=1}^{N}x_{j}^{2}\Phi(x_{j})\Bigr)-\mu^{2}=\langle x^{2}\rangle-\mu^{2}.
\end{equation}
 The extensions to continuous distribution functions are these: 
\begin{equation}
\sum_{j}P(x_{j})\rightarrow\int P(x)dx
\end{equation}
\begin{equation}
\mu=\int_{-\infty}^{\infty}x\Phi(x)dx
\end{equation}
\begin{equation}
\sigma^{2}=\int_{-\infty}^{\infty}(x-\mu)^{2}\Phi(x)dx.
\end{equation}
 Similarly, the expectation value for any function $f$ of measurable
characteristics $x$ is 
\begin{equation}
\langle f(x)\rangle=\int f(x)\Phi(x)dx
\end{equation}
 where $x$ can be a set of variables and the multidimensional integration
must then cover all possible values of $x$.

Other characteristics sometimes cited are the \textit{probable error},
the magnitude of the deviation exceeded by 50\% of the deviations,
and the \textit{average deviation,} the expectation value for the
absolute magnitude of the deviations. For a Gaussian distribution,
the probable error, average deviation, and standard deviation have
the ratios 0.674:0.800:1.

If the distribution in measurement errors follows a known probability
distribution, then \textit{confidence intervals} determined from that
distribution can be used to obtain quantitative estimates of probabilities
associated with such errors. It is this relationship that establishes
the often used correspondence between standard deviation and probability,
for the Gaussian distribution. Specifically, measurements falling
more than two standard deviations ($\pm2\sigma$) from the true value
are expected with about 0.05 probability, so $2\sigma$ limits correspond
to approximate limits providing 95\% coverage. Other distribution
functions can be treated in the same way.

% \furtherread
% 
% Abernethy, R. B., and Benedict, 1984:
% 
% Abernethy, R. B., and B. Ringhiser, 1985: The history and statistical
% development of the new ASME-SAE-AIAA-ISO measurement uncertainty methodology.
% 
% Barford, N.. C., 1985: \textit{Experimental Measurements: Precision,
% Error, and Truth.} John Wiley and Sons, New York, 159 pp.
% 
% Beers, Yardley, 1957: \textit{Introduction to the Theory of Error.}
% Addison-Wesley. Reading, Massachusetts, 66 pp.
% 
% Taylor, J. R., 1982: An Introduction to Error Analysis. University
% Science Books, Mill Valley, California, 270 pp. \endfurther
% {}
% 

\section{Appendix: Reproducibility}

This document is constructed in ways that support duplication of the
study. The processing programs are incorporated into the same file
that generates this document, using principles and techniques described
by \citet{Xie2014a} as implemented in the R package 'knitr' (\citet{Xie2014b}).
The core program, 'WindUncertainty.Rnw', is archived on 'GitHub' in
the directory at \href{https://github.com/WilliamCooper/WindUncertainty.git}{this URL}.
There is some supplemental material in that directory, like the bibliography
and many code segments saved in the 'chunks' subdirectory, so the
full directory should be downloaded in order to run the program. The calculations
use the programming language R (\citet{Rlanguage}) and were run within RStudio
(\citet{RStudio2012}), so this is the most straightforward way to replicate the
calculations and the generation of this document.

A package named Ranadu, containing auxillary functions, is used extensively
in the R code. It is
available on GitHub as \href{https://github.com/WilliamCooper/Ranadu.git}{https://github.com/WilliamCooper/Ranadu.git}. The specific version used for calculations in this
report is included in the 'zip' archive listed below.

The data files used are also preserved in the NCAR High Performance
Storage System (HPSS) in files that are available, and they can be
provided via a request to \url{mailto:mailto:raf-dm@eol.ucar.edu}.
The original files representing the data as produced by the NCAR Earth
Observing Laboratory, Research Aviation Facility, were in netCDF format
(cf.~\href{http://www.unidata.ucar.edu/software/netcdf/}{this URL}),
but in many cases special reprocessing was used and the files may
change after reprocessing so a separate archive is maintained for
this document. The data files in this archive represent R data.frames
and are preserved as binary-format 'Rdata' files via R 'save' commands.
The code in the GitHub archive has appropriate 'load' commands to
ingest these data files from a subdirectory named 'Data' but this
is not part of the GitHub repository because it is too large to be
appropriate there. To reproduce this research, those data files have
to be transferred separately from the NCAR HPSS to the 'Data' directory. 

Extensive use has been made of attributes assigned to the data.frames
and the variables in those data.frames. All the attributes from the
original netCDF files have been transferred, so there is a record
of how the original data were processed, for example recording calibration
coefficients and processing chains for the variables. Once the data.frames
are loaded into R, these attributes can be viewed and provide additional
documentation of what data were used. Key information like the processing
date, the program version, and the selection of primary variables
is thus preserved.

\begin{tabular}{ll}
\textsf{\textsc{\textcolor{blue}{Project:}}} & WindUncertainty\tabularnewline
\textsf{\textsc{\textcolor{blue}{Archive package:}}} & WindUncertainty.zip\tabularnewline
\textsf{\textsc{\textcolor{blue}{Contains:}}} & attachment list below\tabularnewline
\textsf{\textsc{\textcolor{blue}{Program:}}} & WindUncertainty.Rnw\tabularnewline
\textsf{\textsc{\textcolor{blue}{Original Data:}}} & /scr/raf\_data/DEEPWAVE/ \tabularnewline
\textsf{\textsc{\textcolor{blue}{Git:}}} & https://github.com/WilliamCooper/WindUncertainty.git\tabularnewline
\end{tabular}

\attachm{WindUncertainty.Rnw\\WindUncertainty.pdf\\chunks/*\\SessionInfo\\Ranadu\_2.1-15-3-8.tar.gz}


\section*{Acknowledgments}
\label{sec:acknowledgements}

The analyses reported here were mostly performed using R%
\footnote{R Core Team (2014). R: A language and environment for statistical
computing. R Foundation for Statistical Computing, Vienna, Austria.
URL http://www.R-project.org/.%
}\citet{Rlanguage}, with RStudio%
\footnote{RStudio (2012). RStudio: Integrated development environment for R
(Version 0.98.879) {[}Computer software{]}. Boston, MA. Available
from http://www.rstudio.org/%
} \citet{RStudio2012} and knitr%
\footnote{Xie, Y. (2013), ``knitr: A general-purpose package for dynamic report
generation in R. R package version 1.3,'' Version 1.6 was used for
this work. See also Xie, Y (2014), ``Dynamic documents with R and
knitr,'' CRC Press, Chapman and Hall, 190 pp.%
} \citet{Xie2014a,Xie2014b}. Substantial use also was made of the
ggplot2 package%
\footnote{H. Wickham. ggplot2: elegant graphics for data analysis. Springer, 
New York, 2009. %
} \citet{wickham2009} for R.

\bibliographystyle{plainnat}
\label{sec:bibliography}
\bibliography{WAC}

\end{document}
